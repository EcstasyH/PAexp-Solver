%%%%%%%%%%%%%%%%%%%%%%%%%%%%%%%%%%%%%%%%%%%%%%%%%%%%%%%%%%%%%%%%%%%%%
%%                                                                 %%
%% Please do not use \input{...} to include other tex files.       %%
%% Submit your LaTeX manuscript as one .tex document.              %%
%%                                                                 %%
%% All additional figures and files should be attached             %%
%% separately and not embedded in the \TeX\ document itself.       %%
%%                                                                 %%
%%%%%%%%%%%%%%%%%%%%%%%%%%%%%%%%%%%%%%%%%%%%%%%%%%%%%%%%%%%%%%%%%%%%%

%%\documentclass[referee,sn-basic]{sn-jnl}% referee option is meant for double line spacing

%%=======================================================%%
%% to print line numbers in the margin use lineno option %%
%%=======================================================%%

%%\documentclass[lineno,sn-basic]{sn-jnl}% Basic Springer Nature Reference Style/Chemistry Reference Style

%%======================================================%%
%% to compile with pdflatex/xelatex use pdflatex option %%
%%======================================================%%

%%\documentclass[pdflatex,sn-basic]{sn-jnl}% Basic Springer Nature Reference Style/Chemistry Reference Style

%%\documentclass[sn-basic]{sn-jnl}% Basic Springer Nature Reference Style/Chemistry Reference Style
\documentclass[sn-mathphys]{sn-jnl}% Math and Physical Sciences Reference Style
%%\documentclass[sn-aps]{sn-jnl}% American Physical Society (APS) Reference Style
%%\documentclass[sn-vancouver]{sn-jnl}% Vancouver Reference Style
%%\documentclass[sn-apa]{sn-jnl}% APA Reference Style
%%\documentclass[sn-chicago]{sn-jnl}% Chicago-based Humanities Reference Style
%%\documentclass[sn-standardnature]{sn-jnl}% Standard Nature Portfolio Reference Style
%%\documentclass[default]{sn-jnl}% Default
%%\documentclass[default,iicol]{sn-jnl}% Default with double column layout

%%%% Standard Packages
%%<additional latex packages if required can be included here>
\usepackage{amsmath}
\usepackage{amssymb}
\usepackage{graphicx}
\usepackage{amsfonts}
\usepackage{verbatim}
\usepackage{times}
\usepackage{multirow}
\usepackage{bbm}
\usepackage{amsthm}
\usepackage{stmaryrd}
%%%%

%%%%%=============================================================================%%%%
%%%%  Remarks: This template is provided to aid authors with the preparation
%%%%  of original research articles intended for submission to journals published 
%%%%  by Springer Nature. The guidance has been prepared in partnership with 
%%%%  production teams to conform to Springer Nature technical requirements. 
%%%%  Editorial and presentation requirements differ among journal portfolios and 
%%%%  research disciplines. You may find sections in this template are irrelevant 
%%%%  to your work and are empowered to omit any such section if allowed by the 
%%%%  journal you intend to submit to. The submission guidelines and policies 
%%%%  of the journal take precedence. A detailed User Manual is available in the 
%%%%  template package for technical guidance.
%%%%%=============================================================================%%%%

\jyear{2021}%

%% as per the requirement new theorem styles can be included as shown below
\theoremstyle{thmstyleone}%
\newtheorem{theorem}{Theorem}%  meant for continuous numbers
%%\newtheorem{theorem}{Theorem}[section]% meant for sectionwise numbers
%% optional argument [theorem] produces theorem numbering sequence instead of independent numbers for Proposition
\newtheorem{proposition}[theorem]{Proposition}% 
%%\newtheorem{proposition}{Proposition}% to get separate numbers for theorem and proposition etc.

\theoremstyle{thmstyletwo}%
\newtheorem{example}{Example}%
\newtheorem{lemma}{Lemma}%
\newtheorem{remark}{Remark}%

\theoremstyle{thmstylethree}%
\newtheorem{definition}{Definition}%

% Notations
\newcommand{\parseInt}{\textsf{parseInt}}
\newcommand{\pa}{{\sf PA}}
\newcommand{\qfpa}{{\sf QFPA}}
\newcommand{\paexp}{{\sf ExpPA}}
\newcommand{\Def}{\hat{=}}
\newcommand{\strint}{{\sf STR}_\parseInt}

\newcommand{\Nat}{\mathbb{N}}
\newcommand{\Int}{\mathbb{Z}}
\newcommand\Aut{\mathcal{A}}
\newcommand\Lang{\mathcal{L}}
\newcommand\free{\mathrm{Free}}
\newcommand\op{\odot}
\newcommand\parikh{\mathbb{P}}
\newcommand{\hide}[1]{}
\newcommand\encode{\mathit{encode}}
\newcommand\decode{\mathit{decode}}
\newcommand\len{\textsf{len}}
\newcommand\svars{\mathcal{X}}
\newcommand\ivars{\mathbbm{X}}
\newcommand\pos{\mathit{pos}}
\newcommand\ltrue{\mathtt{true}}
\newcommand\lfalse{\mathtt{false}}
\newcommand\flatdom{\mathcal{R}}
\newcommand\flatten{\mathit{flatten}}
\newcommand\svarx{x}
\newcommand\svary{y}
\newcommand\ivarx{\mathbbm{x}}
\newcommand\ivary{\mathbbm{y}}
\newcommand\ivarz{\mathbbm{z}}
%\newcommand\concat{\cdot}
\newcommand\sterm{t}
\newcommand\iterm{\mathbbm{t}}
\newcommand{\kell}{{(k,\ell)}}

\newcommand{\svar}{\textsf{SVar}}
\newcommand{\ivar}{\textsf{IVar}}
\newcommand{\pvar}{\textsf{PVar}}
\newcommand{\flatfun}{\mathcal{F}}


\newcommand{\znj}[1]{\color{red} {NJ: #1 :JN} \color{black}}
\newcommand{\znjSide}[1]{\todo[color=orange!10]{\textbf{ZHAN:} #1}}
\newcommand{\zhilin}[1]{\color{brown} {ZL: #1 :LZ} \color{black}}
\newcommand{\wuhao}[1]{\color{cyan} {WH: #1 :HW} \color{black}}
\newcommand{\yfc}[1]{\color{violet} {YF: #1 :FY} \color{black}}




\raggedbottom
%%\unnumbered% uncomment this for unnumbered level heads

\begin{document}

\title[Article Title]{A Decision Procedure for String Constraints with String-Integer Conversion and Flat Regular Constraints}

%%=============================================================%%
%% Prefix	-> \pfx{Dr}
%% GivenName	-> \fnm{Joergen W.}
%% Particle	-> \spfx{van der} -> surname prefix
%% FamilyName	-> \sur{Ploeg}
%% Suffix	-> \sfx{IV}
%% NatureName	-> \tanm{Poet Laureate} -> Title after name
%% Degrees	-> \dgr{MSc, PhD}
%% \author*[1,2]{\pfx{Dr} \fnm{Joergen W.} \spfx{van der} \sur{Ploeg} \sfx{IV} \tanm{Poet Laureate} 
%%                 \dgr{MSc, PhD}}\email{iauthor@gmail.com}
%%=============================================================%%

\author[1,2]{Hao Wu}\email{wuhao@ios.ac.cn}

\author[3]{Yu-Fang Chen}\email{yfc@iis.sinica.edu.tw}

\author[1,2]{Zhilin Wu}\email{wuzl@ios.ac}

\author*[1,2]{Naijun Zhan}\email{znj@ios.ac.cn}

\affil*[1]{State Key Laboratory of Computer Science,
Institute of Software, Chinese Academy of Sciences, Beijing, China}

\affil[2]{University of Chinese Academy of Sciences,
Beijing, China}

\affil[3]{Institute of Information Science, Academia Sinica, Taiwan, Republic of China}


%%==================================%%
%% sample for unstructured abstract %%
%%==================================%%

\abstract{String constraint solving is the core of various testing and verification approaches for scripting languages. 
Among algorithms for solving string constraints, flattening is a well-known approach that is particularly useful in handling satisfiable instances.
As string-integer conversion is an important function appearing in almost all scripting languages, Abdulla et al. extended the flattening approach to this function recently.
However, their approach supports only a special flattening pattern and leaves the support of the general flat regular constraints as an open problem.
In this paper, we fill the gap and propose a complete flattening approach for the string-integer conversion. The approach is built upon a quantifier elimination procedure for the linear-exponential arithmetic (namely, the extension of Presburger arithmetic with exponential functions) proposed by Point in 1986. The complexity of the quantifier elimination procedure is analyzed for the first time and is shown to be 3-EXPSPACE if the formula contains only existential quantifiers. While the quantifier elimination procedure by Point is too expensive to be implemented efficiently, we propose various optimizations and provide a prototypical implementation.
We evaluate the performance of our implementation on the benchmarks that are generated from the string hash functions as well as randomly.
The experimental results show that our implementation outperforms the state-of-the-art solvers. 
}

%%================================%%
%% Sample for structured abstract %%
%%================================%%

% \abstract{\textbf{Purpose:} The abstract serves both as a general introduction to the topic and as a brief, non-technical summary of the main results and their implications. The abstract must not include subheadings (unless expressly permitted in the journal's Instructions to Authors), equations or citations. As a guide the abstract should not exceed 200 words. Most journals do not set a hard limit however authors are advised to check the author instructions for the journal they are submitting to.
% 
% \textbf{Methods:} The abstract serves both as a general introduction to the topic and as a brief, non-technical summary of the main results and their implications. The abstract must not include subheadings (unless expressly permitted in the journal's Instructions to Authors), equations or citations. As a guide the abstract should not exceed 200 words. Most journals do not set a hard limit however authors are advised to check the author instructions for the journal they are submitting to.
% 
% \textbf{Results:} The abstract serves both as a general introduction to the topic and as a brief, non-technical summary of the main results and their implications. The abstract must not include subheadings (unless expressly permitted in the journal's Instructions to Authors), equations or citations. As a guide the abstract should not exceed 200 words. Most journals do not set a hard limit however authors are advised to check the author instructions for the journal they are submitting to.
% 
% \textbf{Conclusion:} The abstract serves both as a general introduction to the topic and as a brief, non-technical summary of the main results and their implications. The abstract must not include subheadings (unless expressly permitted in the journal's Instructions to Authors), equations or citations. As a guide the abstract should not exceed 200 words. Most journals do not set a hard limit however authors are advised to check the author instructions for the journal they are submitting to.}

\keywords{String-integer conversion, Flat regular constraints, Exponential function, Presburger arithmetic, Quantifier elimination}

%%\pacs[JEL Classification]{D8, H51}

%%\pacs[MSC Classification]{35A01, 65L10, 65L12, 65L20, 65L70}

\maketitle

\section{Introduction}\label{sec:intro}

solve string constraint is hard

strategy: do unsat and sat separately

strategy: using different procedure to (dis)prove validaity

for disprove validatity, there are two approches, first is bound string length

cannot handle $x.y \neq z  \wedge |x| > 2000$

more recent approach is flattening

it is known that word equaltion + flat regular constraints + len constraints
is decidable

it was unknown that whether word equations + flat regular constraints + len constraints + parseInt is decidable



\section{Preliminary}\label{sec:pre}
%!TEX root = paper.tex

%In this section, we introduce some basic concepts and theories that will be used later. 
%\subsection{Basic Concepts}

In this section, we fix the notations and introduce some basic concepts, including Presburger arithmetic, finite-state automata, and flat languages.

\paragraph{Integers, strings, and languges}
Let $\Nat$ denote the set of natural numbers, $\Int$ denote the set of integers, and $\Int^+$ denote the set of positive integers. For $n \in \Int^+$, let $[n]$ denote $\{1,\dots,n\}$.

An \emph{alphabet} $\Sigma$ is a finite set. Elements of $\Sigma$ are called \emph{letters}.
%Let $\Sigma$ denote a finite alphabet,
A \emph{string} $w$ over $\Sigma$ is a (possibly empty) finite sequence $a_1\ldots a_n$ with $a_i \in \Sigma$ for every $i \in [n]$. Let $\varepsilon$ denote the empty string, namely, the empty sequence. For a string $w = a_1 \ldots a_n \in \Sigma^*$, let $|w|$ denote the \emph{length} of $w$, i.e. $n$. In particular, $|\varepsilon| = 0$.
%
For $w_1 = a_1 \ldots a_m, w_2 = b_1 \ldots b_n\in \Sigma^*$, 
let $w_1\cdot w_2$ denote the \emph{concatenation} of $w_1$ and $w_2$, that is, $a_1 \ldots a_m b_1 \ldots b_n$.
%
Let $\Sigma^*$ denote the set of all strings over $\Sigma$ and $\Sigma^+$ denote the set of nonempty strings over $\Sigma$. 
For convenience, we also use $\Sigma_{\epsilon}$ to denote $\Sigma \cup \{\epsilon\}$.
%
A language $L$ over $\Sigma$ is a subset of $\Sigma^*$.

\paragraph{Presburger Arithmetic} \label{PA}
A Presburger Arithmetic ({\pa}) formula 
% is a first order theory over the signature 
%$\Sigma_\mathbb{N}\Def  \{0,1,+, <, |\}$. 
 is defined by the rules 
 $$
 \begin{array}{ l c l}
 \iterm &\Def & c \mid \ivarx \mid \iterm+\iterm \mid \iterm - \iterm, \\
 \phi & \Def & \iterm \ \op\ \iterm \mid c | \iterm \mid \phi \wedge \phi \mid \phi \vee \phi \mid \neg \phi \mid \exists \ivarx.\ \phi \mid \forall \ivarx.\ \phi,

 \end{array}
 $$
where $\op \in \{=, < , >, \le, \ge\}$ and $\ivarx, c$ are integer variables and constants, respectively. A \emph{quantifier-free} {\pa} (\qfpa) formula is a {\pa} formula containing no quantifiers. A {\pa} formula is in \emph{negation normal form} (NNF) if all occurrences of the negation symbol $\neg$ are before the atomic formulas. A {\pa} formula is called \emph{existential} if it is in NNF and contains no occurrences of universal quantifiers.
%An \emph{existential} LIA (ELIA) formula is an LIA formula where there are no universal quantifiers and all existential quantifiers are in the scope of an even number of the negation symbols $\neg$. 
%
The set of free variables of $\phi$, denoted by $\free(\phi)$, is defined in a standard manner. 
We usually write $\phi(\ivarx_1,\cdots, \ivarx_k)$ to denote an PA formula $\phi$ such that $\free(\phi) \subseteq \{\ivarx_1,\cdots, \ivarx_k\}$. Given an PA formula $\phi$, and an integer interpretation of $\free(\phi)$, i.e. a function $I: \free(\phi) \rightarrow \Int$, we denote by $I \models \phi$ that $I$ satisfies $\phi$ (which is defined in the standard manner, with $+$, $-$, and $|$ interpreted as the integer addition, subtraction, and divisibility relation respectively), and call $I$ a \emph{model} of $\phi$. We use $\llbracket \phi \rrbracket$ to denote the set of models of $\phi$. 


%%%%%%%%%%%%%%%%%%%%%%%%%%%%%%%%%%%%%%%%%%%
%%%%%%%%%%%%%%%%%%%%%%%%%%%%%%%%%%%%%%%%%%%
\hide{
Presburger Arithmetic (PA) is a first order theory over the signature 
$\Sigma_\mathbb{N}\Def  \{0,1,+, <, |\}$, where $0,1$ are constants, 
$+$ is a binary function symbol, $<, |$ are  and $=$ is a binary predicate.

PA can be axiomatized by the following axioms \cite{PA} 

\begin{itemize}
    \item $\forall x, \neg (x+1=0)$
    \item $\forall x \forall y. x+1=y+1 \to x=y$
    \item $F(0) \wedge (\forall x. F(x)\to F(x+1)) \to \forall x. F(x)$ 
    \item $\forall x. x+0=x$
    \item $\forall x \forall y. x+(y+1)=(x+y)+1$
\end{itemize}
Given the domain $\mathbb{N}$,
the standard interpretation of PA interprets 
$0,1$ to $0_\mathbb{N},1_\mathbb{N}\in \mathbb{N}$
and $+,=$ to addition and equality over $\mathbb{N}$.
We call a PA formula without quantifiers a quantifier-free PA formula.

PA is a decidable theory, 
and the complexity of decidability is related to 
the number and locations of quantifiers.
Generally, 
the upper bound (on deterministic time and space) 
for deciding a formula of length $n$ is $2^{2^{2^{p n log(n)}}}$,
where $p>1$ is a constant\cite{Oppen69}. 
}
%%%%%%%%%%%%%%%%%%%%%%%%%%%%%%%%%%%%%%%%%%%
%%%%%%%%%%%%%%%%%%%%%%%%%%%%%%%%%%%%%%%%%%%

\paragraph{Finite state automata}
A finite state automaton (FA) is a tuple 
$\Aut=\langle Q,\Sigma,\Delta, q_{\textit{init}}, F\rangle$, 
where $Q$ is a finite set of states, 
$\Sigma$ is a finite alphabet,
$\Delta\subseteq Q\times \Sigma_\epsilon\times Q$ is the transition relation, 
$q_{\textit{init}}$ is the initial state, $F \subseteq Q$ is the set of accepting states. 
A \emph{run} of $\Aut$ on a string $w = a_1 \ldots a_n$ is a sequence $q_0 \xrightarrow{b_1} q_1  \xrightarrow{b_2}  \ldots   \xrightarrow{b_{m-1}}  q_{m-1} \xrightarrow{b_m} q_m$ such that $q_0 = q_{\textit{init}}$, $(q_{i-1}, b_i, q_i) \in \Delta$ for every $i \in [m]$, and $a_1 \ldots a_n = b_1 \ldots b_m$. A \emph{run} $q_0 \xrightarrow{b_1} q_1  \xrightarrow{b_2}  \ldots   \xrightarrow{b_{m-1}}  q_{m-1} \xrightarrow{b_m} q_m$ is \emph{accepting} if $q_m  \in F$. A string $w$ is \emph{accepted} by $\Aut$ if there is an accepting run of $\Aut$ on $w$. Let $\Lang(\Aut)$ denote the set of strings accepted by $\Aut$. A language $L \subseteq \Sigma^*$ is \emph{regular} if it can be defined by some FA $\Aut$, namely, $L = \Lang(\Aut)$.


%%%%%%%%%%%%%%%%%%%%%%%%%%%%%%%%%%%%%%%%%%%%%%%%%%%%%%
\hide{
\paragraph{Parikh Image}
%Given an alphabet $\Sigma$ and a string $w\in \Sigma^*$, 
%we define the set of Parikh variables 
%$\Sigma^\# \Def \{a^\# \mid a\in \Sigma\}$.
%The Parikh image of $w$ is a function 
%$\mathbb{P}(w): \Sigma^\# \mapsto \mathbb{N}$,
%which maps each symbol $a^\#\in \Sigma^\#$ to the number of occurrences
%of $a$ in $w$.
%
The \emph{Parikh image} of a word $w \in \Sigma^*$, denoted by $\parikh(w)$, maps each Parikh (integer) variable $\#a$, where $a \in \Sigma$, to the number of occurrences of $a$ in $w$. 
For instance, let $\Sigma = \{a,b\}$ and $w= aabba$,
then $\parikh(w)(\#a)=3$ and $\parikh(w)(\#b)=2$.
Given an alphabet $\Sigma$, let $\#\Sigma$ denote the set of Parikh variables $\{\#a | a \in \Sigma\}$. 
%The Parikh image of $w$ is a function $\parikh(w): \#\Sigma \rightarrow \mathbb{N}$ such that $\parikh(w)(\#a) = |w|_a$, for each $a \in \Sigma$. 
The Parikh image of a language $L$ is defined as $\parikh(L) = \{\parikh(w) \mid w \in L\}$. It is well known that the Parikh image of a regular language can be characterized by an existential PA formula~\cite{SSMH04}.
}
%%%%%%%%%%%%%%%%%%%%%%%%%%%%%%%%%%%%%%%%%%%%%%%%%%%%%%

%For a language $L\subseteq \Sigma^*$, 
%define the Parikh image of $L$ to be 
%$\mathbb{P}(L)\Def \{\mathbb{P}(w) | w\in L\}$.
%We say a language $L$ is \emph{Parikh-definable} 
%if $\mathbb{P}(L)$ can be characterized by a quantifier-free PA formula over $\Sigma^\#$, 
%where $a^\#$ in the formula encodes the number of occurrences of $a$.
%It is well known that
%any context-free language (therefore regular language) 
%is Parikh definable \cite{Parikh66}.




%\paragraph{Flat automata and languages}

%Given a string constraint $\Psi$,
%the general problem of deciding whether $\lVert \Psi \rVert$ is empty is undecidable.
%However, 
%the problem becomes decidable when certain restrictions are imposed.
%One of the restriction is by flat automata and flat languages, 
%defined below. 

\paragraph{Flat languages}


We will present flat languages. This is a high level, language-theoretical view at the flat automata from \cite{Parosh:20:PLDI}. 
%
%For integers $k$ and $\ell$ and a string variable $x$, 
%we define the family of indexed \emph{character variables} $\charvars = \{x_j^i \mid 1 \le i \le k, 1 \le j \le \ell\}$. 
A \emph{flat language} (FL)  over $\Sigma$
%with the \emph{period} at most $\ell$ and the \emph{cycle count} $k$ 
is the set of strings 
that conform to a regular expression of the form
$
(a^1_1\ldots a^1_{\ell_1})^* \cdot \ldots \cdot (a^k_1\ldots a^k_{\ell_k})^*, 
$
where $a^i_j \in \Sigma$ for each $i \in [k]$ and $j \in [\ell_i]$.
Intuitively, an FL is a set of strings consisting of $k$ consecutive parts, each created by iterating a \emph{cycle}, a string $a^i_1\ldots a^i_{\ell_i}$ of $\ell_i$ letters. 
%A $\langle \ell, k \rangle$-FL is an FL such that its \emph{period} is at most $\ell$ and the \emph{cycle count} is at most $k$.
For instance, the language defined by $(ab)^*(a)^*(bb)^*$ is an FL.

%A central property of FL is that the strings accepted by an FL are fully characterized by their Parikh images. 



%%%%%%%%%%%%%%%%%%%%%%%%%%%%%%%%%%%%%%%
%%%%%%%%%%%%%%%%%%%%%%%%%%%%%%%%%%%%%%%
\hide{
For a fixed alphabet $\Sigma$,
we say a language $L$ over $\Sigma$ to be \emph{$\langle p,q \rangle$-flat} if 
there exist strings $w_1,...,w_q \in \Sigma^*$ such that
$|w_i|\le p$ for all $i:1\le i \le q$ 
and $L = (w_1)^*(w_2)^*...(w_q)^*$. 
We use $\alpha$ to denote $\langle p,q \rangle$, 
and call it the \emph{abstraction parameter} of $L$.
Intuitively,
a flat language with abstraction parameter $\alpha = \langle p,q \rangle$
consists of $q$ loops and the length of each loop body is equal or less than $p$.
For example,
$L = (ab)^*(a)^*(bb)^*$ is a $\langle 2,3 \rangle$-flat language.

Flat automata are a special form of finite state automata that 
recognize flat languages.
Fix the abstraction parameter $\alpha=\langle p,q\rangle$,
a flat automaton consists of $q$ loops,
each loop is a circle of $p$ states.
Formally, 
an $\alpha$-flat automaton contains $p q$ states at most,
and we name the states from $1$ to $p q$,
$1$ is the initial state and $(p q - p + 1)$ is the accepting state.
We use $\cdot$ as a placeholder for some symbol in $\Sigma_\epsilon$,
the transition relations of state $i$ are defined as 
\begin{itemize}
    \item if $i\  \text{mod}\  p = 1$ and $i \neq pq-p+1$, then 
    $(i,\epsilon,i+p)\in \Delta$,
    $(i, \cdot ,i+1) \in \Delta$;
    \item if $i\  \text{mod}\  p = 0$, then 
    $(i,\cdot , i-p+1) \in \Delta$;
    \item otherwise, $(i,\cdot, i+1) \in \Delta$.
\end{itemize}

\begin{figure}[ht]
    \centering 
    \begin{tikzpicture}
        \node[state,           ] (4) {$4$};
        \node[state, left  of=4] (2) {$2$};
        \node[state, right of=4] (6) {$6$};
        \node[state, initial, below of=2] (1) {$1$};
        \node[state, below of=4] (3) {$3$};
        \node[state, accepting, right of=3] (5) {$5$};
        
        \draw 
        (1) edge[above] node{$\epsilon$} (3)
        (3) edge[above] node{$\epsilon$} (5)
        
        (1) edge[bend left,left] node{$a$} (2)
        (2) edge[bend left,right] node{$b$} (1)
        
        (3) edge[bend left,left] node{$a$} (4)
        (4) edge[bend left,right] node{$\epsilon$} (3)
        
        (5) edge[bend left,left] node{$b$} (6)
        (6) edge[bend left,right] node{$b$} (5);
    \end{tikzpicture}
    \caption{A $\langle 2,3 \rangle$-flat automaton 
that recognizes $L\Def  (ab)^*(a)^*(bb)^*$}
    \label{fig: FA}
\end{figure}
} 
 %%%%%%%%%%%%%%%%%%%%%%%%%%%%%%%%%%%%%%%%
  %%%%%%%%%%%%%%%%%%%%%%%%%%%%%%%%%%%%%%%%
\hide{
\paragraph{Generic Flat Languages and Automata}
Fix $\alpha = \langle p,q \rangle$,
we define the \emph{generic $\alpha$-flat language} is the union of all $\alpha$-flat languages, denoted by $\mathbb{F}(\alpha)$.
Now, we try to define an automaton that recognizes all $\alpha$-flat languages,
i.e., collects all behaviors of $\alpha$-flat automata.

Intuitively, 
the generic automaton is obtained by introducing a new alphabet (
a multi-set with $p q$ copies of the original alphabet) and 
adding more transitions (labels),
the states and the overall framework remain unchanged. 
In details, a generic $\alpha$-flat automaton is still a finite state automaton over
$\Sigma(\alpha)\Def \{a(i)| (a\in \Sigma_\epsilon) \wedge i\in \mathbb{N}:1\le i \le pq\}\cup \{\epsilon\}$.
The states are still named from $1$ to $pq$, 
the initial state is $1$ and the accepting state is $(pq-p+1)$.
The transition relations for state $i$ are defined as 
\begin{itemize}
    \item if $i\  \text{mod}\  p = 1$ and $i\neq pq-p+1$, then 
    $(i,\epsilon,i+p)\in \Delta$
    and $\forall s\in \Sigma_{\epsilon}. (i, s(i) ,i+1) \in \Delta$;
    \item if $i\  \text{mod}\  p = 0$, 
    $\forall s \in \Sigma_{\epsilon}. (i,s(i), i-p+1) \in \Delta$;
    \item otherwise, $\forall s \in \Sigma_{\epsilon}. (i,s(i), i+1) \in \Delta$.
\end{itemize}

For $\Sigma = \{a,b\}$, an example of generic $\langle 2,3 \rangle$-flat automaton is shown in figure (\ref{fig: GFA}).

\begin{figure}[ht]
    \centering 
    \begin{tikzpicture}
        \node[state,           ] (4) {$4$};
        \node[state, left  = 2cm of 4] (2) {$2$};
        \node[state, right = 2cm of 4] (6) {$6$};
        \node[state, initial, below of=2] (1) {$1$};
        \node[state, below of=4] (3) {$3$};
        \node[state, accepting, below of=6] (5) {$5$};
        
        \draw 
        (1) edge[above] node{$\epsilon$} (3)
        (3) edge[above] node{$\epsilon$} (5)
        
        (1) edge[bend left, pos =0.2 ,left] node{$a(1)$} (2)
        (1) edge[bend left, pos =0.5 ,left] node{$b(1)$} (2)
        (1) edge[bend left, pos =0.8 ,left] node{$\epsilon(1)$} (2)
        
        (2) edge[bend left, pos = 0.2 ,right] node{$\epsilon(2)$} (1)
        (2) edge[bend left, pos = 0.5 ,right] 
        node{$b(2)$} (1)
        (2) edge[bend left, pos = 0.8 ,right] 
        node{$a(2)$} (1)
        
        (3) edge[bend left, pos =0.2 ,left] node{$a(3)$} (4)
        (3) edge[bend left, pos =0.5 ,left] node{$b(3)$} (4)
        (3) edge[bend left, pos =0.8 ,left] node{$\epsilon(3)$} (4)
        
        (4) edge[bend left, pos = 0.2 ,right] node{$\epsilon(4)$} (3)
        (4) edge[bend left, pos = 0.5 ,right] 
        node{$b(4)$} (3)
        (4) edge[bend left, pos = 0.8 ,right] 
        node{$a(4)$} (3)
        
        (5) edge[bend left, pos =0.2 ,left] node{$a(5)$} (6)
        (5) edge[bend left, pos =0.5 ,left] node{$b(5)$} (6)
        (5) edge[bend left, pos =0.8 ,left] node{$\epsilon(5)$} (6)
        
        (6) edge[bend left, pos = 0.2 ,right] node{$\epsilon(6)$} (5)
        (6) edge[bend left, pos = 0.5 ,right] 
        node{$b(6)$} (5)
        (6) edge[bend left, pos = 0.8 ,right] 
        node{$a(6)$} (5);
    \end{tikzpicture}
    \caption{The generic $\langle 2,3 \rangle$-flat automaton}
    \label{fig: GFA}
\end{figure}
 

However,
the resulted automaton may accept languages that are not in $\mathbb{F}(\alpha)$,
because in different passes inside a loop, 
the automaton can choose different symbols between identical pairs. 
To avoid this problem,
we add a so-called purity condition on the accepting language of generic flat automata,
which is equivalent to intersecting the language of a generic flat automaton 
with a language that encodes the purity condition.

We say a string $w\in (\Sigma(\alpha))^*$ is pure if for all $i: 1\le i \le p q$,
and $a,b\in \Sigma$, 
$a\neq b \wedge \#w(a(i))>0$ implies $\#w(b(i))=0$.
Formally, the purity condition is defined by 
\begin{equation} \label{eq:purity}
 \bigwedge_{1\le i\le pq}\bigwedge_{a,b\in \Sigma, a\neq b} ({a(i)}^\#>0)\to ({b(i)}^\#=0)\, . 
\end{equation}

We denote the accepting language of the generic $\alpha$-flat automaton by $\mathbb{G}(\alpha)$.
Note that $\mathbb{G}(\alpha)$ is a language over $\Sigma_\alpha$,
but what we want is a language over $\Sigma$.
So we define a renaming function $R:\Sigma(\alpha)\mapsto \Sigma$ such that for all $a(i) \in \Sigma_\alpha, R(a(i))=a$,
and $R(\epsilon) = \epsilon$.
Define $\mathbb{G}'(\alpha) \, \Def \, \{R(w) \mid w\in \mathbb{G}(\alpha)\}$, 
for simplicity, we write
$\mathbb{G}'(\alpha)=R(\mathbb{G}(\alpha))$.

The important feature of generic flat autamata
is that every word $w\in \mathbb{G}(\alpha)$ is uniquely determined by its Parikh image $\mathbb{P}(w)$.
}



\section{Flattening string constraints with string-integer conversion}\label{sec:string-solving}
%!TEX root = paper.tex

In this section, we first define the class of string constraints with string-integer conversion, denoted by $\strint$. Then we define the extension of Presburger arithmetic with exponential functions, denoted by $\paexp$. Finally, we show how to flatten the string constraints in $\strint$ into the arithmetic constraints in $\paexp$.

\subsection{String constraints with string-integer conversion ($\strint$)}

In the sequel, we shall define $\strint$, the class of string constraints with the string-integer conversion function {\parseInt}.

The function  {\parseInt} takes a decimal string as the input and returns the integer represented by the string\footnote{The {\parseInt} function in scripting languages e.g. Javascript is more general in the sense that the base can be a number between 2 and 36. Although our approach works for arbitrary positive bases, we choose to focus on the base 10 in this paper, for readibility.}.
%Since we focus on the decidability,
%we define a binary version of {\parseInt},
%which takes a binary string and returns a decimal integer number.
For example,
${\parseInt}('0123') = {\parseInt}('123')=10^2+10*2+3 = 123$. 
Note here we use the quotation marks to delimit the strings, 
%Clearly,  our decision procedure given in this paper 
%can be adapted to string constraints with other string to number conversion function without substantial change. 
%Single quotation marks are used to distinguish a symbol like $'1'\in \Sigma$
%from a number $1_\mathbb{N}\in \mathbb{N}$ when needed.

Formally, the semantics of the $\parseInt$ function is defined as follows. 
%\begin{definition} 
Let $\Sigma_{\textit{num}}=\{0,1, \ldots, 9\}$. Then ${\parseInt}: \Sigma_{\textit{num}}^+ \mapsto \Nat$ is recursively defined by
    for every $w\in \Sigma_{\textit{num}}^+$,
    \begin{itemize}
%        \item if $|w|=0$, i.e., $w=\epsilon$,  ${\parseInt}(w)=0$;
        \item if $w={'i'}$ for $i \in \Sigma_{\textit{num}}$, then ${\parseInt}('i')=i$;
        \item for $w = w_1 {'i'}$ for $i \in \Sigma_{\textit{num}}$ with $|w_1| \ge 1$, 
        ${\parseInt}(w) = 10*{\parseInt}(w_1)+{\parseInt}({'i'})$.
    \end{itemize} 
%\end{definition}
Note that $\parseInt$ is undefined with $\varepsilon$ as the input.


In $\strint$, there are two types of variables, namely, the string variables $\svarx,\svary,\ldots \in \svars$ and the integer variables $\ivarx,\ivary,\ldots \in \ivars$.
%
A $\strint$ formula $\varphi$ is defined by the following rules, where $\len(\sterm)$ denotes the length of a string $\sterm$.
\[
\begin{array}{l c l}
\sterm & \Def & a \mid \svarx \mid \sterm \concat \sterm, \\
\iterm & \Def & n \mid \ivarx \mid \len(\sterm) \mid \parseInt(\sterm) \mid \iterm + \iterm \mid \iterm - \iterm,\\
\varphi & \Def & \sterm = \sterm \mid x \in \Aut \mid \iterm\ \op\ \iterm \mid \varphi \wedge \varphi \mid \varphi \vee \varphi \mid \neg \varphi,
\end{array}
\]
where $a \in (\Sigma_{\textit{num}})_\varepsilon$, $n \in \Nat$, $\Aut$ is an FA, and $\op \in \{=, < , >, \le, \ge\}$. Let us call $\sterm$ as string terms, $\iterm$ as integer terms, $\sterm = \sterm$ as string equality constraints, $x \in \Aut$ as regular constraints, $\iterm \op\ \iterm$ as arithmetic constraints. 
Let  $\svar(\varphi)$ and $\ivar(\varphi)$ denote the set of string variables and integer variables respectively occurring in $\varphi$.


%%%%%%%%%%%%%%%%%%%%%%%%%%%%%%%%
%%%%%%%%%%%%%%%%%%%%%%%%%%%%%%%%
\hide{
\paragraph{String Terms}
Given a finite alphabet $\Sigma$ and 
a finite set $X$ of string variables over $\Sigma^*$,
we define the set of terms $\textit{Terms}(\Sigma,X)$ 
to be the smallest set satisfying
\begin{itemize}
    \item[1] $\Sigma\cup \{\epsilon\} \cup X \subseteq \textit{Terms}(\Sigma,X)$;
    \item[2] if $t_1,t_2\in \textit{Terms}(\Sigma,X)$, then $t_1 \cdot t_2 \in \textit{Terms}(\Sigma,X)$.
\end{itemize} 

We extend $I_X$ to $\textit{Terms}(\Sigma,X)$ by letting $I_X(\epsilon)=\epsilon$, 
for $a\in \Sigma, I_X(a)=a$,
and $I_X(t_1\cdot t_2)= I_X(t_1)\cdot I_X(t_2)$.

\paragraph{String Constraints} \label{par: string constraints}
Given a constraint $\phi$ and an interpretation $I$,
$I\models \phi$ denotes that $I$ satisfies $\phi$,
and $I$ is called a \emph{model} of $\phi$.
We use $\lVert \phi\rVert$ to denote the set of all models of $\phi$.

We define the following three forms of atomic string constraints:
\begin{itemize}
\item An equality constraint $\phi_e$ is of the form 
$t_1 = t_2$, where $t_1,t_2\in \textit{Terms}(\Sigma,X)$.
We define $\lVert \phi_e \rVert = \{I\mid I(t_1)=I(t_2)\}$.
Inequality constraints can be defined analogously.

\item A regular constraint $\phi_r$ is of the form 
$x\in L(\mathcal{A})$,
where $x\in \mathbb{X}$ and $\mathcal{A}$ is a finite state automaton.
We define $\lVert \phi_r \rVert = \{I\mid I(x)\in L(\mathcal{A})\}$.

\item A length constraint $\phi_l$ is a linear constraint over 
$Z \cup \{|x| \mid x\in X\}$, %the values of $|x|$ for all $x\in X$,
where $|\cdot |$ is the length function.
We define $\lVert \phi_l \rVert = \{I \mid I\models \phi_l \}$.
\end{itemize}
}
%%%%%%%%%%%%%%%%%%%%%%%%%%%%%%%%
%%%%%%%%%%%%%%%%%%%%%%%%%%%%%%%%

The $\strint$ formulas are interpreted on $I=(I_s, I_i)$ where $I_s$ is a partial function from $\svars$ (the set of string variables) to $\Sigma^*$ and $I_i$ is a partial function from $\ivars$ (the set of integer variables) to $\Nat$. Moreover, it is required that the domains of $I_s, I_i$ are finite. Given $I = (I_s, I_i)$, the interpretations of string terms, integer terms, as well as the formulas $\varphi$ under $I$ are easy to comprehend, thus omitted to avoid tediousness. For $\varphi \in \strint$ and $I = (I_s, I_i)$, $I$ satisfies $\varphi$ if the interpretation of $\varphi$ under $I$ is $\ltrue$.
Let us use $\lVert \varphi \rVert$ to denote the set of $I = (I_s, I_i)$ satisfying $\varphi$.

\begin{example}
some example for $\strint$ here
\end{example}

%As usual, an interpretation $I$ is a mapping from the set of variables $X\cup Z$ to the respective domain, 
%essentially a pair of two mappings 
%$I_X$ and $I_Z$, i.e., $I= (I_X, I_Z)$,  
%where $I_X$ is a mapping in $X \mapsto \Sigma^*$ and $I_Z$ is a mapping in $Z \mapsto \mathbb{N}$.

The satisfiability problem of $\strint$ is to decide for a given constraint $\varphi \in \strint$,
whether $\lVert  \varphi \rVert \neq \emptyset$.
%if not, to compute an interpretation $I$ that satisfies $\Psi$.

\subsection{An Extension of Presburger Arithmetic with Exponential Functions ($\paexp$)}


%%%%%%%%%%%%%%%%%%%%%%%%%%%%%
%%%%%%%%%%%%%%%%%%%%%%%%%%%%%
\hide{
For a first order theory $T$,
we say theory $T$ admits quantifier elimination (QE) if for any formula in $T$, 
there is a quantifier-free formula equivalent to it.
It is well-known that if a theory admits QE, 
then it is a decidable theory.

The formal definition of PA is given in section \ref{PA}.
Here we add the ordering predicate $\le$ into the signature,
which can be defined by $x \le y\, \Def \, \exists z. x+z=y$.
However, the theory PA so far does not admit QE, 
for example, consider the formula $\exists x.x = y + y$. 
We augment the theory with countable unary divisible predicates
$n|x$, where $n\in \mathbb{N}$, 
$n|x$ is true if and only if $x\ \text{mod}\ n=0$ holds.
This structure of PA that admits QE is denoted by $(\mathbb{N},+)$.
}
%%%%%%%%%%%%%%%%%%%%%%%%%%%%%
%%%%%%%%%%%%%%%%%%%%%%%%%%%%%

{\paexp} extends Presburger arithmetic with the exponential function $10^x$ as well as the (partial) functions $\ell_{10}(\ivarx)$ and $\lambda_{10}(\ivarx)$ (cf. \cite{Point86}). The functions $\ell_{10}(\ivarx)$ and $\lambda_{10}(\ivarx)$ are inductively defined as follows.
\begin{itemize}
\item The (partial) function $\ell_{10}(\ivarx)$ is inductively defined as follows: For $n \ge 1$, $\ell_{10}(n) = m$ iff $10^m \le n < 10^{m+1}$. \wuhao{For $n \le 0$, we define $\ell_{10}(n)=0$}

%(Note that $\ell_{10}(0)$ is undefined.) 
%
\item The (partial) function $\lambda_{10}(\ivarx)$ can be defined by $\ell_{10}(\ivarx)$: For $n \ge 1$, $\lambda_{10}(n) = 10^{\ell_{10}(n)}$. (Intuitively, $\lambda_{10}(\ivarx)$ is the maximum power of 10 that is no greater than $\ivarx$.)
\end{itemize}

The syntax of {\paexp} is obtained from that of {\pa} by adding $10^\iterm$, $\ell_{10}(\iterm)$ and $\lambda_{10}(\iterm)$ to the definition of integer terms. Specifically, {\paexp} formulas are defined by the rules,
%
 $$
 \begin{array}{ l c l}
 \iterm &\Def& c \mid \ivarx \mid \iterm + \iterm \mid \iterm - \iterm \mid 10^\iterm \mid \ell_{10}(\iterm) \mid \lambda_{10}(\iterm), \\
 \phi &\Def & \iterm \ \op\ \iterm \mid c | \iterm \mid \phi \wedge \phi \mid \phi \vee \phi \mid \neg \phi \mid \exists \ivarx.\ \phi \mid \forall \ivarx.\ \phi.
 \end{array}
 $$
 The semantics of {\paexp} are defined similarly as {\pa}.

\begin{example}
Some example of  {\paexp} here.
\end{example} 
 
%Moreover, we introduce an abbreviation $\lambda_{10}(\ivarx) \equiv 2^{\ell_{10}(\ivarx)}$.
From the definition of $\lambda_{10}(\ivarx)$, we know that for all $n \ge 1$, $\lambda_{10}(n) \le n \le 10\lambda_{10}(n)-1$ holds.


%%%%%%%%%%%%%%%%%%%%%%%%%%%%%%%%%%%%
%%%%%%%%%%%%%%%%%%%%%%%%%%%%%%%%%%%%
\hide{
\begin{definition}
    Let $\mathcal{L}=\{0,1,+,\le,n \, |x(n\in \mathbb{N}),2^x,l_2(x)\}$, 
     $(\mathbb{N},+,2^x)$ be a $\mathcal{L}$-theory that has domain $\mathbb{N}$, where 
    \begin{itemize}
        \item  $2^x$ is interpreted to the exponential function of $2$ over $\mathbb{N}$; 
        \item interpretations of $0,1,+,\le,=$ are consistent with PA;
        \item for $n\ge 1$,$n|x$ holds iff $\exists y.x=ny$;
        \item $2^0=1$, for $n \ge 1, 2^n = 2^{n-1}+2^{n-1}$;
        We further assume that if $m,n\in \mathbb{N}, m\le n$, 
        then $2^{m-n} = 1$
        \item $l_2(0)=0$; for $n \ge 1,l_2(n) = y$ iff $2^y \le n < 2^{y+1}$;  $l_2(m-n) = 0$
        if $m,n\in \mathbb{N}$ and $m\le n$. 
    \end{itemize}
\end{definition}

$\lambda_2(x) = 2^{l_2(x)}$ can be defined by $l_2(x)$,
intuitively, $\lambda_2(x)$ means the maximal power of 2 that is not larger than $x$.
Then we have $\lambda_2(x) \le x \le 2\lambda_2(x)-1$,
which will be useful in our proof. 
}
%%%%%%%%%%%%%%%%%%%%%%%%%%%%%%%%%%%%
%%%%%%%%%%%%%%%%%%%%%%%%%%%%%%%%%%%%

\subsection{Flattening $\strint$ into $\paexp$}

We first recall the flattening approach for string constraints in \cite{Parosh:20:PLDI}, then show how to extend it to deal with the {\parseInt} function.

A \emph{flat domain restriction} for a string constraint $\varphi$ is a function $\flatfun_\varphi$ that maps each string variable $\svarx \in \svar(\varphi)$ to a flat language $(w_{\svarx,1})^* \cdots (w_{\svarx, k_\svarx})^*$, where $w_{\svarx, i} \in \Sigma_{\textit{num}}^+$ for every $i \in [k_\svarx]$. 
%
The flattened semantics of $\phi \in \strint$ is defined as $\llbracket \varphi \rrbracket_{\flatfun_\varphi} = \{(I_s, I_i) \in \llbracket \varphi  \rrbracket \mid \forall \svarx \in \svar(\varphi).\ I_s(x) \in  \flatfun_\varphi(\svarx)\}$.  

The flattening of $\varphi \in \strint$ under a flat domain restriction $\flatfun_\varphi$ is a {\paexp} formula, denoted by $\flatten_{\flatfun_\varphi}(\varphi)$, that encodes its flattened semantics.
%
More concretely, $\flatten_{\flatfun_\varphi}(\varphi)$ is a formula over the integer variables $\ivar(\varphi)$,  and Parikh variables $\pvar_{\flatfun_\varphi}(\varphi) = \bigcup_{\svarx \in \svar(\varphi)} \pvar_{\flatfun_\varphi}(\svarx)$, where $\pvar_{\flatfun_\varphi}(\svarx) = \{\#_{\svarx,i} \mid i \in [k_\svarx]\}$, called \emph{flattening variables}, plus some other auxiliary variables, such that 
%
$$\llbracket \phi \rrbracket_\kell =\decode_{\flatfun_\varphi}(\llbracket \flatten_{\flatfun_\varphi}(\varphi) \rrbracket |_{\ivar(\phi) \cup \pvar_{\flatfun_\varphi}(\varphi)})$$
%
The decoding function above decodes an interpretation of integer and flattening variables $I_e: \ivar(\varphi) \cup \pvar_{\flatfun_\varphi}(\varphi) \rightarrow \Nat$ as a set $\decode_{\flatfun_\varphi}(I_e)$ of interpretations of the $\varphi$'s integer and string variables $(I_s, I_i)$ with $I_s: \svar(\varphi) \rightarrow \Sigma^*$ and $I_i: \ivar(\phi) \rightarrow \Nat$ such that 
\begin{itemize}
\item  
for every $ \svarx \in \svar(\varphi)$, $I_s(\svarx) = w_{\svarx,1}^{I_e(\#_{\svarx,1})} \ldots  w_{\svarx,k_\svarx}^{I_e(\#_{\svarx,k_\svarx})}$, 
%
\item for every $ \ivarx \in \ivar(\varphi)$, $I_i(\ivarx) = I_e(\ivarx)$.
\end{itemize}

The formula $\flatten_{\flatfun_\varphi}(\varphi)$ is constructed inductively by following the structure of $\varphi$: $\flatten_{\flatfun_\varphi}(\varphi_1\ \mathfrak{o}\ \varphi_2) = \flatten_{\flatfun_\varphi}(\varphi_1) \ \mathfrak{o}\  \flatten_{\flatfun_\varphi}(\varphi_2)$, where $\mathfrak{o} \in \{\wedge, \vee\}$, and $\flatten_{\flatfun_\varphi}(\neg \varphi_1) = \neg \flatten_{\flatfun_\varphi}(\varphi_1)$. Therefore, it is sufficient to show how to construct $\flatten_{\flatfun_\varphi}(\varphi)$ for atomic constraints $\varphi$. 
In the sequel, we will show how to construct $\flatten_{\flatfun_\varphi}(\iterm_1 \op \iterm_2)$ where $\parseInt(\sterm)$ may occur in $\iterm_1$ or $\iterm_2$. The construction of $\flatten_{\flatfun_\varphi}(\varphi)$ for the other atomic constraints is essentially the same as that in \cite{Parosh:20:PLDI} and thus omitted. 


%%%%%%%%%%%%%%%%%%%%%%%%%%%%%%%%%%%%%%%%%%%%%%%%%
%%%%%%%%%%%%%%%%%%%%%%%%%%%%%%%%%%%%%%%%%%%%%%%%%
\hide{
The flattening technique was first introduced 
in \cite{Abdulla 2017}.
Fix an alphabet $\Sigma$ and an abstraction parameter $\alpha$, 
for a given atomic string constraint $\phi$, 
flattening $\phi$ with parameter $\alpha$ 
results in a new string constraint $\phi_\alpha$,
such that 
$R(\lVert \phi_\alpha \rVert) = \lVert \phi \rVert \cap \{I \mid \forall x\in X, I(x)\in \mathbb{G}'(\alpha)\}$,
where $R$ is the renaming function with its domain extended to interpretations in the normal manner.
Intuitively, it restricts $\phi$ to interpret 
string variables over $\mathbb{G}'(\alpha)$.

\cite{Abdulla 2017} discussed the flattening of basic 
string constraints including equality, integer, (regular) grammar and transducer constraints. 
For an atomic string constraint $\phi$,
the flattening $\phi_\alpha$ is still an atomic string constraint 
and is Parikh definable,
so its Parikh image can be expressed by a quantifier free PA formula.
Together with the purity condition,
we obtain an existential quantified PA formula $\rho$.
$\rho$ will be sent to a SMT solver,
if the solver returns an solution $\theta$,
then we can construct an interpretation for $\phi_\alpha$ from $\theta$,
otherwise it means $\phi$ is unsatistiable when 
string variables are interpreted  to $\alpha$-flat languages.

Take a regular constraint $\phi = x\in L(A)$ for example,
the flattening of $\phi$ resutls in a new finite state automaton $A'$ over $\Sigma(\alpha)$,
which encodes running $A$ ``in parallel" 
with the generic $\alpha$-flat automaton.
Let $\rho_1$ be the formula describing the Parikh image of $A'$,
which is a formula over variable sets $\Sigma(\alpha)^\#$.
Let $\rho_2$ be the purity condition \eqref{eq:purity}.
Then we obtain the PA formula $\exists (\Sigma(\alpha))^\#. \rho_1 \wedge \rho_2$.
In order to distinguish between different string variables,
we may replace $a(i)^\# \in \Sigma(\alpha)^\#$ by $(x,a(i))^\#$.

Since the structure of a flat automaton is decided 
by its abstraction parameter $\alpha$, 
a Counter-Example Guided Abstraction 
Refinement (CEGAR) framework is designed, which contains both an under- and an 
over-approximation module, to search the possible values of $\alpha$.
The termination for the overall algorithm is not guaranteed.


The string-number conversion functions are commonly used functions
in most of programming languages,
for example,
\verb+parseInt()+ in Java and \verb+Int()+ in Python.
The functions usually take two parameters, 
a string over the agreed alphabet $\Sigma$
and an optional parameter denotes the radix.
They parse the string according to the rules indicated by the radix,
and return an integer denoted by the string.

From the view of string constraints,
string-number conversion functions give rise to a new form of string constraints
and are more expressive than length constraints.
So we consider extending string constraints with 
{\parseInt} function. 
As the general problem of string constraints is undecidable,
we still adopt the idea of flattening, 
i.e., variables are restricted to (generic) flat languages.
This problem has been investigated in \cite{POPL20},
which defined a special form of flat restriction (straight-line PFA) and 
proposed a heuristic search method.

In this section, 
we describe the problem of interest first, and then 
present an reduction from the problem of solving flat string constraints with {\parseInt} function
to the decidability problem of Presburger Arithmetic with exponential functions.
Hence, we identify a decidable subset of string constraints, which is the largest one with decidability so far to the best of our knowledge.
}
%%%%%%%%%%%%%%%%%%%%%%%%%%%%%%%%%%%%%%%%%%%%%%%%%
%%%%%%%%%%%%%%%%%%%%%%%%%%%%%%%%%%%%%%%%%%%%%%%%%

Before presenting the construction of $\flatten_{\flatfun_\varphi}(\iterm_1 \op \iterm_2)$, for every $\svarx \in \svar(\varphi)$, we define $\flatten_{\flatfun_\varphi}(\parseInt(\svarx)) = \iterm_{\svarx,1}$  such that $(\iterm_{\svarx, i})_{i \in [k_\svarx]}$ and $(\ell_{\svarx, i})_{i \in [k_\svarx]}$ are inductively defined as follows: 
\begin{itemize}
\item for $i = k_\svarx$, 
$$\iterm_{\svarx, i} = \frac{\parseInt(w_{\svarx, k_\svarx}) (10^{|w_{\svarx,k_\svarx}| \#_{\svarx, k_\svarx} } -1 )}{(10^{|w_{\svarx, k_\svarx}|} -1 )}$$ 
and $\ell_{\svarx, i} = |w_{\svarx,k_\svarx}| \#_{\svarx, k_\svarx}$,
%
\item for $i \in [k_\svarx-1]$, 
%
$$\iterm_{\svarx, i} =  \frac{\parseInt(w_{\svarx, i}) (10^{|w_{\svarx, i} | \#_{\svarx, i} } -1 ) 10^{\ell_{\svarx, i+1}}} {(10^{|w_{\svarx, i}|} -1 )} + \iterm_{\svarx, i+1}$$
%
and $\ell_{\svarx, i} = |w_{\svarx, i} | \#_{\svarx, i} + \ell_{\svarx, i+1}$.
\end{itemize}

Then $\flatten_{\flatfun_\varphi}(\iterm_1 \op \iterm_2)$ is obtained from $\iterm_1 \op \iterm_2$ by the following procedure.
\begin{enumerate}
\item Replace each integer term $\len(t)$ such that $t = \alpha_1 \ldots \alpha_m$ with $\alpha_i \in \Sigma_{\textit{num}} \cup \svar(\varphi)$ for every $i \in [m]$, by 
 $\sum \limits_{i \in [m]} \theta_i$, where for every $i \in [m]$, $\theta_i = 1$ if $\alpha_i \in \Sigma_{\textit{num}}$ and $ \theta_i = \sum \limits_{j \in [k_{\alpha_i}]} |w_{\alpha_i, j} | \#_{\alpha_i, j}$ otherwise.
%
\item  Replace each integer term $\parseInt(\sterm)$ such that $t = \alpha_1 \ldots \alpha_m$ with $\alpha_i \in \Sigma_{\textit{num}} \cup \svar(\varphi)$ for every $i \in [m]$, by $\iterm_{\parseInt(\sterm)}$, where $\iterm_{\parseInt(\sterm)} = \iterm_{t,1}$ such that $(\iterm_{t,i})_{i \in [m]}$ and $(\ell_{t, i})_{i \in [m]}$ are inductively defined as follows: 
\begin{itemize}
\item if $\alpha_m \in \Sigma_{\textit{num}}$, then $\iterm_{t, m} = \alpha_m$ and $\ell_{t, m} = 1$, otherwise, $\iterm_{t, m} = \flatten_{\flatfun_\varphi}(\parseInt(\alpha_m))$ and $\ell_{t, m} = \sum \limits_{j \in [k_{\alpha_m}]} |w_{\alpha_m, j}| \#_{\alpha_m, j}$,
%
\item for $i \in [m-1]$, if $\alpha_i \in \Sigma_{\textit{num}}$, then $\iterm_{t, i} = \alpha_i 10^{\ell_{t, i+1}} + \iterm_{t, i+1}$ and $\ell_{t, i} = \ell_{t, i+1}+1$, otherwise, $\iterm_{t, i} = 
\flatten_{\flatfun_\varphi}(\parseInt(\alpha_i))10^{\ell_{t, i+1}}  + \iterm_{t, i+1}$ and $\ell_{t, i} = \ell_{t, i+1} + \sum \limits_{j \in [k_{\alpha_i}]} |w_{\alpha_i, j}| \#_{\alpha_i, j}$.
\end{itemize}
%
\item Let $\iterm'_1 \op \iterm'_2$ be the formula resulting from the aforementioned replacements. (Note that strictly speaking, $\iterm'_1 \op \iterm'_2$ is not a $\paexp$ formula since it contains divisions.) Let $N$ be the least common multiplier of the denominators of $\iterm'_1$ and $\iterm'_2$.  Then $\flatten_{\flatfun_\varphi}(\iterm_1 \op \iterm_2)$ is obtained by multiplying the both sides of $\iterm'_1 \op \iterm'_2$ with $N$,  so that the division operator disappears. 
\end{enumerate}

\begin{example}
Suppose $\parseInt(\svarx) = 2\ivarx$ is an atomic constraint and $\flatfun_\varphi(\svarx) = 1^*2^*$. Then 
\[
\small
\begin{array}{l l}
& \flatten_{\flatfun_\varphi}(\parseInt(\svarx)  =  2\ivarx)  \\
\Def & 1 \frac{10^{\#_{\svarx,1}}-1}{10-1}10^{\#_{\svarx,2}}  + 2 \frac{10^{\#_{\svarx,2}}-1}{10-1} = 2\ivarx   \\
\equiv & 10^{\#_{\svarx,1}+\#_{\svarx,2}} - 10^{\#_{\svarx,2}}  + 2 (10^{2\#_{\svarx,2}}-1) = 18\ivarx \\
\equiv & 10^{\#_{\svarx,1}+\#_{\svarx,2}} +  10^{\#_{\svarx,2}} = 18\ivarx + 2.
\end{array}
\]
%Take $n={\parseInt}((11)^a(10)^b)$ for example.
%\begin{align}
%    n=& {\parseInt}((11)^a)\cdot 2^{2b} + {\parseInt}((10)^b) \notag \\
%    =& {\parseInt}(11)\cdot \frac{2^{2a}-1}{2^2-1}\cdot 2^{2b} + 
%    {\parseInt}(10) \cdot \frac{2^{2b}-1}{2^2-1} \notag 
%\end{align}
\end{example}



%%%%%%%%%%%%%%%%%%%%%%%%%%%%%%%%%%%%%%%%%%%
%%%%%%%%%%%%%%%%%%%%%%%%%%%%%%%%%%%%%%%%%%%
\hide{
Now, we introduce a new form of atomic string constraints: 
a {\parseInt} constraint $\phi$ is of the form 
$n \sim {\parseInt}(t)$,
where $n$ is an integer term, $\sim \in \{\le,<,=,>,\ge\}$ and $t\in \textit{Terms}(\Sigma_{\textit{num}},X)$ is a string term.
$\lVert \phi\rVert \Def  \{I \mid I(n)\sim {\parseInt}(I(t))\}$.

In what follows, we only  consider the problem in the case when $t=x$ and $x$ is restricted to (generic) flat languages. 
%If $t$ is not a single variable, 
For the general case $t\in \textit{Terms}(\Sigma_{\textit{num}},X)$, 
it can be reduced to this special case by induction on the structure of $t$. 
%we can reduce the problem by separating $t$ (corresponding to the $|w|>2$ case in definition).

Given an $\alpha$-flat language $L$,
we assume $\alpha=\langle p,q \rangle$ and $L=(w_1)^*...(w_q)^*$,
where $p,q$ and $w_i(1\le i \le q) $ are known.
We further assume that $x = (w_1)^{\beta_1} ... (w_q)^{\beta_q}$,
then we have
\begin{align}
    &{\parseInt}(x) \notag \\
    =&{\parseInt}((w_1)^{\beta_1} ... (w_q)^{\beta_q}) \notag \\
    =&{\parseInt}((w_1)^{\beta_1} ... (w_{q-1})^{\beta_{q-1}})\cdot 2^{\beta_q |w_q|}
    + {\parseInt}((w_q)^{\beta_q}) \label{parse}
\end{align}
(\ref{parse}) is a recursive expression.
So we only need to deal with the basic case ${\parseInt}((w_q)^{\beta_q})$, 
where $w_q \neq \epsilon$
\begin{align}
    {\parseInt}((w_q)^{\beta_q}) &= \sum_{i=0}^{\beta_q-1} {\parseInt}(w_q)\cdot 2^{|w_q|\cdot i}  \notag \\
    &={\parseInt}(w_q)\frac{2^{|w_q|\cdot \beta_q}-1}{2^{|w_q|}-1}
    \label{parseInt-basic}
\end{align}
In (\ref{parseInt-basic}), since $w_q$ and $|w_q|$ are known, 
they can be regarded as constants.
The only unknown variable is $\beta_q$.

Combine (\ref{parse}) and (\ref{parseInt-basic}),
the constraint $n={\parseInt}(x)$ can be expressed by an arithmetic expression with 
$n$ and $(\beta_1,...,\beta_q)$,
inevitably with exponential components.

Observe the form of the above equation,
$a,b,n$ are integer variables and either occur in 
an exponential term or a linear term.
This is always the case,
so the problem can be reduced to 
the decidability of PA with exponential function.

When $x$ of {\parseInt} constraints is restricted to the
generic $\alpha$-flat language ($\alpha$ is fixed),
the (\ref{parse}) and (\ref{parseInt-basic}) still hold 
but $w_i(1\le i\le q)$ is known.
However,
by definition,
the generic $\alpha$-flat language is the finite union of all $\alpha$-flat languages,
so we can enumerate all possible values for $w_i$.
In this way,
the problem can still be reduced to the decidability of PA with exponential function ($\paexp$), i.e.,
\begin{theorem} \label{thm:string-parInt}
If {$\paexp$} is decidable, then the satisfiability (validity) of string constraints with {\parseInt} in which all string variables 
are ranged over flat strings is decidable. 
\end{theorem}
}
%%%%%%%%%%%%%%%%%%%%%%%%%%%%%%%%%%%%%%%%%%%
%%%%%%%%%%%%%%%%%%%%%%%%%%%%%%%%%%%%%%%%%%%





\section{Decision procedure for {\paexp}}\label{sec-dec}
%!TEX root = paper.tex

Semënov first proved that  {\paexp} admits quantifier elimination in \cite{Semenov84}, thus its satisfiability problem is decidable. However, Semënov did not give a concrete quantifier elimination procedure. Remedying this, Point proposed a quantifier elimination procedure for {\paexp} in \cite{Point86}. 

In this section, we describe how Point's quantifier elimination procedure works. 
%
\begin{theorem}[\cite{Point86}]
\label{thm-paexp}
{\paexp} admits quantifier elimination. 
\end{theorem}
%
Compared to \cite{Point86}, the presentation here is more accessible to computer science researchers. Moreover, the procedure presented here slightly improves Point's procedure, in the following two aspects: 1) DNF (disjunctive normal form) was required in Point's procedure, which is not required here, 2) in Point's procedure, the divisibility constraints produced by the elimination of linear occurrences of a variable should be converted to equality constraints (by introducing fresh variables) before the elimination of exponential occurrences of some other variable, which is unnecessary here, since the divisibility constraints are directly dealt with in the elimination of exponential occurrences of variables. 
%

However, in the next section we analyse the complexity of Point's procedure to be 3-EXPSPACE, which  is quite expensive and a faithful implementation would not scale\footnote{We did implement Point's algorithm and discovered that the implementation could only solve formulas of very small size.}.
So in Section~\ref{sec-opt}, we will propose various optimizations to Point's algorithm, aiming at an efficient implementation. 

\smallskip

As $\forall \ivarx. \ \varphi$ is equivalent to $\neg \exists \ivarx. \ \neg \varphi$,  thus in the sequel, we only need to  show that every $\paexp$ formula $\exists \ivarx.\ \varphi \in \paexp$, where $\varphi$ is quantifier-free, can be transformed into an equivalent quantifier-free formula $\varphi' \in \paexp$.

% when the quantifier elimination problem of $\paexp$  can be easily reduced to the above special case. 

% such that $\free(\varphi') = \free(\varphi) \setminus \{\ivarx\}$. 
%From this, we can easily derive that every {\paexp} formula $\varphi$ can be transformed effectively into an equivalent quantifier-free Presburger arithmetic formula $\varphi'$.

Before a formal description of the quantifier elimination procedure, let us use a simple example to illustrate the main idea and give an overview of the procedure.

\vspace{-4mm}

%\begin{example}
\subsection{An overview of the quantifier elimination procedure}
Consider $\varphi \Def \exists \ivarx_2.\ 10^{\ivarx_1 + \ivarx_2} - 10^{\ivarx_2} \le \ivary + 1001$. 

At first, we \emph{normalize} $\varphi$ by introducing a fresh variable $\ivarx_3$ for $\ivarx_1 + \ivarx_2$ and get the formula 
$$\varphi_1 \Def \exists \ivarx_3 \exists \ivarx_2.\ 10^{\ivarx_3} - 10^{\ivarx_2} \le \ivary + 1001 \wedge \ivarx_3 = \ivarx_1 + \ivarx_2.$$

Then, we enumerate different \emph{orders} of the quantified variables, i.e. $\ivarx_2$ and $\ivarx_3$. Since $\ivarx_3 = \ivarx_1 + \ivarx_2$, there is only one possible order, that is, $\ivarx_3 \ge \ivarx_2$.

Next, we illustrate how to eliminate the quantifier $\exists \ivarx_3$, assuming $\ivarx_3 \ge \ivarx_2$. The elimination of $\exists \ivarx_2$ is similar and simpler, thus omitted.

The elimination of $\exists \ivarx_3$ consists of two steps, namely, eliminating the exponential occurrences of $\ivarx_3$ first, and the linear occurrences next.

The main idea of the elimination of the exponential occurrences of $\ivarx_3$ is to observe that if $\ivarx_3 \ge \ell_{10}(\ivary+1001)+2$ and $\ivarx_3 \ge \ivarx_2 + 3$, then $10^{\ivarx_3} - 10^{\ivarx_2}$ is dominated by $10^{\ivarx_3}$, that is, $10^{\ivarx_3} - 10^{\ivarx_2} \ge 10^{\ivarx_3} - 10^{\ivarx_3 - 3} = (1-10^{-3}) 10^{\ivarx_3} \ge 10^{\ell_{10}(\ivary+1001)+1} = 10\lambda_{10}(\ivary+1001) > \ivary+1001$ (see Lemma~\ref{lem:exp-ineq} for the choice of the constants $2$ and $3$ in $\ivarx_3 \ge \ell_{10}(\ivary+1001)+2$ and $\ivarx_3 \ge \ivarx_2 + 3$.). 
Therefore, a necessary condition for $10^{\ivarx_3} - 10^{\ivarx_2} \le \ivary + 1001$ is that either $\ivarx_3 \le \ell_{10}(\ivary+1001)+1$ or $\ivarx_3 \le \ivarx_2 + 2$ holds.  

%\yfc{why $+2$ and $+3$ in $\ivarx_3 \ge \ell_{10}(\ivary+1001)+2$ and $\ivarx_3 \ge \ivarx_2 + 3$?}
\begin{itemize}
\item If $\ivarx_3 \le \ell_{10}(\ivary+1001)+1$, then we distinguish between whether $\ivarx_3 \le \ell_{10}(\ivary+1001)$ or  $\ivarx_3 = \ell_{10}(\ivary+1001)+1$. 
\begin{itemize}
\item If $\ivarx_3 \le \ell_{10}(\ivary+1001)$, then $10^{\ivarx_3} - 10^{\ivarx_2} \le 10^{\ell_{10}(\ivary+1001)} = \lambda_{10}(\ivary+1001) \le \ivary + 1001$. In this case, $10^{\ivarx_3} - 10^{\ivarx_2} \le \ivary + 1001 \wedge \ivarx_3 = \ivarx_1 + \ivarx_2$ can simplified into $\ltrue$.
%
\item If $\ivarx_3 = \ell_{10}(\ivary+1001)+1$, then $10^{\ivarx_3} - 10^{\ivarx_2} \le \ivary + 1001$ can be turned into $10^{\ell_{10}(\ivary+1001)+1} - 10^{\ivarx_2} \le \ivary + 1001 \equiv 10 \lambda_{10}(\ivary+1001) - 10^{\ivarx_2} \le \ivary + 1001 $.
\end{itemize} 
%
\item If $\ivarx_3 \le \ivarx_2 + 2$, then $\ivarx_3 = \ivarx_2 + j$ for $j \in \{0,1,2\}$. Thus $10^{\ivarx_3} - 10^{\ivarx_2} \le \ivary + 1001$ can be transformed to $\bigvee \limits_{j \in \{0,1,2\}} 10^{\ivarx_2 + j} - 10^{\ivarx_2} \le \ivary + 1001$.
\end{itemize}

To summarize, $\varphi_1$ is transformed into 
\[
\small
\begin{array}{l}
\varphi_2 \Def \exists \ivarx_3 \exists \ivarx_2. \\
\begin{array}{l}
\vspace{2mm}
\left(
\begin{array}{l}
\ivarx_3 \le \ell_{10}(\ivary+1001)\ \vee \\
\left(
\begin{array}{l}
\ivarx_3 = \ell_{10}(\ivary+1001)+1\ \wedge \\
10 \lambda_{10}(\ivary+1001) - 10^{\ivarx_2} \le \ivary + 1001 
\end{array}
\right) \vee \\
%
 \bigvee \limits_{j \in \{0,1,2\}}  \left(\ivarx_3 = \ivarx_2 +j \wedge 10^{\ivarx_2 + j} - 10^{\ivarx_2} \le \ivary + 1001\right)
\end{array}
\right) 
\wedge \\
 \ivarx_3 = \ivarx_1 + \ivarx_2.
 \end{array}
\end{array}
\]
Note that $\varphi_2$ contains \emph{only linear} occurrences of $\ivarx_3$.

Finally, we can eliminate the linear occurrences of $\ivarx_3$, thus the quantifier $\exists \ivarx_3$, by applying the quantifier elimination algorithm of {\pa}, e.g. Cooper's algorithm in \cite{Cooper72}. The elimination of $\ivarx_3$ in $\varphi_2$ here is simple, with $\ivarx_3$ replaced by $\ivarx_1 + \ivarx_2$. 

In the remainder of this section, we are going to describe the aforementioned steps of the decision procedures: Normalization, the enumeration of the variable orders, 
the elimination of the exponential occurrences of variables. The elimination of the  linear occurrences of variables is essentially the quantifier elimination of the {\pa} and omitted.

Let us assume that $\varphi \Def \exists \ivarx.\ \varphi'(\ivarx, \vec{\ivary})$, where $\varphi'$ is a quantifier-free formula. 

\vspace*{-3mm}
\subsection{Normalization}

The normalization step comprises the following sub-steps.
\begin{enumerate}
\item \textit{NNF transformation} At first, we transform $\varphi'(\ivarx,\vec{\ivary})$ into the NNF (negation normal form). Moreover, we remove the occurrences of $\neg$ by replacing 
\hide{(a) replacing $\neg c | \iterm$ with $\bigvee \limits_{j \in [c-1]} c | (\iterm+j)$, }
(a) $\neg c | \iterm$ with $ c \nmid \iterm$, 
(b) $\neg (\iterm_1 = \iterm_2)$ with $\iterm_1 < \iterm_2 \vee \iterm_2 < \iterm_1$, 
(c) $\neg (\iterm_1 < \iterm_2)$ with $\iterm_2 \le \iterm_1$, 
(d) $\neg (\iterm_1 \le \iterm_2)$ with $\iterm_2 < \iterm_1$, and so on.
%
\item \textit{replace $\ell_{10}(\iterm)$ terms} Repeat the following procedure, until there are no $\ell_{10}(\iterm)$ with $\ivarx$ occurs in $\iterm$: for each occurrence of $\ell_{10}(\iterm)$ such that $\ivarx$ occurs in $\iterm$, introduce a fresh variable, say $\ivarz$, and replace all occurrences of $\ell_{10}(\iterm)$ by $\ivarz$, moreover, add the constraint $10^\ivarz \le \iterm < 10^{\ivarz + 1}$ as a conjunct. Note that if $\iterm$ contains no variables, then $\ell_{10}(\iterm)$ is a constant. In this case, we can also assume that $\iterm$ contains $\ivarx$ and perform the same replacements, which helps in the analysis of complexity in \ref{app:cpx}. Let the resulting formula be $\varphi''$.
%

\item \textit{flatten $10^\iterm$ terms} Then repeat the following procedure to $\varphi''$, until for each occurrence of $10^{\iterm}$ with $\ivarx$ occurs in $\iterm$, we have $\iterm = \ivarx$: For each occurrence of the $10^{\iterm}$ in $\varphi''$, such that $\iterm$ contains $\ivarx$ but is not $\ivarx$, introduce a fresh variable, say $\ivarz$, and replace all occurrences of $10^{\iterm}$ by $10^\ivarz$, moreover, add the constraint $\ivarz = \iterm$ as a conjunct. Let $\varphi'''$ denote the resulting formula.  

\item \textit{$\le$ transformation} Do the following replacements to $\varphi'''$, so that all the atomic formulas in $\varphi^\dag$ are of the form $\iterm_1 \le \iterm_2$ ,$c | \iterm$ or $c \nmid \iterm$: Replace every occurrence of $\iterm_1 \ge \iterm_2$ with $\iterm_2 \le \iterm_1$. Replace every occurrence of $\iterm_1 < \iterm_2$ (resp. $\iterm_1 > \iterm_2$) with $\iterm_1 \le \iterm_2 - 1$ (resp. $\iterm_2 \le \iterm_1-1$). Replace ever occurrence of $\iterm_1 = \iterm_2$ wtih $\iterm_2 \le \iterm_1 \wedge \iterm_1 \le \iterm_2$. Let $\varphi^\dag$ the resulting formula. 

\item Let $\vec{\ivarz} = \ivarz_1,\ldots, \ivarz_n$ be an enumeration of the freshly introduced variables. Then the result of the normalization procedure is 
$\exists \vec{\ivarz}\exists \ivarx.\ \varphi^\dag$.
\end{enumerate}

Intuitively, the normalization step first absorbs all negation operators by transforming the formula into NNF. Then it removes the occurrences of $\ell_{10}(\iterm)$ where $\ivarx$ occurs in $\iterm$, by encoding them with the exponential function. Moreover, for each occurrence of $10^\iterm$ such that $\ivarx$ occurs in $\iterm$, it introduces a fresh variable $\ivarz$, replaces $10^\iterm$ with $10^\ivarz$, and adds the equality $\ivarz = \iterm$. All equalities and inequalities will be rewritten into the form $\iterm_1\le\iterm_2$. Finally, add quantifiers for the introduced fresh variables.

After the normalization, the resulting formula is of the following shape: 1) it is in NNF (negation normal form),  2) it contains no occurrences of $\ell_{10}(\iterm)$ such that $\ivarx$ occurs in $\iterm$, 3)  it contains no occurrences of $10^\iterm$ such that $\ivarx$ occurs in $\iterm$, but $\iterm \neq \ivarx$, 4) all the atomic formulas are of the form $\iterm_1 \le \iterm_2$ or $c | \iterm$. Denote the negation of $c | \iterm$ by $c\nmid t$, then the formula contains no negation symbol. 

%%%%%%%%%%%%%%%%%%%%%%%%%%%%%%%%%%%%%%%
\hide{
The normalization step first removes the occurrences of $\ell_{10}(\iterm)$ where $\ivarx$ occurs in $\iterm$, by encoding them with the exponential function. Moreover, for each occurrence of $10^\iterm$ such that $\ivarx$ occurs in $\iterm$, it introduces a fresh variable $\ivarz$, replaces $10^\iterm$ with $10^\ivarz$, and adds the equality $\ivarz = \iterm$.  It also applies some additional transformations. After the normalization, the resulting formula is of the following shape: 1) it is in NNF (negation normal form),  2) it contains no occurrences of $\ell_{10}(\iterm)$ such that $\ivarx$ occurs in $\iterm$, 3)  it contains no occurrences of $10^\iterm$ such that $\ivarx$ occurs in $\iterm$, but $\iterm \neq \ivarx$, 4) all the atomic formulas are of the form $\iterm_1 \le \iterm_2$ or $c | \iterm$. Denote the negation of $c | \iterm$ by $c\nmid t$, then the formula contains no negation symbol. A formula is \textit{normalized} if it satisfies the above properties. More details can be found in Appendix~\ref{app-norm}.
}
%%%%%%%%%%%%%%%%%%%%%%%%%%%%%%%%%%%%%%%

\vspace{-2mm}
\subsection{Enumeration of the variable orders} 

Suppose $n-1$ fresh variables are introduced in the normalization procedure, rename the original variable $\ivarx$ and the $n-1$ introduced variables $\ivarx_i,1\le i\le n$.
Let the output of the normalization procedure be $\exists \vec{\ivarx}.\ \varphi'$ with $\vec{\ivarx} = (\ivarx_1,\ldots, \ivarx_n)$. 
We then enumerate all the linear orders of $\{\ivarx_1,\ldots, \ivarx_n\}$. Each linear order can be represented by a permutation $\sigma \in \mathcal{S}_n$ (where $\mathcal{S}_n$ is the permutation group on $[n]$), with the intention that $\ivarx_{\sigma(n)} \ge \cdots \ge \ivarx_{\sigma(1)}$.

Assuming a linear order $\sigma \in \mathcal{S}_n$ of $\{\ivarx_1,\ldots, \ivarx_n\}$, we then consider $\varphi'_\sigma  = \exists \vec{\ivarx}.\ \varphi' \wedge \bigwedge \limits_{i \in [n-1]} \ivarx_{\sigma(i)} \le \ivarx_{\sigma(i+1)}$ and eliminate the quantifiers $\exists \ivarx_{\sigma(n)}$, $\ldots$, $\exists \ivarx_{\sigma(1)}$,  one by one and from $\ivarx_n$ to $\ivarx_1$. Let $\varphi''_\sigma$ denote the resulting formula.

Finally, $\exists \vec{\ivarx}.\ \varphi'$ is transformed into the quantifier-free formula $\bigvee \limits_{\sigma \in \mathcal{S}_n} \varphi''_{\sigma}$. 

In the sequel, assuming a linear order $\sigma \in \mathcal{S}_n$, for $i \in [n]$, let $\exists \ivarx_{\sigma(1)} \ldots \exists \ivarx_{\sigma(i)}.\ \varphi''_{\sigma,i}$ be the formula obtained from $\varphi'_\sigma$ by eliminating the quantifiers $\exists \ivarx_{\sigma(n)}$, $\ldots$, $\exists \ivarx_{\sigma(i+1)}$, we show how to eliminate the exponential occurrences of $\ivarx_{\sigma(i)}$ in $\exists \ivarx_{\sigma(1)} \ldots \exists \ivarx_{\sigma(i)}.\ \varphi''_{\sigma,i}$. We would like to remark that the linear occurrences of $\ivarx_{\sigma(i)}$ should be eliminated further so that the quantifier $\exists \ivarx_{\sigma(i)}$ can be eliminated. The elimination of linear occurrences of $\ivarx_{\sigma(i)}$ is essentially the quantifier elimination algorithm of {\pa}.
%$\varphi'_\sigma =  \exists \vec{\ivarx}.\ \varphi' \wedge \bigwedge \limits_{i \in [n-1]} \ivarx_{\sigma(i)} \le \ivarx_{\sigma(i+1)}$.

Note that the order $\ivarx_{\sigma(i)} \ge \ldots \ge \ivarx_{\sigma(1)}$ guarantees the maximality of $\ivarx_{\sigma(i)}$ among $\ivarx_{\sigma(i)}, \ldots, \ivarx_{\sigma(1)}$, which is essential for the elimination of $10^{\ivarx_{\sigma(i)}}$ from $\varphi''_{\sigma,i}$ (see Lemma~\ref{lem:exp-ineq}).

%\yfc{Explain why only the largest variable can  be eliminated. Which part of the procedure we need this condition? Or if we do not have the order, what can be wrong?}

\vspace*{-3mm}
\subsection{Elimination of  exponential occurrences of variables}\label{sec-elim-exp}

Let $i \in [n]$ and $\exists \ivarx_{\sigma(1)} \ldots \exists \ivarx_{\sigma(i)}.\ \varphi''_{\sigma,i}(\ivarx_{\sigma(i)}, \ldots, \ivarx_{\sigma(1)}, \vec{\ivary})$ be the formula obtained from $\varphi'_\sigma$ by eliminating the quantifiers $\exists \ivarx_{\sigma(n)}$, $\ldots$, $\exists \ivarx_{\sigma(i+1)}$. We show how to eliminate the exponential occurrences of $\ivarx_{\sigma(i)}$ in $\varphi''_{\sigma,i}$. The elimination is \emph{local} in the sense that it is applied to the atomic formulas independently. 

Recall that after normalization, the atomic formulas are of the form $\iterm_1 \le \iterm_2$, $c | \iterm$ or $c \nmid \iterm$. Therefore, we can assume that the atomic formulas in $\varphi''_{\sigma,i}$ are  of the form 
%
$a_i 10^{\ivarx_{\sigma(i)}}+\sum_{j=1}^{i-1} a_j 10^{\ivarx_{\sigma(j)}} + \sum_{k=1}^{i}b_k \ivarx_{\sigma(k)} \le \iterm(\vec{\ivary})$
or  
$c \ \big  | \ \big(a_i 10^{\ivarx_{\sigma(i)}}+\sum_{j=1}^{i-1} a_j 10^{\ivarx_{\sigma(j)}} + \sum_{k=1}^{i}b_k \ivarx_{\sigma(k)} + \iterm(\vec{\ivary})\big)$
(or $\nmid$).

%%%%%%%%%%%%%%%%%%%%%%%%%%%%%%%%%%%%%
\hide{
    For convenience, let us call these formulas as inequality respectively (in)divisibility atomic formulas. In the sequel, we illustrate how to eliminate the exponential occurrences of $\ivarx_{\sigma(i)}$ for these inequality atomic formulas. The elimination of the exponential occurrences of the (in)divisibility formulas are simpler and omitted, which can be found in Appendix~\ref{app-div}. 
}
%%%%%%%%%%%%%%%%%%%%%%%%%%%%%%%%%%%%%
\subsubsection{Inequality atoms}
In the sequel, we illustrate how to eliminate the exponential occurrences of $\ivarx_{\sigma(i)}$ for these inequality atomic formulas. Let us consider 
$$
\begin{array}{l}
\tau(\ivarx_{\sigma(i)}, \ldots, \ivarx_{\sigma(1)}, \vec{\ivary}) \Def  \\
\hspace{4mm} 
a_i 10^{\ivarx_{\sigma(i)}}+\sum_{j=1}^{i-1} a_j 10^{\ivarx_{\sigma(j)}} + \sum_{k=1}^{i}b_k \ivarx_{\sigma(k)} \le \iterm(\vec{\ivary}).
\end{array}
$$
%
The elimination of the exponential occurrences of $\ivarx_{\sigma(i)}$ in $\tau(\ivarx_{\sigma(i)}, \ldots, \ivarx_{\sigma(1)}, \vec{\ivary})$ relies on the following lemma. Intuitively, the lemma states the fact that if $\ivarx_{\sigma(i)}$ is sufficiently greater than $\ivarx_{\sigma(i-1)}$, then the left-hand-side of $\tau(\ivarx_{\sigma(i)}, \ldots, \ivarx_{\sigma(1)}, \vec{\ivary})$ is \emph{dominated} by $a_i10^{\ivarx_{\sigma(i)}}$, moreover, if $a_i > 0$ and the value of $\ivarx_{\sigma(i)}$ is sufficiently small (resp. big), then $\tau(\ivarx_{\sigma(i)}, \ldots, \ivarx_{\sigma(1)}, \vec{\ivary})$ holds (resp. does not hold), similarly for $a_i < 0$.

\vspace{-1mm}
\begin{lemma} \label{lem:exp-ineq}
Let  
%
$$
\small
\begin{array}{l}
\tau(\ivarx_{\sigma(i)}, \ldots, \ivarx_{\sigma(1)}, \vec{\ivary}) \Def  \\
\hspace{4mm} 
\vspace{-1mm}
a_i 10^{\ivarx_{\sigma(i)}}+\sum_{j=1}^{i-1} a_j 10^{\ivarx_{\sigma(j)}} + \sum_{k=1}^{i}b_k \ivarx_{\sigma(k)} \le \iterm(\vec{\ivary}).
\end{array}
$$
%
with $a_i \neq 0$, $A\Def \sum_{j=1}^{i-1}|a_j|$, 
%$B\Def  \sum_{j=1}^{i}|b_j|$, 
$B \Def 2(\ell_{10}(\sum_{j=1}^{i}|b_j|)+3)$,
and $\delta\Def  \ell_{10}(A)+3$. 
\begin{itemize}
    \item If $a_i > 0$, let $\alpha(\vec{\ivary}) \Def \ell_{10}(\iterm(\ivary))- \ell_{10}(a_i)$, then 
    \begin{itemize}
        \item if $\ivarx_{\sigma(i)} \le \alpha(\vec{\ivary})  -1$, $\ivarx_{\sigma(i)} \ge B$ and $\ivarx_{\sigma(i)} \ge \ivarx_{\sigma(i-1)} +\delta $, then $\tau(\ivarx_{\sigma(i)}, \ldots, \ivarx_{\sigma(1)}, \vec{\ivary})$ holds,
        \item if $\ivarx_{\sigma(i)} \ge \alpha(\vec{\ivary})  +2$, $\ivarx_{\sigma(i)} \ge B$ and $\ivarx_{\sigma(i)}  \ge \ivarx_{\sigma(i-1)} +\delta$, then $\tau(\ivarx_{\sigma(i)}, \ldots, \ivarx_{\sigma(1)}, \vec{\ivary})$ \textbf{does not} hold.
    \end{itemize}
    \item If $a_i < 0$, let $\alpha(\vec{\ivary})  \Def \ell_{10}(-\iterm(\ivary))- \ell_{10}(-a_i)$, then 
    \begin{itemize}
        \item if $\ivarx_{\sigma(i)} \le \alpha(\vec{\ivary})  -1$, $\ivarx_{\sigma(i)} \ge B$ and $\ivarx_{\sigma(i)} \ge \ivarx_{\sigma(i-1)} +\delta $, then $\tau(\ivarx_{\sigma(i)}, \ldots, \ivarx_{\sigma(1)}, \vec{\ivary})$ \textbf{does not} hold,
        \item if $\ivarx_{\sigma(i)} \ge \alpha(\vec{\ivary})  +2$, $\ivarx_{\sigma(i)} \ge B$ and $\ivarx_{\sigma(i)} \ge \ivarx_{\sigma(i-1)} +\delta $, then $\tau(\ivarx_{\sigma(i)}, \ldots, \ivarx_{\sigma(1)}, \vec{\ivary})$ holds.
    \end{itemize}
\end{itemize}
\end{lemma}

If $a_i > 0$, then the exponential occurrences of $\ivarx_{\sigma(i)}$ in $\tau(\ivarx_{\sigma(i)}, \ldots, \ivarx_{\sigma(1)}, \vec{\ivary})$ can be eliminated by utilizing  Lemma~\ref{lem:exp-ineq} and enumerating the constraints on $\ivarx_{\sigma(i)}$ and $\ivarx_{\sigma(i-1)}$. Specifically, 
$\tau(\ivarx_{\sigma(i)}, \ldots, \ivarx_{\sigma(1)}, \vec{\ivary})$ is equivalent to 
\vspace{-2mm}
\[
\small
\begin{array}{l}
\bigvee \limits_{p=0}^{B-1} a_i 10^{p}+\sum_{j=1}^{i-1} a_j 10^{\ivarx_{\sigma(j)}} + b_i p + \sum_{k=1}^{i-1}b_k \ivarx_{\sigma(k)} \le \iterm(\vec{\ivary}) \\
\bigvee \big(\ivarx_{\sigma(i)} \ge B \wedge \ivarx_{\sigma(i)} \le \alpha(\vec{\ivary})  -1  \wedge \ivarx_{\sigma(i)} \ge \ivarx_{\sigma(i-1)} +\delta \big)\\
%
\bigvee \bigvee \limits_{p=0}^{\delta-1} 
\left(
\begin{array}{l}
\ivarx_{\sigma(i)} \ge B \wedge \ivarx_{\sigma(i)} \le \alpha(\vec{\ivary})  -1 \ \wedge \\
 \ivarx_{\sigma(i)} = \ivarx_{\sigma(i-1)} +p\ \wedge\\
 \tau(\ivarx_{\sigma(i)}, \ldots, \ivarx_{\sigma(1)}, \vec{\ivary})[\ivarx_{\sigma(i-1)} + p /\ivarx_{\sigma(i)}] 
\end{array}
\right)\\
\bigvee 
\left(
\begin{array}{l}
\ivarx_{\sigma(i)} \ge B \wedge \ivarx_{\sigma(i)} = \alpha(\vec{\ivary})\ \wedge \\
\tau(\ivarx_{\sigma(i)}, \ldots, \ivarx_{\sigma(1)}, \vec{\ivary})[\alpha(\vec{\ivary})/\ivarx_{\sigma(i)}]
\end{array}
\right)  \\
\bigvee 
\left(
\begin{array}{l}
\ivarx_{\sigma(i)} \ge B \wedge \ivarx_{\sigma(i)} = \alpha(\vec{\ivary})+1\ \wedge \\
\tau(\ivarx_{\sigma(i)}, \ldots, \ivarx_{\sigma(1)}, \vec{\ivary})[\alpha(\vec{\ivary})+1/\ivarx_{\sigma(i)}]
\end{array}
\right)  \\
\bigvee \bigvee \limits_{p=0}^{\delta-1} 
\left(
\begin{array}{l}
\ivarx_{\sigma(i)} \ge B \wedge \ivarx_{\sigma(i)} \ge \alpha(\vec{\ivary})+2\ \wedge\\
 \ivarx_{\sigma(i)} = \ivarx_{\sigma(i-1)} +p\ \wedge \\
 \vspace{-1mm}
 \tau(\ivarx_{\sigma(i)}, \ldots, \ivarx_{\sigma(1)}, \vec{\ivary})[\ivarx_{\sigma(i-1)}+ p /\ivarx_{\sigma(i)}]
\end{array}
\right),
\end{array}
\]
where the exponential occurrences of $\ivarx_{\sigma(i)}$ disappear.  
%
The elimination of the exponential occurrences of $\ivarx_{\sigma(i)}$ for the case $a_i < 0$ is similar. 

\subsubsection{Divisibility atoms}
Consider a divisiblilty atomic formula 
%
$$
\begin{array}{l}
\tau(\ivarx_{\sigma(i)}, \ldots, \ivarx_{\sigma(1)}, \vec{\ivary}) \Def  \\
d\ \big |\ \big(a_i 10^{\ivarx_{\sigma(i)}} + \sum_{j=1}^{i-1} a_j 10^{\ivarx_{\sigma(j)}} + \sum_{k=1}^{i} b_k \ivarx_{\sigma(k)} 
+ \iterm(\vec{\ivary}) \big)
\end{array}
$$
with $a_i \neq 0$. The indivisibility case can be treated analogously.
%

Let $d = 2^{r_1} 5^{r_2}  d_0$ such that $d_0$ is divisible by neither $2$ nor $5$. Moreover, let $r = \max(r_1, r_2)$. Then $d | (10^rd_0)$. 

If $d_0 = 1$, then $10^r$ is divisible by $d = 2^{r_1}5^{r_2}$. Thus for every $n \ge r$, $d \ |\ 10^n$.  Therefore, in this case, $\tau(\ivarx_{\sigma(i)}, \ldots, \ivarx_{\sigma(1)}, \vec{\ivary})$ is equivalent to 
\[
\small
\begin{array}{l}
\bigvee \limits_{p = 0}^{r - 1} d\ \big | \big(a_i 10^{p} + \sum_{j=1}^{i-1} a_j 10^{\ivarx_{\sigma(j)}} + b_kp + \sum_{k=1}^{i-1} b_k \ivarx_{\sigma(k)} 
+ \iterm(\vec{\ivary}) \big)\\
%
\vee \big(\ivarx_{\sigma(i)} \ge r \wedge d\ \big | \big(\sum_{j=1}^{i-1} a_j 10^{\ivarx_{\sigma(j)}} + \sum_{k=1}^{i} b_k \ivarx_{\sigma(k)} 
+ \iterm(\vec{\ivary}) \big)\big),
\end{array}
\]
where the exponential occurrences of $\ivarx_{\sigma(i)}$ disappear.

Next, let us assume $d_0 > 1$. Since $10$ and $d_0$ are relatively prime, according to Euler's theorem (cf. \cite{HW80}), $10^{\phi(d_0)} \equiv 1 \bmod d_0$, where $\phi$ is the Euler function. Suppose $10^{\phi(d_0)} = kd_0 + 1$ for some $k \in \Nat$. 
Then for every $n \in \Nat$ with $n \ge r$, 
$$
\begin{array}{l}
10^{n + \phi(d_0)} \bmod d =10^{n-r} 10^r (kd_0 + 1) \bmod d = \\
10^{n-r} (k 10^rd_0 + 10^r) \bmod d = \\
10^{n-r} (0+10^r) \bmod d = 10^n \bmod d.
\end{array}
$$

Then $\tau(\ivarx_{\sigma(i)}, \ldots, \ivarx_{\sigma(1)}, \vec{\ivary})$ is equivalent to 
\[
\begin{array}{l}
\bigvee \limits_{p=0}^{r-1} \tau(\ivarx_{\sigma(i)}, \ldots, \ivarx_{\sigma(1)}, \vec{\ivary})[p/\ivarx_{\sigma(i)}]\ \vee \\
\left(
\begin{array}{l}
\ivarx_{\sigma(i)} \ge r\ \wedge \\
\bigvee \limits_{q = 0}^{\phi(d_0)-1} 
\left(
\begin{array}{l}
\phi(d_0)\ \big |\ (\ivarx_{\sigma(i)} - r - q)\ \wedge \\
d\ \big | 
\left(
\begin{array}{l}
a_i 10^{r+q} + \sum_{j=1}^{i-1} a_j 10^{\ivarx_{\sigma(j)}} + \\
\sum_{k=1}^{i} b_k \ivarx_{\sigma(k)} + \iterm(\vec{\ivary})
\end{array}
\right) 
\end{array}
\right)
\end{array}
\right),
\end{array}
\]
where the exponential occurrences of $\ivarx_{\sigma(i)}$ disappear.

\wuhao{describe the whole procedure using pseudo code}

\section{Complexity Analysis}\label{sec-cpx}
%!TEX root = paper.tex

In this section, we analyse the complexity of Point's procedure applied on an existential quantified {\paexp} formula. 

Consider a formula of the form $$\exists \ivarx^1\cdots\exists \ivarx^m.\varphi(\ivarx^1,\cdots,\ivarx^m,\vec{\ivary})$$ with free variables $\vec{\ivary}$. Let $N$ denote the length of the formula. We will prove that the space complexity to eliminate all (existential) quantifiers has a 3-EXPSPACE upper bound. Since Cooper's quantifier elimination algorithm for $\pa$ works as a subprocedure and also has a 3-EXPSPACE upper bound, this bound may not be easily improved.

We first give a brief analysis of the normalization step, in which fresh variables are introduced and the length of the formula grows linearly at most. 
For the rest of the algorithm, we adopt the strategy of Oppen's analysis of Cooper's algorithm (cf. \cite{Oppen73}). The idea is that the upper bound of the formula length can be expressed using the product of the number of atoms, the number of coefficients and the length of the maximum constant. 
The critical point of our analysis is that only coefficients of linear occurrences of quantified variables need to be considered.

\paragraph{Normalization} In the normalization step, we normalize the formula w.r.t quantified variables from $\ivarx^m$ to $\ivarx^1$. Suppose that fresh variables $\ivarx^i_1,\cdots,\ivarx^i_{n_i}$ are introduced for each $\ivarx^i(1\le i\le m)$, the formula then becomes $$\exists \ivarx^1\exists \ivarx^1_1\cdots\exists\ivarx^1_{n_1}\cdots\exists \ivarx^m\exists \ivarx^m_1\cdots\exists\ivarx^m_{n_m}.\varphi'(\ivarx^1,\cdots,\ivarx^m_{n_m},\vec{\ivary}).$$ 

Since the number of newly introduced variables is less than the number of occurrences of exponential and logarithmic functions in the original formula, we have at most $m+\sum_{i=1}^m n_i\le N$ quantified variables after normalization.  

The increase in the length of the formula during normalization comes from two sources, the conjuncts for each introduced variables and the additional formulas for translating $\neq,=$ relations to $\le$. It is not hard to see that both these operations will at most increase the length of the formula by a constant factor. 

In the sequel, to avoid redundant symbols, we still use $m$ to denote the number of quantified variables and $N$ to denote the length of the formula after normalization. Just keep in mind that $m\le N$. 

\hide{
    Given a formula $\exists \ivarx. \varphi(\ivarx, \vec{\ivary})$ with length $n$. After the normalization step, suppose we obtain a formula with $m$ quantified variables $\exists \ivarx_1 \dots\exists \ivarx_m. \varphi'(\ivarx_1,\dots, \ivarx_m, \vec{\ivary})$. We show that the length of the new formula is at most $10n$ and the number of quantified variables $m\le n$. In each sub-step of normalization, we analyse the worst situation: 
    (1) \textit{NNF transformation} suppose all atoms are of the form $\iterm_1=\iterm_2$, then taking negation will double the number of atoms, the length increases to $2n$ at most. 
    (2) \textit{replace $\ell_{10}(\iterm)$ terms} the original formula has at most $n$ terms of the form $\ell_{10}(\iterm)$, for each of them, we introduce a fresh variable and two conjuncts. The formula length increases to $4n$.
    (3) \textit{flatten $10^\iterm$ terms} similar to (2), the formula has at most $n$ such terms and for each term, a fresh variable and a conjunct are added. The formula length increases to $5n$. 
    (4) \textit{$\le$ transformation} similar to (1), here we assume all atoms are of the form $\iterm_1  = \iterm_2$, so the length of the formula with increase to $10n$ at most. 
    
    The analysis for the normalization step is coarse, for example, the worst case in (1)\&(4) or (2)\&(3) can not happen at the same time. However, what we need is that the increased length of the formula is bounded by a constant factor,i.e. 10. In addition, we can also conclude that the number of fresh variables, $m$, is less than $n$ since there are at most $n$ different forms of terms.     
}


\paragraph{Enumeration of linear orders (the outer for-loop)}

Denote the normalized formula by $\exists \vec{\ivarx}. \varphi(\vec{\ivarx},\vec{\ivary})$ with $\vec{\ivarx} = (\ivarx_1,\dots,\ivarx_m)$. According to Point's procedure, we then eliminate the quantified variables one by one: each time select the largest variable among $\ivarx_i,1\le i\le m$, first eliminate the exponential occurrences, then linear occurrences. Each linear order of quantified variables is given by a permutation. For $m$ quantified variables, there are $m!$ possible linear orders, which means the inner for-loop repeats $m!$ times.

\paragraph{Elimination of quantifiers (the inner for-loop)}
Suppose the specified linear order is $\ivarx_m\ge \dots \ge \ivarx_1$. Let $\exists \ivarx_1\dots \exists\ivarx_{m-k}.\varphi_k$ denote the formula obtained by eliminating quantifiers $\exists \ivarx_m,\dots,\exists\ivarx_{m-k+1}$ from $\exists \vec{\ivarx}. \varphi(\vec{\ivarx},\vec{\ivary})$. Let $\varphi'_k$ denote the formula obtained by further eliminating exponential occurrences of $\ivarx_{m-k+1}$ (from $\varphi_k$). Let $\varphi_0$ denote $\varphi$.

Let $c_k$ be the number of distinct divisor $d$ in atoms of the form $d \mid \iterm$ and $d \nmid \iterm$ plus the number of distinct coefficients of \emph{linear occurrences} of quantified variables in $\varphi_k$. Let $s_k$ be the largest constant (including coefficients) and $a_k$ be the number of atomic formulas in $\varphi_k$. Similarly we define $c'_k$, $s'_k$ and $a'_k$ for $\varphi'_k$.

First we analyse the sub-procedure to eliminate exponential occurrences of the inner most quantified variable $\ivarx_m$. We prove the following lemma.

\begin{lemma}\label{lem:cpx exp}
$$c'_0\le {c_0}^2 \qquad s'_0\le m{s_0}^2 \qquad a'_0\le s_0a_0$$ 
\end{lemma}

\hide{\begin{align}
    c'_0&\le c_0^2 \notag \\
    s'_0&\le ms_0^2 \notag\\
    a'_0&\le s_0a_0\notag 
\end{align}}
\begin{proof}
The analysis is divided into two cases by assuming all atomic formulas are of the same form (inequalities atoms or divisibility atoms). 

If all atoms are inequalities of the form in Lemma~\ref{lem:exp-ineq}. We know that each atomic formula $\tau$ with exponential occurrence of $\ivarx_m$ is replaced by a new formula. Note that coefficients of linear occurrences of $\ivarx_i$, for $i\le m-2$, remain unchanged throughout the substitutions, only the coefficients of linear occurrences of $\ivarx_m$ and $\ivarx_{m-1}$ will be changed: constant 1 is introduced as a coefficient for $\ivarx_m$, and if we substitute $\ivarx_m$ by $\ivarx_{m-1}+p$ for some constant $p$, coefficient of linear occurrence of $\ivarx_{m-1}$ will become $b_m+b_{m-1}$ ($b_m,b_{m-1}$ are coefficients for $\ivarx_{m}$ and $\ivarx_{m-1}$ in $\tau$, see Lemma~\ref{lem:exp-ineq} ). Since the new coefficient is obtained by adding two linear coefficient together, we have $c'_1\le {c_0}^2$. Note that $\delta$ and $B$ in Lemma~\ref{lem:exp-ineq} are at most $\ell_{10}(ms_0)$. When we substitute $\ivarx_m$ by $\ivarx_{m-1}+\delta-1$ or by $B-1$, the largest constant in the formula becomes at most $s_0\cdot  10^{\ell_{10}(ms_0)}\le m{s_0}^2$. And an inequality is replaced by at most $4\ell_{10}(ms_0)$ atomic formulas, so $a'_1\le 4\ell_{10}(ms_0)a_0$.

If all atoms are divisibility atomic formulas of the form $d \ \big \vert  \ \iterm$ or $d \ \nmid \ \iterm$. We have $c'_1<2c_0$ because a divisibility atomic formula  will produce at most two forms of atomic formulas $d \ \big \vert  \ \iterm$ and $\phi(d) \ \big \vert  \ \iterm$. In a divisibility atom, any constant in the divident $\iterm$ , say $l$, can be replaced by ($l \bmod d)$. So we have $s'_1\le s_0$. When $d$ is a large prime number, $\phi(d)=d-1$, a divisibility atomic formula is replaced by roughly $d$ atomic formulas, so $a'_1\le s_0a_0$. 

Choose larger upper bounds for $c'_1$, $s'_1$ and $a'_1$ respectively, then the lemma is proved.\end{proof}

When $\varphi_0'$ is transformed into $\varphi_{1}$ by eliminating linear occurrences of $\ivarx_m$, Oppen's analysis gives the following lemma.

\begin{lemma}\cite{Oppen73}\label{lem:cpx oppen}
$$c_1\le {c_0'}^4\qquad
s_1\le {s_0'}^{4c_0'}\qquad
a_1\le {a_0'}^4 {s_0'}^{2c_0'}
$$
\end{lemma}

\hide{    \begin{align}
    c_1&\le {c_0'}^4 \notag \\
    s_1&\le {s_0'}^{4c_0'} \notag\\
    a_1&\le {a_0'}^4 {s_0'}^{2c_0'}\notag 
\end{align} }

Combining Lemma~\ref{lem:cpx exp} and Lemma~\ref{lem:cpx oppen}, we have
$$c_1\le {c_0}^8\qquad
s_1\le (m{s_0}^2)^{4{c_0}^2} \qquad
a_1\le (s_0a_0)^4(m_0{s_0}^2)^{2{c_0}^2} 
$$

\hide{\begin{align}
    c_1&\le {c_0}^8 \notag \\
    s_1&\le (m{s_0}^2)^{4{c_0}^2} \notag\\
    a_1&\le (s_0a_0)^4(m_0{s_0}^2)^{2{c_0}^2}
\end{align} }

By induction on $k$ and assuming $m\le N$, we get

\begin{lemma}
$$c_k\le {c_0}^{8^k}\qquad
s_k\le N^{(4c_0)^{ 8^k}} {s_0}^{(8c_0)^{ 8^k}} \qquad
a_k\le {a_0}^{4^k}n^{(4c_0)^{ 8^k}} s_0^{(8c_0)^{ 8^k}} $$
\end{lemma}

\hide{    \begin{align}
    c_k&\le {c_0}^{8^k} \notag \\
    s_k&\le N^{(4c_0)^{ 8^k}} {s_0}^{(8c_0)^{ 8^k}} \notag\\
    a_k&\le {a_0}^{4^k}n^{(4c_0)^{ 8^k}} s_0^{(8c_0)^{ 8^k}} \notag
\end{align} 
}
If we adopt the assumptions in \cite{Oppen73}, by assuming $c_0\le N$, $a_0\le N$ and $s_0\le N$, the space required to store the quantifier free formula $\varphi_k$, is bounded by the product of the number of linear orders $m!$, the number of atoms $a_k$, the maximum number of constants $2m+2$ per atom, the maximum amount of space $s_k$ to store each constant and some constant $q$. So the space complexity is $O(m! \cdot a_k \cdot (2m+2) \cdot s_k)=O(2^{2^{2^{pn \log n}}})$, which belongs to 3-EXPSPACE.
 
\hide{$${\sf space}\le q \cdot m! \cdot a_k \cdot (2m+2) \cdot s_k \le 2^{2^{2^{pn \log n}}}$$
for some large constant $p$. }

The above analysis is focused on the existential {\paexp} formulas and is sufficient for our string constraints solving context. 
Unfortunately, directly extending our analysis on general {\paexp} formulas with alternating quantifiers by transforming $\forall \ivarx.\varphi$ into $\neg \exists \ivarx.\neg \varphi$ will not give an elementary upper bound. 


\hide{We will explain the main idea without going into details on this issue. To eliminate quantifiers in a formula, say $\forall \ivarx_2\exists \ivarx_1.\varphi(\ivarx_1,\ivarx_2,\vec \ivary)$, we first eliminate $\exists \ivarx_1$, treating $\ivarx_2$ and $\vec \ivary$ both as free variables. The subprocedure to eliminate exponential occurrences of $\ivarx_1$ is at the cost of introducing logarithmic expressions of $\ivarx_2$ and $\vec \ivary$, which in the worst case increases \emph{exponentially} the number of fresh variables in the normalization step w.r.t $\ivarx_2$. We conjecture that the space complexity of quantifier elimination procedure over {\paexp} formulas still has an elementary upper bound, which may require a more detailed analysis or improvements of the algorithm.}


\section{Optimizations}\label{sec-opt}
%!TEX root = paper.tex

In the last section, we illustrated the main idea of the decision procedure for  {\paexp}. The decision procedure has a high complexity in the sense that the elimination of the exponential occurrences   of each variable incurs an exponential blow-up, similarly for linear occurrences. Note that this high complexity holds even for quantifier-free formulas containing exponential terms: For a quantifier-free formula $\varphi$ containing exponential terms, we solve its satisfiability problem by adding the existential quantifiers for all the variables occurring in $\varphi$, then eliminate the quantifiers one by one, resulting into $\ltrue$ or $\lfalse$ in the end. The original formula $\varphi$ is satisfiable if $\ltrue$ is obtained in the end.

In this section, we focus on quantifier-free {\paexp} formulas (or existential {\paexp} formulas since they are satisfiability-equivalent), and present various optimizations of the quantifier elimination procedure for {\paexp}, aiming at an efficient implementation. The focus on quantifier-free {\paexp} formulas is motivated by the following two facts: 1) the flattening of {$\strint$} constraints results into such formulas, 2) these formulas are already challenging for state-of-the-art SMT solvers (with exponential functions defined as recursive functions). 

Let $\varphi$ be a quantifier-free {\paexp} formula in the remainder of this section. Moreover, we assume that $\varphi$ is normalized since the optimizations presented in the sequel are for normalized formulas. Furthermore, for technical convenience, we assume that all the inequality atomic formulas are of the form 
%
$\sum_{j=1}^{n} a_j 10^{\ivarx_{j}} + \sum_{k=1}^{n}b_k \ivarx_{k} \le c$, 
%
where $c$ is an integer constant. (Implicitly, we assume that there are no free variables and all the variables are existentially quantified.)

%In the sequel, we present two major optimizations and omit some additional ones, which can be found in Appendix~\ref{app-opt}. 
In the sequel, we will explain two major optimizations in \ref{para: opt reduce} and \ref{para: opt under appro}. Additional optimizations are listed in \ref{para: opts}.

\subsection{Reduce the number of enumerated variable orders by over approximation}\label{para: opt reduce}

Recall that in the decision procedure for {\paexp}, after the normalization, the variable orders are enumerated and for each order, the exponential and linear occurrences of variables are eliminated. Since the quantifier elimination is expensive and applied to each possible order of variables, if we could reduce the candidate variable orders in the very beginning, it would facilitate considerable speed-up for the decision procedure. 

Our main idea is to consider an over approximation of $\varphi$, which is a {\pa} formula $\varphi'$, and use $\varphi'$ to remove the infeasible candidate variable orders.

Note that all the exponential terms in $\varphi$ is of the form $10^{\ivarz}$ for some integer variable $\ivarz$. 
%
The over approximation is based on the observation that $10^n \ge 9 n + 1$ for every $n \in \Nat$. Then we obtain the over approximation $\varphi'$ from $\varphi$ by replacing each exponential term $10^{\ivarz}$ with a fresh variable $\ivarz'$ and add $\ivarz' \ge 9 \ivarz +1$ as a conjunct.

Then during the enumeration of the linear orders for the variables $\ivarx_1, \ldots, \ivarx_n$, we can quickly remove those infeasible candidates $\sigma$ such that $\varphi' \wedge \bigwedge \limits_{i \in [n-1]}\ivarx_{\sigma(i)} \le \ivarx_{\sigma(i+1)}$ is unsatisfiable. A special case is that if $\varphi'$ is unsatisfiable, then we can directly conclude that original formula $\varphi$ is unsatisfiable.

%Given any formula in $\paexp$, we first invoke \textbf{Normalization} to translate it into a simpler form. Then we  try to detect if there are obvious conflicts, for example, if the linear constraints combined are unsatisfiable, the whole formula is of course unsatisfiable.

%Suppose the normalized formula $\theta$ contains variables $\{x_i:i\in[n]\}$, so the exponential terms could be $e^{x_i}$. For each $e^{x_i}$, we subtitute $e^{x_i}$ by a fresh variable $y_i$ in $\theta$ and add the constraints $(e-1)x_i+1\le y_i$ to get $\theta'$. Since $(e-1)x_i+1\le e^{x_i}$ holds for all $e,x\in \Nat$, $\theta'$ is an over-approximation of $\theta$ and if $\theta'$ is unsatisfiable over positive reals, then $\theta$ is unsatisfiable over $\Nat$.

%\vspace*{-3mm}
\subsection{Avoid the elimination of linear occurrences of variables by under approximation}\label{para: opt under appro}

The decision procedure of {\paexp} in Section~\ref{sec-dec} requires the elimination of both exponential and linear occurrences of variables. Considering the fact that {\pa} formulas can be solved efficiently by the state-of-the-art solvers, e.g. CVC4 and Z3, one natural idea is to try to only eliminate the exponential occurrences, but not the linear occurrences, of variables, and obtain the {\pa} formulas in the end, which can then be solved by the state-of-the-art solvers.

Recall that Lemma~\ref{lem:exp-ineq} enables us to eliminate the exponential occurrences of $\ivarx_{\sigma(i)}$ from 
%\vspace{-1mm}
$$
\begin{array}{l}
\tau(\ivarx_{\sigma(i)}, \ldots, \ivarx_{\sigma(1)}) \Def  \\
\hspace{4mm} 
%\vspace{-1mm}
a_i 10^{\ivarx_{\sigma(i)}}+\sum_{j=1}^{i-1} a_j 10^{\ivarx_{\sigma(j)}} + \sum_{k=1}^{i}b_k \ivarx_{\sigma(k)} \le c.
\end{array}
$$
Actually, Lemma~\ref{lem:exp-ineq} does more in the sense that all occurrences of $\ivarx_{\sigma(i)}$, including the linear ones, are eliminated from the atomic formulas resulted from $\tau(\ivarx_{\sigma(i)}, \ldots, \ivarx_{\sigma(1)})$, e.g. $\tau(\ivarx_{\sigma(i)}, \ldots, \ivarx_{\sigma(1)})[\ivarx_{\sigma(i-1)} + p /\ivarx_{\sigma(i)}]$. Then we can continue eliminating the exponential occurrences of $\ivarx_{\sigma(i-1)}$ from $\tau(\ivarx_{\sigma(i)}, \ldots, \ivarx_{\sigma(1)})[\ivarx_{\sigma(i-1)} + p /\ivarx_{\sigma(i)}]$, provided that the coefficient of $\ivarx_{\sigma(i-1)}$ therein is nonzero. Iterating this process would produce a {\pa} formula eventually.

Nevertheless, the side condition of Lemma~\ref{lem:exp-ineq}, namely $a_i \neq 0$, undermines the aforementioned natural idea. If $a_i = 0$, but $b_i \neq 0$, 
%namely, 
%$$
%\begin{array}{l}
%\tau(\ivarx_{\sigma(i)}, \ldots, \ivarx_{\sigma(1)}) \Def  \\
%\hspace{4mm} 
%\sum_{j=1}^{i-1} a_j 10^{\ivarx_{\sigma(j)}} + \sum_{k=1}^{i}b_k \ivarx_{\sigma(k)} \le c,
%\end{array}
%$$
then we are unable to utilize Lemma~\ref{lem:exp-ineq} to eliminate the linear occurrences of $\ivarx_{\sigma(i)}$ from $\tau(\ivarx_{\sigma(i)}, \ldots, \ivarx_{\sigma(1)})$. In this case, the quantifier elimination algorithm of {\pa} has to be applied to eliminate $\ivarx_{\sigma(i)}$ from $\tau(\ivarx_{\sigma(i)}, \ldots, \ivarx_{\sigma(1)})$, so that later on, we can eliminate the exponential occurrences of $\ivarx_{\sigma(i-1)}$, which requires that $\ivarx_{\sigma(i-1)}$ is the maximum variable in the left-hand side of the inequality.

To avoid applying the quantifier elimination algorithm of {\pa}, we consider the following under approximation of $\tau(\ivarx_{\sigma(i)}, \ldots, \ivarx_{\sigma(1)})$, namely, we additionally assume that $\ivarx_{\sigma(i)} \le 10^{u}$ for some constant bound $u \in \Nat$ with $u \ge 1$. Then $\tau(\ivarx_{\sigma(i)}, \ldots, \ivarx_{\sigma(1)})$ can be rewritten as 
%\begin{equation}\label{eqn-org}
%\vspace{-2mm}
\[\tau'(\ivarx_{\sigma(i)}, \ldots, \ivarx_{\sigma(1)}) \Def \sum_{j=1}^{i-1} a_j 10^{\ivarx_{\sigma(j)}} + \sum_{k=1}^{i-1}b_k \ivarx_{\sigma(k)} \le c - b_i  \ivarx_{\sigma(i)}.\]
%\vspace{-1mm}
%\end{equation}
Let us assume $a_{i-1} > 0$.
Define $c_1, c_2$ as follows: If $b_i > 0$, then $c_1 = c- b_i 10^u$ and $c_2 = c$, otherwise, $c_1 = c$ and $c_2 = c - b_i 10^u$.
It is easy to observe that $c_1 \le c - b_i \ivarx_{\sigma(i)} \le c_2$.
Then we can apply Lemma~\ref{lem:exp-ineq} to the following two inequalities to eliminate $10^{\ivarx_{\sigma(i-1)}}$,
%\vspace{-2mm}
\begin{equation}\label{eqn-lb}
\sum_{j=1}^{i-1} a_j 10^{\ivarx_{\sigma(j)}} + \sum_{k=1}^{i-1}b_k \ivarx_{\sigma(k)} \le c_1
\end{equation}
and
%\vspace{-2mm}
\begin{equation}\label{eqn-ub}
\sum_{j=1}^{i-1} a_j 10^{\ivarx_{\sigma(j)}} + \sum_{k=1}^{i-1}b_k \ivarx_{\sigma(k)} \le c_2.
\end{equation}
Let $\alpha_1 = \ell_{10}(c_1) -  \ell_{10}(a_{i-1})$ and $\alpha_2 = \ell_{10}(c_2) -  \ell_{10}(a_{i-1})$. Then from Lemma~\ref{lem:exp-ineq}, 
\begin{itemize}
\item if $\ivarx_{\sigma(i-1)} \ge B$, $\ivarx_{\sigma(i-1)} \le \alpha_1 - 1$, and $\ivarx_{\sigma(i-1)} \ge \ivarx_{\sigma(i-2)} + \delta$, then inequality (\ref{eqn-lb}), thus also $\tau'(\ivarx_{\sigma(i)}, \ldots, \ivarx_{\sigma(1)})$, is evaluated to $\ltrue$,
\item if $\ivarx_{\sigma(i-1)} \ge B$, $\ivarx_{\sigma(i-1)} \ge \alpha_2 + 2$, and $\ivarx_{\sigma(i-1)} \ge \ivarx_{\sigma(i-2)} + \delta$, then inequality (\ref{eqn-ub}), thus also $\tau'(\ivarx_{\sigma(i)}, \ldots, \ivarx_{\sigma(1)})$, is evaluated to $\lfalse$.
\end{itemize}
Therefore, $\tau'(\ivarx_{\sigma(i)}, \ldots, \ivarx_{\sigma(1)})$, thus also $\tau(\ivarx_{\sigma(i)}, \ldots, \ivarx_{\sigma(1)})$, is equivalent to 
\[
\small
\begin{array}{l}
\bigvee \limits_{p=0}^{B-1} ( \ivarx_{\sigma(i-1)} = p \wedge \tau'(\ivarx_{\sigma(i)}, \ldots, \ivarx_{\sigma(1)}) [p/\ivarx_{\sigma(i-1)}])   \\
\bigvee \big(\ivarx_{\sigma(i-1)} \ge B \wedge \ivarx_{\sigma(i-1)} \le \alpha_1  - 1  \wedge \ivarx_{\sigma(i-1)} \ge \ivarx_{\sigma(i-2)} +\delta \big)\\
%
\bigvee \bigvee \limits_{p=0}^{\delta-1} 
\left(
\begin{array}{l}
\ivarx_{\sigma(i-1)} \ge B \wedge \ivarx_{\sigma(i-1)} \le \alpha_1  -1 \ \wedge \\
 \ivarx_{\sigma(i-1)} = \ivarx_{\sigma(i-2)} +p\ \wedge\\
 \tau'(\ivarx_{\sigma(i)}, \ldots, \ivarx_{\sigma(1)})[\ivarx_{\sigma(i-2)} + p /\ivarx_{\sigma(i-1)}] 
\end{array}
\right)\\
\bigvee \bigvee \limits_{p=0}^{\delta-1} 
\left(
\begin{array}{l}
\ivarx_{\sigma(i-1)} \ge B \wedge \ivarx_{\sigma(i-1)} \ge \alpha_2 + 2 \ \wedge \\
 \ivarx_{\sigma(i-1)} = \ivarx_{\sigma(i-2)} +p\ \wedge\\
 \tau'(\ivarx_{\sigma(i)}, \ldots, \ivarx_{\sigma(1)})[\ivarx_{\sigma(i-2)} + p /\ivarx_{\sigma(i-1)}] 
\end{array}
\right)\\
\bigvee \bigvee \limits_{p = \alpha_1}^{\alpha_2 + 1}
\left(
\begin{array}{l}
\ivarx_{\sigma(i-1)} \ge B \wedge \ivarx_{\sigma(i-1)} = p\ \wedge \\
 \tau'(\ivarx_{\sigma(i)}, \ldots, \ivarx_{\sigma(1)})[p /\ivarx_{\sigma(i-1)}] 
\end{array}
\right),
\end{array}
\]
where all exponential occurrences of $\ivarx_{\sigma(i-1)}$ are eliminated.

Similarly, we can eliminate the exponential occurrences of $\ivarx_{\sigma(i-2)}$ from $\tau'(\ivarx_{\sigma(i)}, \ldots, \ivarx_{\sigma(1)})[\ivarx_{\sigma(i-2)} + p /\ivarx_{\sigma(i-1)}]$ as well as  $\tau'(\ivarx_{\sigma(i)}, \ldots, \ivarx_{\sigma(1)})[p /\ivarx_{\sigma(i-1)}]$, and so on. Eventually, we obtain a {\pa} formula.

%We also propose some additional optimizations, which are omitted due to the page limit and can be found in Appendix~\ref{app-opt}.



\hide{
\paragraph{Synchronize the elimination of exponential occurrences of the same variable in different atomic formulas}

Although Lemma~\ref{lem:exp-ineq} is stated for a single atomic formula, the elimination of the exponential occurrences of the same variable in different atomic formulas can actually be synchronized. That is,  let $\alpha^\tau_{1}, \alpha^\tau_{2}, B^\tau, \delta^\tau$ be the constants as stated in the aforementioned under-approximation of an inequality $\tau$, define $\alpha^{\min}_1, \alpha^{\max}_2, B^{\max}, \delta^{\max}$ as the minimum of $\alpha^\tau_1$, the maximum of $\alpha^\tau_2$, the maximum of $B^\tau$, and the maximum of $\delta^\tau$ respectively with $\tau$ ranging over the inequalities of $\varphi$. Then we can use the same constants $\alpha^{\min}_1, \alpha^{\max}_2, B^{\max}, \delta^{\max}$ for different inequalities when eliminating the exponential occurrences of the same variable. 
}

%%%%%%%%%%%%%%%%%%%%%%%%%%%%%%%%%%%%%%%%%%%%%%
%%%%%%%%%%%%%%%%%%%%%%%%%%%%%%%%%%%%%%%%%%%%%%
\hide{
Note that when Lemma~\ref{lem:exp-ineq} is applied to an atomic formula $\tau$ to eliminate exponential occurrences of $\ivarx$, it computes 4 parameters, namely $(\alpha^\tau_1,\alpha^\tau_2,B^\tau,\delta^\tau)$, and obtains an equivalent {\pa} formula according to these parameters. Since the parameters are used as boundary conditions in enumeration, we can choose the strictest boundary conditions and apply to all atomic formulas.  

More specifically, we first compute $(\alpha^\tau_1,\alpha^\tau_2,B^\tau,\delta^\tau)$ 
for all atomic formulas that contain $10^\ivarx$. Let $\alpha^{\min}_1$ denote the minimal element in $\{\alpha^\tau_1\}_\tau$, let $\alpha^{\max}_2,B^{\max},\delta^{\max}$ denote the maximal element in $\{\alpha^\tau_2\}_\tau$, $\{B^\tau\}_\tau$ and $\{\delta^\tau\}_\tau$ respectively. Then, we apply $(\alpha^{\min}_1,\alpha^{\max}_2,B^{\max},\delta^{\max})$ to all atomic formulas $\tau$ in parallel and obtain $\varphi'$ equivalent to $\varphi$. 

For instance, assume coefficients of $10^\ivarx$ in atomic formulas of $\varphi$ are all non-negative. Let $\varphi^+$ denote the formula obtained by replace atomic formulas where coefficients of $10^\ivarx$ are positive by $\ltrue$. The resulted formula $\varphi'$ is
\[
\small
\begin{array}{l}
\bigvee \limits_{p=0}^{B^{\max}-1} ( \ivarx_{\sigma(i-1)} = p \wedge \varphi(\ivarx_{\sigma(i)}, \ldots, \ivarx_{\sigma(1)}) [p/\ivarx_{\sigma(i-1)}])   \\
\bigvee \left(
    \begin{array}{l}
        \ivarx_{\sigma(i-1)} \ge B^{\max} \wedge \ivarx_{\sigma(i-1)} \le \alpha^{\min}_1  - 1  \wedge \\
        \ivarx_{\sigma(i-1)} \ge \ivarx_{\sigma(i-2)} +\delta^{\max} \wedge \varphi^+  
    \end{array}
    \right)\\
%
\bigvee \bigvee \limits_{p=0}^{\delta^{\max}-1} 
\left(
\begin{array}{l}
\ivarx_{\sigma(i-1)} \ge B^{\max} \wedge \ivarx_{\sigma(i-1)} \le \alpha^{\min}_1  -1 \ \wedge \\
 \ivarx_{\sigma(i-1)} = \ivarx_{\sigma(i-2)} +p\ \wedge\\
 \varphi(\ivarx_{\sigma(i)}, \ldots, \ivarx_{\sigma(1)})[\ivarx_{\sigma(i-2)} + p /\ivarx_{\sigma(i-1)}] 
\end{array}
\right)\\
\bigvee \bigvee \limits_{p=0}^{\delta^{\max}-1} 
\left(
\begin{array}{l}
\ivarx_{\sigma(i-1)} \ge B^{\max} \wedge \ivarx_{\sigma(i-1)} \ge \alpha^{\max}_2 + 2 \ \wedge \\
 \ivarx_{\sigma(i-1)} = \ivarx_{\sigma(i-2)} +p\ \wedge\\
 \varphi(\ivarx_{\sigma(i)}, \ldots, \ivarx_{\sigma(1)})[\ivarx_{\sigma(i-2)} + p /\ivarx_{\sigma(i-1)}] 
\end{array}
\right)\\
\bigvee \bigvee \limits_{p = \alpha^{\min}_1}^{\alpha^{\max}_2 + 1}
\left(
\begin{array}{l}
\ivarx_{\sigma(i-1)} \ge B^{\max} \wedge \ivarx_{\sigma(i-1)} = p\ \wedge \\
 \varphi(\ivarx_{\sigma(i)}, \ldots, \ivarx_{\sigma(1)})[p /\ivarx_{\sigma(i-1)}] 
\end{array}
\right),
\end{array}
\]
}
%%%%%%%%%%%%%%%%%%%%%%%%%%%%%%%%%%%%%%%%%%%%%%
%%%%%%%%%%%%%%%%%%%%%%%%%%%%%%%%%%%%%%%%%%%%%%


\hide{
\paragraph{Avoid the formula-size blow-up by depth-first search}

The {\pa} formula resulting from the elimination of exponential occurrences is essentially a big disjunction of the formulas of small size. If we store this big disjunction naively, then the formula size quickly blows up and exhausts the memory. Instead, we choose to do a depth-first search (DFS) and consider the disjuncts, which are of small sizes, one by one, and solve the satisfiability problem for these disjuncts. If during the search, a satisfiable disjunct is found, then the search terminates and ``SAT'' is reported.

\paragraph{Preprocess with small upper bound}

We believe that if a quantifier-free {\paexp} formula is satisfiable, then most probably it is satisfiable with small values assigned to variables. Consequently, as a preprocessing step, we put a small upper bound, e.g. the biggest constant occurring in the formula, on the values of variables, and perform a depth-first search, so that a model can be quickly found, if there is any. If this preprocessing is unsuccessful, then we continue the search with the greater upper bound $10^u$ for some proper $u \ge 1$.
}

%\zhilin{stopped here}



%%%%%%%%%%%%%%%%%%%%%%%%%%%%%%%%%%%%%%%%%%%%%%%%%%%%%%%
%%%%%%%%%%%%%%%%%%%%%%%%%%%%%%%%%%%%%%%%%%%%%%%%%%%%%%%
\hide{
In the last section, 
we introduced quantifier elimination as a decision procedure for {$\paexp$}. According to theorem \ref{thm:string-parInt}, the satisfiability of {$\strint$} fragment is decidable.
However, it is not practical to directly apply quantifier elimination when solving constraints over {$\paexp$} due to two main reasons.

First, quantifier elimination does not return a model that satisfies the constraints. In constrast, it only leaves a formula with constants and free variables. To obtain satisfiability, all free variables need to be treated as existential quantified variables and be eliminated one by one until the formula is equal to \textit{true} or \textit{false}.
Second, the quantifier elimination procedure has a high complexity, both time and space. Note that we alternatively invoke QE-exp and Cooper's algorithm to eliminate exponential terms and linear terms of a quantified variable, the length of the formula grows super-exponentially in the procedure. 

In this section, we introduced some optimizations on the QE procedure restricted to quantifier-free fragment in {$\paexp$}. Given a quantifier-free formula, we wish to decide its satisfiability and find an assignment for variables in the formula if it exists. Since the Cooper's algorithm is the main contribution to the complexity, we try to only eliminate exponential terms using an variation of \textbf{Elim-exp} and send the obtained linear formula to SMT-solver. The cost is that the algorithm is no longer complete in some situations, in which case some practical restrictions are needed.

A quantifier-free formula is a boolean combination of constraints. Here we allow only equalities and inequalities, divisible predicates are replaced by equalities, for example, $2|x$ is replace to $x=2y$, where $y$ is a fresh variable. Since we only eliminate exponential terms, the newly introduced $y$ will not bother. For convenience, we call variables that occurs exponentially in the formula \emph{exponential variables}, variables that have only linear occurrences \emph{linear variables}. Similarly,
constraints that have exponential terms are called \textit{exponential constraints}, others are called \textit{linear constraints.}

\paragraph{over-approximation}

Given any formula in $\paexp$, we first invoke \textbf{Normalization} to translate it into a simpler form. Then we  try to detect if there are obvious conflicts, for example, if the linear constraints combined are unsatisfiable, the whole formula is of course unsatisfiable.

Suppose the normalized formula $\theta$ contains variables $\{x_i:i\in[n]\}$, so the exponential terms could be $e^{x_i}$. For each $e^{x_i}$, we subtitute $e^{x_i}$ by a fresh variable $y_i$ in $\theta$ and add the constraints $(e-1)x_i+1\le y_i$ to get $\theta'$. Since $(e-1)x_i+1\le e^{x_i}$ holds for all $e,x\in \Nat$, $\theta'$ is an over-approximation of $\theta$ and if $\theta'$ is unsatisfiable over positive reals, then $\theta$ is unsatisfiable over $\Nat$.

This technique also helps us to rule out impossible orders of exponential variables. In this case, constraints $\bigwedge_{1\le i\le n-1} x_{\sigma(i)}<x_{\sigma(i+1)}$ is added in $\theta'$.

\paragraph{modified \textbf{Elim-exp}}\label{para: modified Elim-exp}

\textbf{Elim-exp} is performed recursively on atoms according to orders of all variables. The idea is that if the upper bound of variables is given, it is enough to consider orders of exponential variables and this step can be performed on all atoms in the formula simultaneously.

Assume we have $m$ exponential variables with an order $x_1\le ...\le x_m$ and $n-m$ linear variables $x_{m+1},...,x_n$. We further assume that there is an upper bound for all variables, say $x_{m+1},...,x_n< e^\gamma$, the bound may be inferred from constraints or specified manually. When we try to eliminate exponential terms of $x_i(1\le i\le m)$, an atom with $e^{x_i}$ is of the form 

\begin{equation}\label{eq:modify-Elim-exp}
    \sum_{j=1}^i a_j e^{x_j}+\sum_{k=1}^n b_j x_j\le t 
\end{equation}
where $a_m\neq 0$ and $t$ is a constant. Note that the form of (\ref{eq:modify-Elim-exp}) is different from formula in theorem \ref{thm:exp-ineq}, exponential variables $x_{i+1},...,x_{m}$ may appear linearly in formula (\ref{eq:modify-Elim-exp}) because we do not invoke \textbf{Elim-linear} to eliminate linear terms. Following the method of theorem \ref{thm:exp-ineq}, we seperate terms in the left hand side into the following form.
\begin{equation}
    \sum_{j=1}^i (a_j e^{x_j} + b_j x_j) \le t -\sum_{k=i+1}^n b_k x_k \notag
\end{equation}

The left hand side $\sum_{j=1}^i (a_j e^{x_j} + b_j x_j)$ can be estimated using the same technique in theorem \ref{thm:exp-ineq}. For the right hand side, define  $ t_1 \Def t-\gamma (\sum_{k=i+1}^n b_k)$, $t_2\Def t+\gamma (\sum_{k=i+1}^n b_k)$, we have
\begin{equation}
    t_1  \le t - \sum_{k=i+1}^n b_k x_k \le  t_2 \notag
\end{equation} 

Similar to \textbf{Elim-exp}, we will discuss 3 subcases,
assuming w.l.o.g $a_i>0$, define $\alpha_1 \Def l(t_1)-l(a_i)$, $\alpha_2 \Def l(t_2)-l(a_i)$. 

\begin{itemize}
    \item $x_i\le \alpha_1 -1$, corresponds to $\rho_1$ (in \textbf{Elim-exp}), the atom is evaluated \textit{true}
    \item $\alpha_1 -1\le x_i \le \alpha_2 +2$, corresponds to $\rho_2$ and $\rho_3$
    \item $\alpha_2 + 2\le x_i$, corresponds to $\rho_4$, the atom is evaluated \textit{false}
\end{itemize}

For each atom with $e^{x_i}$, we compute $\alpha_1$ and $\alpha_2$ according to coefficients in the atom. Select the minimal $\alpha_1$ and the maximal $\alpha_2$, in this way we can perform \textbf{Elim-exp} simultaneously on all atoms in the formula.

Note that the manually specified bound is for variables that only appear linearly in exponential constraints, if there are no variables of this form, the modified algorithm is still complete.

\paragraph{DFS pruning}

Given an order for exponential variables $x_1\le x_2 \le ... \le x_m$, the process of \textbf{Elim-exp} from $x_m$ to $x_1$ can be seen as a search tree with depth $m+1$. A depth $i(1\le i\le m)$ non-leaf node is a disjunction, each of its child corresponds to a subcase in \textbf{Elim-exp} of $x_i$, a leaf is labeled by a linear formula in which all exponential terms have been eliminated.  

In our implement, we adopt a DFS strategy to search the tree to avoid the explosion of length of the formula. When eliminating $x_{i-1}$, the upper bound for $x_i$ in some subcases are passed down for pruning. 

\paragraph{pre-search}
To answer \textit{unsat}, the algorithm must search for all possible orders of exponential variables. Intuitively, in most satisfiable cases exponential variables are small, so we first give a small bound (roughly the biggest constants in the formula) for exponential variables to perform a pre-search.
}
%%%%%%%%%%%%%%%%%%%%%%%%%%%%%%%%%%%%%%%%%%%%%%%%%%%%%%%
%%%%%%%%%%%%%%%%%%%%%%%%%%%%%%%%%%%%%%%%%%%%%%%%%%%%%%%


\subsection{Additional optimization techniques}\label{para: opts}
\paragraph{Synchronize the elimination of exponential occurrences of the same variable in different atomic formulas}

Although Lemma~\ref{lem:exp-ineq} is stated for a single atomic formula, the elimination of the exponential occurrences of the same variable in different atomic formulas can actually be synchronized. That is,  let $\alpha^\tau_{1}, \alpha^\tau_{2}, B^\tau, \delta^\tau$ be the constants as stated in the aforementioned under-approximation of an inequality $\tau$, define $\alpha^{\min}_1, \alpha^{\max}_2, B^{\max}, \delta^{\max}$ as the minimum of $\alpha^\tau_1$, the maximum of $\alpha^\tau_2$, the maximum of $B^\tau$, and the maximum of $\delta^\tau$ respectively with $\tau$ ranging over the inequalities of $\varphi$. Then we can use the same constants $\alpha^{\min}_1, \alpha^{\max}_2, B^{\max}, \delta^{\max}$ for different inequalities when eliminating the exponential occurrences of the same variable. 

\paragraph{Avoid the formula-size blow-up by depth-first search}

The {\pa} formula resulting from the elimination of exponential occurrences is essentially a big disjunction of the formulas of small size. If we store this big disjunction naively, then the formula size quickly blows up and exhausts the memory. Instead, we choose to do a depth-first search (DFS) and consider the disjuncts, which are of small sizes, one by one, and solve the satisfiability problem for these disjuncts. If during the search, a satisfiable disjunct is found, then the search terminates and ``SAT'' is reported.

\paragraph{Preprocess with small upper bound}

We believe that if a quantifier-free {\paexp} formula is satisfiable, then most probably it is satisfiable with small values assigned to variables. Consequently, as a preprocessing step, we put a small upper bound, e.g. the biggest constant occurring in the formula, on the values of variables, and perform a depth-first search, so that a model can be quickly found, if there is any. If this preprocessing is unsuccessful, then we continue the search with the greater upper bound $10^u$ for some proper $u \ge 1$.


\section{Implementation and Experiments}\label{sec-exp}
\input{7_impl_exp.tex}

\section{Conclusion}
\input{8_conclusion.tex}

\hide{
\begin{appendices}
\section{Appendix}\label{secA1}
%!TEX root = paper.tex
%%%%%%%%%%%%%%%%%%%%%%%%%%%%%%%%%%%
\hide{
    \subsection{Details of the normalization step}\label{app-norm}
    %
    The normalization step comprises the following sub-steps.
    \begin{enumerate}
    \item \textit{NNF transformation} At first, we transform $\varphi'(\ivarx,\vec{\ivary})$ into the NNF (negation normal form). Moreover, we remove the occurrences of $\neg$ by replacing 
    \hide{(a) replacing $\neg c | \iterm$ with $\bigvee \limits_{j \in [c-1]} c | (\iterm+j)$, }
    (a) $\neg c | \iterm$ with $ c \nmid \iterm$, 
    (b) $\neg (\iterm_1 = \iterm_2)$ with $\iterm_1 < \iterm_2 \vee \iterm_2 < \iterm_1$, 
    (c) $\neg (\iterm_1 < \iterm_2)$ with $\iterm_2 \le \iterm_1$, 
    (d) $\neg (\iterm_1 \le \iterm_2)$ with $\iterm_2 < \iterm_1$, and so on.
    %
    \item \textit{replace $\ell_{10}(\iterm)$ terms} Repeat the following procedure, until there are no $\ell_{10}(\iterm)$ with $\ivarx$ occurs in $\iterm$: for each occurrence of $\ell_{10}(\iterm)$ such that $\ivarx$ occurs in $\iterm$, introduce a fresh variable, say $\ivarz$, and replace all occurrences of $\ell_{10}(\iterm)$ by $\ivarz$, moreover, add the constraint $10^\ivarz \le \iterm < 10^{\ivarz + 1}$ as a conjunct. Note that if $\iterm$ contains no variables, then $\ell_{10}(\iterm)$ is a constant. In this case, we can also assume that $\iterm$ contains $\ivarx$ and perform the same replacements, which helps in the analysis of complexity in \ref{app:cpx}. Let the resulting formula be $\varphi''$.
    %

    \item \textit{flatten $10^\iterm$ terms} Then repeat the following procedure to $\varphi''$, until for each occurrence of $10^{\iterm}$ with $\ivarx$ occurs in $\iterm$, we have $\iterm = \ivarx$: For each occurrence of the $10^{\iterm}$ in $\varphi''$, such that $\iterm$ contains $\ivarx$ but is not $\ivarx$, introduce a fresh variable, say $\ivarz$, and replace all occurrences of $10^{\iterm}$ by $10^\ivarz$, moreover, add the constraint $\ivarz = \iterm$ as a conjunct. Let $\varphi'''$ denote the resulting formula.  

    \item \textit{$\le$ transformation} Do the following replacements to $\varphi'''$, so that all the atomic formulas in $\varphi^\dag$ are of the form $\iterm_1 \le \iterm_2$ ,$c | \iterm$ or $c \nmid \iterm$: Replace every occurrence of $\iterm_1 \ge \iterm_2$ with $\iterm_2 \le \iterm_1$. Replace every occurrence of $\iterm_1 < \iterm_2$ (resp. $\iterm_1 > \iterm_2$) with $\iterm_1 \le \iterm_2 - 1$ (resp. $\iterm_2 \le \iterm_1-1$). Replace ever occurrence of $\iterm_1 = \iterm_2$ wtih $\iterm_2 \le \iterm_1 \wedge \iterm_1 \le \iterm_2$. Let $\varphi^\dag$ the resulting formula. 

    \item Let $\vec{\ivarz} = \ivarz_1,\ldots, \ivarz_n$ be an enumeration of the freshly introduced variables. Then the result of the normalization procedure is 
    $\exists \vec{\ivarz}\exists \ivarx.\ \varphi^\dag$.
    \end{enumerate}
}
%%%%%%%%%%%%%%%%%%%%%%%%%
\subsection{Proof of Lemma~\ref{lem:exp-ineq}}

Note that $\ell_{10}(n)$ is undefined for $n\le 0$ in $\paexp$. For convenience, we define for all $n\le 0$ that $\ell_{10}(n)=\ell_{10}(\max\{1,n\})=0$ in the following proof. 

\noindent {\bf Lemma~\ref{lem:exp-ineq}}.
\emph{Let  
%
$$
\begin{array}{l}
\tau(\ivarx_{\sigma(i)}, \ldots, \ivarx_{\sigma(1)}, \vec{\ivary}) \Def  \\
\hspace{4mm} 
a_i 10^{\ivarx_{\sigma(i)}}+\sum_{j=1}^{i-1} a_j 10^{\ivarx_{\sigma(j)}} + \sum_{k=1}^{i}b_k \ivarx_{\sigma(k)} \le \iterm(\vec{\ivary}).
\end{array}
$$
%
with $a_i \neq 0$, $A\Def \sum_{j=1}^{i-1}|a_j|$, 
%$B\Def  \sum_{j=1}^{i}|b_j|$, 
$B \Def 2(\ell_{10}(\sum_{j=1}^{i}|b_j|)+3)$,
and $\delta\Def  \ell_{10}(A)+3$. 
\begin{itemize}
    \item If $a_i > 0$, let $\alpha(\vec{\ivary}) \Def \ell_{10}(\iterm(\ivary))- \ell_{10}(a_i)$, then 
    \begin{itemize}
        \item if $\ivarx_{\sigma(i)} \le \alpha(\vec{\ivary})  -1$, $\ivarx_{\sigma(i)} \ge B$ and $\ivarx_{\sigma(i)} \ge \ivarx_{\sigma(i-1)} +\delta $, then $\tau(\ivarx_{\sigma(i)}, \ldots, \ivarx_{\sigma(1)}, \vec{\ivary})$ holds,
        \item if $\ivarx_{\sigma(i)} \ge \alpha(\vec{\ivary})  +2$, $\ivarx_{\sigma(i)} \ge B$ and $\ivarx_{\sigma(i)}  \ge \ivarx_{\sigma(i-1)} +\delta$, then $\tau(\ivarx_{\sigma(i)}, \ldots, \ivarx_{\sigma(1)}, \vec{\ivary})$ \textbf{does not} hold.
    \end{itemize}
    \item If $a_i < 0$, let $\alpha(\vec{\ivary})  \Def \ell_{10}(-\iterm(\ivary))- \ell_{10}(-a_i)$, then 
    \begin{itemize}
        \item if $\ivarx_{\sigma(i)} \le \alpha(\vec{\ivary})  -1$, $\ivarx_{\sigma(i)} \ge B$ and $\ivarx_{\sigma(i)} \ge \ivarx_{\sigma(i-1)} +\delta $, then $\tau(\ivarx_{\sigma(i)}, \ldots, \ivarx_{\sigma(1)}, \vec{\ivary})$ \textbf{does not} hold,
        \item if $\ivarx_{\sigma(i)} \ge \alpha(\vec{\ivary})  +2$, $\ivarx_{\sigma(i)} \ge B$ and $\ivarx_{\sigma(i)} \ge \ivarx_{\sigma(i-1)} +\delta $, then $\tau(\ivarx_{\sigma(i)}, \ldots, \ivarx_{\sigma(1)}, \vec{\ivary})$ holds.
    \end{itemize}
\end{itemize}
}

We need the following proposition for the proof of Lemma~\ref{lem:exp-ineq}.

\begin{proposition} \label{prop:1}
If $n\ge m\ge 1$ and $p \ge 2(\ell_{10}(n)- \ell_{10}(m)+1)$, then 
$n p \le m10^p$ holds.
\end{proposition}

\begin{proof}

First we show that for any $n' \in \Nat$, if $p \ge 2n'$, then $10^p \ge 10^{n'}10^{(p-n')}\ge 10^{n'}10(p-n')\ge 10^{n'}(5p+5(p-2n'))\ge  10^{n'}p$.

If $ p \ge 2 (\ell_{10}(n) - \ell_{10}(m)+1)$, then  $n p \le 10\lambda_{10}(n) p = 10 * 10^{\ell_{10}(n)} p = 10^{\ell_{10}(n)-\ell_{10}(m) + 1}  10^{\ell_{10}(m)} p$.

Because $p \ge 2(\ell_{10}(n)-\ell_{10}(m) + 1)$, we deduce that $10^{\ell_{10}(n)-\ell_{10}(m) + 1}  p \le 10^p$.  Therefore, $10^{\ell_{10}(n)-\ell_{10}(m) + 1}  10^{\ell_{10}(m)} p \le 10^{\ell_{10}(m)} 10^p \le m 10^p$.
We conclude that $np \le m 10^p$.
\end{proof}


%Then we give the proof for Theorem~\ref{thm:exp-ineq}.
\begin{proof}[Proof of Lemma~\ref{lem:exp-ineq}]
We only prove for the case $a_i > 0$, the other case is symmetric. 


Let $A\Def \sum_{j=1}^{i-1}|a_j|$, $B \Def 2(\ell_{10}(\sum_{j=1}^{i}|b_j|)+3)$, and $\delta\Def  \ell_{10}(A)+3$. 

Suppose $\ivarx_{\sigma(i)} \ge B$ and $\ivarx_{\sigma(i)} \ge \ivarx_{\sigma(i-1)} + \delta$. Moreover, let $\alpha(\vec{\ivary}) \Def  \ell_{10}(\iterm(\ivary))- \ell_{10}(a_i)$.

Note that
 \begin{equation} 
   A10^{-\delta} =A 10^{-\ell_{10}(A)-3} = \frac{A}{1000\lambda_{10}(A)} \le \frac{1}{100}.   \label{eq:thm-ineq-1}
 \end{equation}
 
 From $\ivarx_{\sigma(i)} \ge \ivarx_{\sigma(i-1)} + \delta$ and $\ivarx_{\sigma(i-1)} \ge \ldots \ge \ivarx_{\sigma(1)}$, 
we know   
$$-A 10^{\ivarx_{\sigma(i)} - \delta} \le  \sum_{j=1}^{i-1} a_j 10^{\ivarx_{\sigma(j)}} \le A 10^{\ivarx_{\sigma(i)} - \delta}$$
and
$$-(\sum_{j=1}^{i}|b_j|) \ivarx_{\sigma(i)} \le \sum_{k=1}^{i} b_k \ivarx_{\sigma(k)} \le (\sum_{j=1}^{i}|b_j|) \ivarx_{\sigma(i)}.$$ 

Moreover, let $n = 100\sum_{j=1}^{i}|b_j|$, $m = 1$, and $p = \ivarx_{\sigma(i)}$, then 
$n \ge m \ge 1$. From $2(\ell_{10}(n)- \ell_{10}(m)+1) = 2 (\ell_{10}(100 \sum_{j=1}^{i}|b_j|) + 1) = 2 (\ell_{10}(\sum_{j=1}^{i}|b_j|) + 3) = B$, we deduce that $p = \ivarx_{\sigma(i)} \ge B = 2(\ell_{10}(n)- \ell_{10}(m)+1)$.
Then according to Proposition~\ref{prop:1}, 
$$100(\sum_{j=1}^{i}|b_j|)\ivarx_{\sigma(i)} = np \le m 10^p = 10^{\ivarx_{\sigma(i)}}.$$
Thus 
$(\sum_{j=1}^{i}|b_j|)\ivarx_{\sigma(i)}  \le \frac{1}{100} 10^{\ivarx_{\sigma(i)}}.$

If $\ivarx_{\sigma(i)} \ge   \alpha(\vec{\ivary}) + 2$, then 

$
\begin{array}{l}
a_i 10^{\ivarx_{\sigma(i)}}+\sum_{j=1}^{i-1} a_j 10^{\ivarx_{\sigma(j)}} + \sum_{k=1}^{i}b_k \ivarx_{\sigma(k)} \ge \\
a_i 10^{\ivarx_{\sigma(i)}} - A 10^{\ivarx_{\sigma(i)} - \delta} - (\sum_{j=1}^{i}|b_j|) \ivarx_{\sigma(i)}  \ge \\
a_i 10^{\ivarx_{\sigma(i)}} - \frac{1}{100}10^{\ivarx_{\sigma(i)}} - \frac{1}{100} 10^{\ivarx_{\sigma(i)}} = \\
(a_i - \frac{1}{50}) 10^{\ivarx_{\sigma(i)}} \ge (a_i - \frac{1}{50}) 10^{ \alpha(\vec{\ivary}) + 2} = \\
(a_i - \frac{1}{50}) 10^{  \ell_{10}(\iterm(\ivary))- \ell_{10}(a_i) + 2} = \\
\frac{10(a_i - \frac{1}{50})}{10^{\ell_{10}(a_i)} } 10^{ \ell_{10}(\iterm(\ivary))+1} \ge \frac{10(a_i - \frac{1}{50})} {a_i} 10^{ \ell_{10}(\iterm(\ivary))+1} \ge \\
(10 - \frac{1}{5a_i}) 10^{ \ell_{10}(\iterm(\ivary))+1} > 10^{ \ell_{10}(\iterm(\ivary))+1} \ge \iterm(\ivary).
\end{array}
$

Therefore, in this case, $\tau(\ivarx_{\sigma(i)}, \ldots, \ivarx_{\sigma(1)}, \vec{\ivary})$ \emph{does not} hold.

If $\ivarx_{\sigma(i)} \le   \alpha(\vec{\ivary}) - 1$, then 

$
\begin{array}{l}
a_i 10^{\ivarx_{\sigma(i)}}+\sum_{j=1}^{i-1} a_j 10^{\ivarx_{\sigma(j)}} + \sum_{k=1}^{i}b_k \ivarx_{\sigma(k)} \le \\
a_i 10^{\ivarx_{\sigma(i)}} + A 10^{\ivarx_{\sigma(i)} - \delta} + (\sum_{j=1}^{i}|b_j|) \ivarx_{\sigma(i)} \le \\
a_i 10^{\ivarx_{\sigma(i)}} + \frac{A} {10^\delta}10^{\ivarx_{\sigma(i)}} + \frac{1}{100} 10^{\ivarx_{\sigma(i)}} \le \\
(a_i + \frac{1}{100} + \frac{1}{100})10^{\ivarx_{\sigma(i)}} = (a_i + \frac{1}{50}) 10^{\ivarx_{\sigma(i)}} \le \\
(a_i + \frac{1}{50}) 10^{\alpha(\vec{\ivary}) - 1} = (a_i + \frac{1}{50}) 10^{\ell_{10}(\iterm(\ivary))- \ell_{10}(a_i) - 1} = \\
\frac{a_i + \frac{1}{50}} {10^{\ell_{10}(a_i) + 1}} 10^{\ell_{10}(\iterm(\ivary))} = \frac{a_i + \frac{1}{50}} {10 \lambda_{10}(a_i)} 10^{\ell_{10}(\iterm(\ivary))} \le \\
\frac{a_i + \frac{1}{50}} {a_i+1} 10^{\ell_{10}(\iterm(\ivary))} \le 10^{\ell_{10}(\iterm(\ivary))} \le \iterm(\ivary).
\end{array}
$

Therefore,  in this case, $\tau(\ivarx_{\sigma(i)}, \ldots, \ivarx_{\sigma(1)}, \vec{\ivary})$ holds.


\end{proof}

We would like to remark that Lemma~\ref{lem:exp-ineq} still holds when the base of exponential function is changed to any natural number $n\ge 2$.



\hide{
    \subsection{Elimination of exponential occurrences of variables for (in)divisibility atomic formulas}\label{app-div}

    Consider a divisiblilty atomic formula 
    %
    $$
    \begin{array}{l}
    \tau(\ivarx_{\sigma(i)}, \ldots, \ivarx_{\sigma(1)}, \vec{\ivary}) \Def  \\
    d\ \big |\ \big(a_i 10^{\ivarx_{\sigma(i)}} + \sum_{j=1}^{i-1} a_j 10^{\ivarx_{\sigma(j)}} + \sum_{k=1}^{i} b_k \ivarx_{\sigma(k)} 
    + \iterm(\vec{\ivary}) \big)
    \end{array}
    $$
    with $a_i \neq 0$. The indivisibility case can be treated analogously by taking a negation.
    %

    Let $d = 2^{r_1} 5^{r_2}  d_0$ such that $d_0$ is divisible by neither $2$ nor $5$. Moreover, let $r = \max(r_1, r_2)$. Then $d | (10^rd_0)$. 

    If $d_0 = 1$, then $10^r$ is divisible by $d = 2^{r_1}5^{r_2}$. Thus for every $n \ge r$, $d \ |\ 10^n$.  Therefore, in this case, $\tau(\ivarx_{\sigma(i)}, \ldots, \ivarx_{\sigma(1)}, \vec{\ivary})$ is equivalent to 
    \[
    \small
    \begin{array}{l}
    \bigvee \limits_{p = 0}^{r - 1} d\ \big | \big(a_i 10^{p} + \sum_{j=1}^{i-1} a_j 10^{\ivarx_{\sigma(j)}} + b_kp + \sum_{k=1}^{i-1} b_k \ivarx_{\sigma(k)} 
    + \iterm(\vec{\ivary}) \big)\\
    %
    \vee \big(\ivarx_{\sigma(i)} \ge r \wedge d\ \big | \big(\sum_{j=1}^{i-1} a_j 10^{\ivarx_{\sigma(j)}} + \sum_{k=1}^{i} b_k \ivarx_{\sigma(k)} 
    + \iterm(\vec{\ivary}) \big)\big),
    \end{array}
    \]
    where the exponential occurrences of $\ivarx_{\sigma(i)}$ disappear.

    Next, let us assume $d_0 > 1$. Since $10$ and $d_0$ are relatively prime, according to Euler's theorem (cf. \cite{HW80}), $10^{\phi(d_0)} \equiv 1 \bmod d_0$, where $\phi$ is the Euler function. Suppose $10^{\phi(d_0)} = kd_0 + 1$ for some $k \in \Nat$. 
    Then for every $n \in \Nat$ with $n \ge r$, 
    $$
    \begin{array}{l}
    10^{n + \phi(d_0)} \bmod d =10^{n-r} 10^r (kd_0 + 1) \bmod d = \\
    10^{n-r} (k 10^rd_0 + 10^r) \bmod d = \\
    10^{n-r} (0+10^r) \bmod d = 10^n \bmod d.
    \end{array}
    $$

    Then $\tau(\ivarx_{\sigma(i)}, \ldots, \ivarx_{\sigma(1)}, \vec{\ivary})$ is equivalent to 
    \[
    \begin{array}{l}
    \bigvee \limits_{p=0}^{r-1} \tau(\ivarx_{\sigma(i)}, \ldots, \ivarx_{\sigma(1)}, \vec{\ivary})[p/\ivarx_{\sigma(i)}]\ \vee \\
    \left(
    \begin{array}{l}
    \ivarx_{\sigma(i)} \ge r\ \wedge \\
    \bigvee \limits_{q = 0}^{\phi(d_0)-1} 
    \left(
    \begin{array}{l}
    \phi(d_0)\ \big |\ (\ivarx_{\sigma(i)} - r - q)\ \wedge \\
    d\ \big | 
    \left(
    \begin{array}{l}
    a_i 10^{r+q} + \sum_{j=1}^{i-1} a_j 10^{\ivarx_{\sigma(j)}} + \\
    \sum_{k=1}^{i} b_k \ivarx_{\sigma(k)} + \iterm(\vec{\ivary})
    \end{array}
    \right) 
    \end{array}
    \right)
    \end{array}
    \right),
    \end{array}
    \]
    where the exponential occurrences of $\ivarx_{\sigma(i)}$ disappear.
}
\subsection{Some additional optimizations of the decision procedure for {\paexp}}\label{app-opt}

\paragraph{Synchronize the elimination of exponential occurrences of the same variable in different atomic formulas}

Although Lemma~\ref{lem:exp-ineq} is stated for a single atomic formula, the elimination of the exponential occurrences of the same variable in different atomic formulas can actually be synchronized. That is,  let $\alpha^\tau_{1}, \alpha^\tau_{2}, B^\tau, \delta^\tau$ be the constants as stated in the aforementioned under-approximation of an inequality $\tau$, define $\alpha^{\min}_1, \alpha^{\max}_2, B^{\max}, \delta^{\max}$ as the minimum of $\alpha^\tau_1$, the maximum of $\alpha^\tau_2$, the maximum of $B^\tau$, and the maximum of $\delta^\tau$ respectively with $\tau$ ranging over the inequalities of $\varphi$. Then we can use the same constants $\alpha^{\min}_1, \alpha^{\max}_2, B^{\max}, \delta^{\max}$ for different inequalities when eliminating the exponential occurrences of the same variable. 

\paragraph{Avoid the formula-size blow-up by depth-first search}

The {\pa} formula resulting from the elimination of exponential occurrences is essentially a big disjunction of the formulas of small size. If we store this big disjunction naively, then the formula size quickly blows up and exhausts the memory. Instead, we choose to do a depth-first search (DFS) and consider the disjuncts, which are of small sizes, one by one, and solve the satisfiability problem for these disjuncts. If during the search, a satisfiable disjunct is found, then the search terminates and ``SAT'' is reported.

\paragraph{Preprocess with small upper bound}

We believe that if a quantifier-free {\paexp} formula is satisfiable, then most probably it is satisfiable with small values assigned to variables. Consequently, as a preprocessing step, we put a small upper bound, e.g. the biggest constant occurring in the formula, on the values of variables, and perform a depth-first search, so that a model can be quickly found, if there is any. If this preprocessing is unsuccessful, then we continue the search with the greater upper bound $10^u$ for some proper $u \ge 1$.


\subsection{Detailed experiment results on the STRINGHASH benchmark suite}\label{app-exp}

% Please add the following required packages to your document preamble:
% \usepackage{multirow}
\begin{table}[ht]
    \caption{Experimental results on STRINGHASH,  O: Output, S:SAT, U: UNSAT, B: Bounded UNSAT, F: Fail, 
    $\#$: number of problems, $T$: average time in seconds}
    \centering

    \renewcommand{\arraystretch}{1.1}
    \begin{tabular}{|c|c|c|c|c|c|c|c|c|c|}
    \hline
        \multirow{2}{*}{Group }  & \multirow{2}{*}{O} & \multicolumn{2}{c|}{Z3} & \multicolumn{2}{c|}{CVC4} &  \multicolumn{2}{c|}{Trau} & \multicolumn{2}{c|}{$\paexp$} \\
        \cline{3-10}
       &  & $\#$ & $T$ &  $\#$ & $T$ & $\#$ & $T$ &  $\#$ & $T$ \\ 
       \hline
       \cline{1-10}
       \multirow{3}{*}{\scriptsize{$12345(w_1)^+(w_2)^+$}} & S & 5 & 14.0 & 29 & 8.5 & 3 & {\bf  0.1} & {\bf 37} & 9.9 \\
         \cline{2-10}
        & U & 0 & - & 0 & - & {\bf 60} & {\bf 1.3} & {\bf 60} & 47.2 \\
         \cline{2-10}
        & F & 95 & - & 71 & - & 37 & - & {\bf 3} & - \\ \hline
        \cline{1-10}
       \multirow{3}{*}{\scriptsize{$\begin{array}{l}12345(w_1)^+ \\
       \hspace{2mm}(w_2)^+6789\end{array}$}} & S & 11 & 13.0 & 29 & 12.0 & 0 & - & {\bf 37} & {\bf 10.6} \\
        \cline{2-10}
        & U & 0 & - & 0 & - & {\bf 63} & {\bf 1.2} & {\bf 63} & 50.0 \\
         \cline{2-10}
        & F & 89 & - & 71 & - & 37 & - & {\bf 0} & - \\ \hline
        \cline{1-10}
       \multirow{3}{*}{\scriptsize{$(w_1)^+(w_2)^+6789$}} & S & 18 & 24.0 & 30 & 9.3 & 2 & {\bf  0.1} & {\bf 41} & 16.1 \\
      	 \cline{2-10}
        & U & 0 & - & 1 & 4.0 & {\bf 59} & {\bf 2.5} & {\bf 59} & 45.8 \\
         \cline{2-10}
        & F & 82 &  & 69 &  & 39 &  & {\bf 0} &  \\ \hline
        \cline{1-10}
       \multirow{3}{*}{\scriptsize{$12345\Sigma^*_{\sf num}$}} & S & 82 & 8.7 & {\bf 100} & {\bf 2.2} & 28 & 5.9 & {\bf 100} & 18.5 \\
        \cline{2-10}
        & U & 0 & - & 0 & - & 0 & - & 0 & - \\
         \cline{2-10}
        & F & 18 & - & {\bf 0} & - & 72 & - & {\bf 0} & - \\ \hline
        \cline{1-10}
       \multirow{3}{*}{\scriptsize{$12345\Sigma^*_{\sf num}6789$}} & S & 60 & 9.3 & 17 & 7.8 & 3 & {\bf 0.3} & {\bf 100} & 16.0 \\
        \cline{2-10}
        & U & 0 & - & 0 & - & 0 & - & 0 & - \\
         \cline{2-10}
        & F & 40 & - & 83 & - & 97 & - & {\bf 0} & - \\ \hline
        \cline{1-10}
       \multirow{3}{*}{\scriptsize{$\Sigma^*_{\sf num}6789$}} & S & 68 & 5.5 & 27 & 13.0 & 24 & {\bf 9.0} & {\bf 100} & 15.7 \\
        \cline{2-10}
        & U & 0 & - & 0 & - & 0 & - & 0 & - \\
         \cline{2-10}
        & F & 32 & - & 73 & - & 76 & - & {\bf 0} &- \\
        \hline
       \end{tabular}
           \label{table:string}
\end{table}

%%=============================================%%
%% For submissions to Nature Portfolio Journals %%
%% please use the heading ``Extended Data''.   %%
%%=============================================%%

%%=============================================================%%
%% Sample for another appendix section			       %%
%%=============================================================%%

%% \section{Example of another appendix section}\label{secA2}%
%% Appendices may be used for helpful, supporting or essential material that would otherwise 
%% clutter, break up or be distracting to the text. Appendices can consist of sections, figures, 
%% tables and equations etc.

\end{appendices}
}
%%===========================================================================================%%
%% If you are submitting to one of the Nature Portfolio journals, using the eJP submission   %%
%% system, please include the references within the manuscript file itself. You may do this  %%
%% by copying the reference list from your .bbl file, paste it into the main manuscript .tex %%
%% file, and delete the associated \verb+\bibliography+ commands.                            %%
%%===========================================================================================%%

\bibliography{sn-bibliography}% common bib file
%% if required, the content of .bbl file can be included here once bbl is generated
%%\input sn-article.bbl

%% Default %%
%%\input sn-sample-bib.tex%

\end{document}
