%!TEX root = paper.tex

In this section, we analyse the complexity of Point's procedure applied on an existential quantified {\paexp} formula. 

Consider a formula of the form $$\exists \ivarx^1\cdots\exists \ivarx^m.\varphi(\ivarx^1,\cdots,\ivarx^m,\vec{\ivary})$$ with free variables $\vec{\ivary}$. Let $N$ denote the length of the formula. We will prove that the space complexity to eliminate all (existential) quantifiers has a 3-EXPSPACE upper bound. Since Cooper's quantifier elimination algorithm for $\pa$ works as a subprocedure and also has a 3-EXPSPACE upper bound, this bound may not be easily improved.

We first give a brief analysis of the normalization step, in which fresh variables are introduced and the length of the formula grows linearly at most. 
For the rest of the algorithm, we adopt the strategy of Oppen's analysis of Cooper's algorithm (cf. \cite{Oppen73}). The idea is that the upper bound of the formula length can be expressed using the product of the number of atoms, the number of coefficients and the length of the maximum constant. 
The critical point of our analysis is that only coefficients of linear occurrences of quantified variables need to be considered.

\paragraph{Normalization} In the normalization step, we normalize the formula w.r.t quantified variables from $\ivarx^m$ to $\ivarx^1$. Suppose that fresh variables $\ivarx^i_1,\cdots,\ivarx^i_{n_i}$ are introduced for each $\ivarx^i(1\le i\le m)$, the formula then becomes $$\exists \ivarx^1\exists \ivarx^1_1\cdots\exists\ivarx^1_{n_1}\cdots\exists \ivarx^m\exists \ivarx^m_1\cdots\exists\ivarx^m_{n_m}.\varphi'(\ivarx^1,\cdots,\ivarx^m_{n_m},\vec{\ivary}).$$ 

Since the number of newly introduced variables is less than the number of occurrences of exponential and logarithmic functions in the original formula, we have at most $m+\sum_{i=1}^m n_i\le N$ quantified variables after normalization.  

The increase in the length of the formula during normalization comes from two sources, the conjuncts for each introduced variables and the additional formulas for translating $\neq,=$ relations to $\le$. It is not hard to see that both these operations will at most increase the length of the formula by a constant factor. 

In the sequel, to avoid redundant symbols, we still use $m$ to denote the number of quantified variables and $N$ to denote the length of the formula after normalization. Just keep in mind that $m\le N$. 

\hide{
    Given a formula $\exists \ivarx. \varphi(\ivarx, \vec{\ivary})$ with length $n$. After the normalization step, suppose we obtain a formula with $m$ quantified variables $\exists \ivarx_1 \dots\exists \ivarx_m. \varphi'(\ivarx_1,\dots, \ivarx_m, \vec{\ivary})$. We show that the length of the new formula is at most $10n$ and the number of quantified variables $m\le n$. In each sub-step of normalization, we analyse the worst situation: 
    (1) \textit{NNF transformation} suppose all atoms are of the form $\iterm_1=\iterm_2$, then taking negation will double the number of atoms, the length increases to $2n$ at most. 
    (2) \textit{replace $\ell_{10}(\iterm)$ terms} the original formula has at most $n$ terms of the form $\ell_{10}(\iterm)$, for each of them, we introduce a fresh variable and two conjuncts. The formula length increases to $4n$.
    (3) \textit{flatten $10^\iterm$ terms} similar to (2), the formula has at most $n$ such terms and for each term, a fresh variable and a conjunct are added. The formula length increases to $5n$. 
    (4) \textit{$\le$ transformation} similar to (1), here we assume all atoms are of the form $\iterm_1  = \iterm_2$, so the length of the formula with increase to $10n$ at most. 
    
    The analysis for the normalization step is coarse, for example, the worst case in (1)\&(4) or (2)\&(3) can not happen at the same time. However, what we need is that the increased length of the formula is bounded by a constant factor,i.e. 10. In addition, we can also conclude that the number of fresh variables, $m$, is less than $n$ since there are at most $n$ different forms of terms.     
}


\paragraph{Enumeration of linear orders (the outer for-loop)}

Denote the normalized formula by $\exists \vec{\ivarx}. \varphi(\vec{\ivarx},\vec{\ivary})$ with $\vec{\ivarx} = (\ivarx_1,\dots,\ivarx_m)$. According to Point's procedure, we then eliminate the quantified variables one by one: each time select the largest variable among $\ivarx_i,1\le i\le m$, first eliminate the exponential occurrences, then linear occurrences. Each linear order of quantified variables is given by a permutation. For $m$ quantified variables, there are $m!$ possible linear orders, which means the inner for-loop repeats $m!$ times.

\paragraph{Elimination of quantifiers (the inner for-loop)}
Suppose the specified linear order is $\ivarx_m\ge \dots \ge \ivarx_1$. Let $\exists \ivarx_1\dots \exists\ivarx_{m-k}.\varphi_k$ denote the formula obtained by eliminating quantifiers $\exists \ivarx_m,\dots,\exists\ivarx_{m-k+1}$ from $\exists \vec{\ivarx}. \varphi(\vec{\ivarx},\vec{\ivary})$. Let $\varphi'_k$ denote the formula obtained by further eliminating exponential occurrences of $\ivarx_{m-k+1}$ (from $\varphi_k$). Let $\varphi_0$ denote $\varphi$.

Let $c_k$ be the number of distinct divisor $d$ in atoms of the form $d \mid \iterm$ and $d \nmid \iterm$ plus the number of distinct coefficients of \emph{linear occurrences} of quantified variables in $\varphi_k$. Let $s_k$ be the largest constant (including coefficients) and $a_k$ be the number of atomic formulas in $\varphi_k$. Similarly we define $c'_k$, $s'_k$ and $a'_k$ for $\varphi'_k$.

First we analyse the sub-procedure to eliminate exponential occurrences of the inner most quantified variable $\ivarx_m$. We prove the following lemma.

\begin{lemma}\label{lem:cpx exp}
$$c'_0\le {c_0}^2 \qquad s'_0\le m{s_0}^2 \qquad a'_0\le s_0a_0$$ 
\end{lemma}

\hide{\begin{align}
    c'_0&\le c_0^2 \notag \\
    s'_0&\le ms_0^2 \notag\\
    a'_0&\le s_0a_0\notag 
\end{align}}
\begin{proof}
The analysis is divided into two cases by assuming all atomic formulas are of the same form (inequalities atoms or divisibility atoms). 

If all atoms are inequalities of the form in Lemma~\ref{lem:exp-ineq}. We know that each atomic formula $\tau$ with exponential occurrence of $\ivarx_m$ is replaced by a new formula. Note that coefficients of linear occurrences of $\ivarx_i$, for $i\le m-2$, remain unchanged throughout the substitutions, only the coefficients of linear occurrences of $\ivarx_m$ and $\ivarx_{m-1}$ will be changed: constant 1 is introduced as a coefficient for $\ivarx_m$, and if we substitute $\ivarx_m$ by $\ivarx_{m-1}+p$ for some constant $p$, coefficient of linear occurrence of $\ivarx_{m-1}$ will become $b_m+b_{m-1}$ ($b_m,b_{m-1}$ are coefficients for $\ivarx_{m}$ and $\ivarx_{m-1}$ in $\tau$, see Lemma~\ref{lem:exp-ineq} ). Since the new coefficient is obtained by adding two linear coefficient together, we have $c'_1\le {c_0}^2$. Note that $\delta$ and $B$ in Lemma~\ref{lem:exp-ineq} are at most $\ell_{10}(ms_0)$. When we substitute $\ivarx_m$ by $\ivarx_{m-1}+\delta-1$ or by $B-1$, the largest constant in the formula becomes at most $s_0\cdot  10^{\ell_{10}(ms_0)}\le m{s_0}^2$. And an inequality is replaced by at most $4\ell_{10}(ms_0)$ atomic formulas, so $a'_1\le 4\ell_{10}(ms_0)a_0$.

If all atoms are divisibility atomic formulas of the form $d \ \big \vert  \ \iterm$ or $d \ \nmid \ \iterm$. We have $c'_1<2c_0$ because a divisibility atomic formula  will produce at most two forms of atomic formulas $d \ \big \vert  \ \iterm$ and $\phi(d) \ \big \vert  \ \iterm$. In a divisibility atom, any constant in the divident $\iterm$ , say $l$, can be replaced by ($l \bmod d)$. So we have $s'_1\le s_0$. When $d$ is a large prime number, $\phi(d)=d-1$, a divisibility atomic formula is replaced by roughly $d$ atomic formulas, so $a'_1\le s_0a_0$. 

Choose larger upper bounds for $c'_1$, $s'_1$ and $a'_1$ respectively, then the lemma is proved.\end{proof}

When $\varphi_0'$ is transformed into $\varphi_{1}$ by eliminating linear occurrences of $\ivarx_m$, Oppen's analysis gives the following lemma.

\begin{lemma}\cite{Oppen73}\label{lem:cpx oppen}
$$c_1\le {c_0'}^4\qquad
s_1\le {s_0'}^{4c_0'}\qquad
a_1\le {a_0'}^4 {s_0'}^{2c_0'}
$$
\end{lemma}

\hide{    \begin{align}
    c_1&\le {c_0'}^4 \notag \\
    s_1&\le {s_0'}^{4c_0'} \notag\\
    a_1&\le {a_0'}^4 {s_0'}^{2c_0'}\notag 
\end{align} }

Combining Lemma~\ref{lem:cpx exp} and Lemma~\ref{lem:cpx oppen}, we have
$$c_1\le {c_0}^8\qquad
s_1\le (m{s_0}^2)^{4{c_0}^2} \qquad
a_1\le (s_0a_0)^4(m_0{s_0}^2)^{2{c_0}^2} 
$$

\hide{\begin{align}
    c_1&\le {c_0}^8 \notag \\
    s_1&\le (m{s_0}^2)^{4{c_0}^2} \notag\\
    a_1&\le (s_0a_0)^4(m_0{s_0}^2)^{2{c_0}^2}
\end{align} }

By induction on $k$ and assuming $m\le N$, we get

\begin{lemma}
$$c_k\le {c_0}^{8^k}\qquad
s_k\le N^{(4c_0)^{ 8^k}} {s_0}^{(8c_0)^{ 8^k}} \qquad
a_k\le {a_0}^{4^k}n^{(4c_0)^{ 8^k}} s_0^{(8c_0)^{ 8^k}} $$
\end{lemma}

\hide{    \begin{align}
    c_k&\le {c_0}^{8^k} \notag \\
    s_k&\le N^{(4c_0)^{ 8^k}} {s_0}^{(8c_0)^{ 8^k}} \notag\\
    a_k&\le {a_0}^{4^k}n^{(4c_0)^{ 8^k}} s_0^{(8c_0)^{ 8^k}} \notag
\end{align} 
}
If we adopt the assumptions in \cite{Oppen73}, by assuming $c_0\le N$, $a_0\le N$ and $s_0\le N$, the space required to store the quantifier free formula $\varphi_k$, is bounded by the product of the number of linear orders $m!$, the number of atoms $a_k$, the maximum number of constants $2m+2$ per atom, the maximum amount of space $s_k$ to store each constant and some constant $q$. So the space complexity is $O(m! \cdot a_k \cdot (2m+2) \cdot s_k)=O(2^{2^{2^{pn \log n}}})$, which belongs to 3-EXPSPACE.
 
\hide{$${\sf space}\le q \cdot m! \cdot a_k \cdot (2m+2) \cdot s_k \le 2^{2^{2^{pn \log n}}}$$
for some large constant $p$. }

The above analysis is focused on the existential {\paexp} formulas and is sufficient for our string constraints solving context. 
Unfortunately, directly extending our analysis on general {\paexp} formulas with alternating quantifiers by transforming $\forall \ivarx.\varphi$ into $\neg \exists \ivarx.\neg \varphi$ will not give an elementary upper bound. 


\hide{We will explain the main idea without going into details on this issue. To eliminate quantifiers in a formula, say $\forall \ivarx_2\exists \ivarx_1.\varphi(\ivarx_1,\ivarx_2,\vec \ivary)$, we first eliminate $\exists \ivarx_1$, treating $\ivarx_2$ and $\vec \ivary$ both as free variables. The subprocedure to eliminate exponential occurrences of $\ivarx_1$ is at the cost of introducing logarithmic expressions of $\ivarx_2$ and $\vec \ivary$, which in the worst case increases \emph{exponentially} the number of fresh variables in the normalization step w.r.t $\ivarx_2$. We conjecture that the space complexity of quantifier elimination procedure over {\paexp} formulas still has an elementary upper bound, which may require a more detailed analysis or improvements of the algorithm.}
