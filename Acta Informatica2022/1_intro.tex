%!TEX root = paper.tex

The emerging of scripting languages boosted the needs of better approaches and tools to ensure program quality.
Comparing with traditional programming languages, string data type plays a more critical role in its analysis.
String constraint solvers are the engine of modern scripting program analysis techniques. 
Due to the high demand, in recent years, there is a boosting amount of publications on this subject.

However, research progress of string constraint solving has been hampered by many major challenges in both theory and tool implementation aspects (including long-standing open problems). 
Logical theories over strings have to allow string concatenation, which is arguably the most fundamental operation of strings. 
The most celebrated result concerning theories of strings is Makanin’s result on deciding the satisfiability problem for \emph{word equations}~\cite{makanin77}.
A simple example of a word equation is $xaby = ybax$, where $x, y$ are variables, and $a, b \in \Sigma$ are constant letters. 
A word equation is satisfiable if it has a solution, i.e., an assignment that maps variables to strings over the alphabet $\Sigma$ which equates the left-hand side with the right-hand side of the equation.
The correctness proof of Makanin’s algorithm is arguably one of the most complex termination proofs in computer science. 
Makanin’s result can be extended to include \emph{regular constraints} (a.k.a. regular expression matching, e.g., $x \in (ba)^*)$, and arbitrary Boolean connectives.
This extension is called word equations with regular constraints. 
However, the satisfiability problem of word equations together with length constraints (e.g., $\vert x\vert =\vert y\vert +1 \wedge wx=yx$) is still open.
The complexity of the satisfiability of word equations with regular constraints was proven to be PSPACE-complete by Plandowski~\cite{plandowski99}, after decades of improvement of the original algorithm by Makanin.

Satisfiability of word equations is a special instance of Hilbert’s 10th problem. 
In the past, the original motivation of studying word equations was for finding an undecidability proof of Hilbert’s 10th problem. 
However, the motivation is no longer valid since Makanin finds a decision procedure. Recently, driven by the need for program analysis, people started to revisit the problem and its extensions to describe the complete string library APIs in conventional programming languages. Many highly efficient solvers for string constraints, to name a few, CVC4~\cite{cvc4Tool}, Z3~\cite{z3}, Z3-Str3~\cite{zheng2013z3}, S3~\cite{trinh2014s3}, Norn~\cite{abdulla2015norn}, Ostrich~\cite{chen2017decidable}, Sloth~\cite{sloth}, ABC~\cite{aydin2018parameterized}, Stranger~\cite{yu2010stranger}, and Trau~\cite{abdulla2018trau}, are developed in the last decade.
The satisfiability of \emph{string-integer conversion constraints}, e.g., $wx=yx \wedge \vert x\vert  > \parseInt(y)$, has been proven undecidable in~\cite{DayGHMN18}. 
However, this kind of constraints is pervasive in scripting language programs. 
For example, it is common that programs read string inputs from text files and converts a part of the string input to integers.
Even more crucially, in many programming languages, the string-integer conversion is a part of the definition of their core semantics~\cite{ecmascript2019ecmascript}. 
JavaScript, which powers most interactive content on the Web and increasingly server-side code with Node.js, is one of such languages. 

Due to the difficulties in solving string constraints and, in practice, satisfiable string constraints are more critical for automatic testing, one idea is to have separate specialized procedures for solving satisfiable sub-problems. 
Currently, there are two main specialized approaches for proving satisfiability.
The first is to consider only strings of bounded length.
This approach is taken in the first-generation solvers such as Hampi~\cite{KiezunGAGHE12} and Kuluza~\cite{SaxenaAHMMS10}.
Although they are useful in handling many practical cases, they fail short when the all string solutions exceed the selected bound.
For example, a constraint of the form $x.y \neq z  \wedge \vert x\vert  > 2000$ would be quite challenging to handle using those solvers.

One more recent approach is flattening~\cite{Parosh:20:PLDI,AbdullaACDHRR18,AbdullaACDHRR17}.
The idea is to restrict the solution space of string variables to 
%some pattern defined by 
\emph{flat languages} (see Section~\ref{sec:string-solving}). 
The major benefit of considering this class is two-fold.
First, under the restriction, the potential solution space is still infinite, which gives us a higher potential of finding solutions.
For instance, we can find a solution for $x.y \neq z  \wedge \vert x\vert  > 2000$ under a very simple restriction: all variables are in $a^*$, which $a$ is an element of the alphabet.
Second, and more importantly, because we can convert the membership problem of a flat language to the satisfiability problem of a Presburger arithmetic formula, the class of word equation + flat languages + length constraints is decidable.

The paper of Abdulla et al.~\cite{Parosh:20:PLDI} has considered adding string-integer conversion constraints to the above class.
It proposed an algorithm for a restricted form of flat languages and left the support of general flat languages as an open problem.
For string-integer conversion, their approach projects the solution to string-integer conversion to
%, in terms of numerical solutions, 
a finite solution space (in a way similar to the PASS~\cite{goudong2013pass} approach).

In this paper, we give a complete solution to this problem.
We propose a decision procedure for the class of word equation + flat languages + length constraints + string-integer conversion.
The basic idea of our approach can be sketched as follows: we first reduce the satisfiability problem to the corresponding satisfiability problem of the theory of Presburger Arithmetic with exponential functions (denoted by $\paexp$), more precisely, the existential fragment of $\paexp$; then, according to the decidability of $\paexp$, we obtain the decidability of the original satisfiability problem.  

The decidability of $\paexp$ was first shown by Semen\"{o}v in \cite{Semenov84}. Nevertheless, Semen\"{o}v did not provide an explicit decision procedure. To remedy this, in \cite{Point86}, Point presented the first quantifier elimination procedure for the satisfiability of $\paexp$. Partially attributed to the fact that Point's procedure in \cite{Point86} was presented  in a mathematical and dense way, this quantifier elimination procedure has mostly eluded the attentions of computer science researchers\footnote{There are only two citations in Google Scholar.}.  To the best of our knowledge, no implementation based on Point's procedure has been available up to now. 

Aiming at introducing Point's procedure to computer science researchers, we reformulate (and slightly improve) Point's procedure in a way, which, hopefully, is more accessible for computer science researchers (Section~\ref{sec-dec}). This can be seen as another contribution of this paper.
We also analyze the deterministic upper bound of complexity for the first time: given a $\paexp$ formula with only existential quantifiers, eliminating all quantifiers has a 3-EXPSPACE complexity. Furthermore, we propose various optimizations (Section~\ref{sec-opt}) and achieve the first prototypical implementation of Point's procedure (Section~\ref{sec-exp}). 

In fact, other than the theoretical difficulties, in practice, the string-integer conversion is quite challenging for state-of-the-art solvers. 
Here we illustrate a toy example that mimic the ``mining'' step of block-chain construction. 
Essentially, given a string hash function ${\sf hash}(w)$, the goal of the mining step is to find a nonce $n$ such that when inserting $n$ to the text to be protected, say $w_1,w_2$, ${\sf hash}(w_1.n.w_2)$ satisfies a certain pattern, e.g., the last $k$ digits are zeros. 
If $w_1$ or $w_2$ are modified, one needs to compute another $n$ satisfies the desired pattern.
Below we consider a simple hash function:
${\sf hash}(w) = \sum \limits_{i = 1}^n a_i p^{n-i} \bmod m,$
where $p, m \in \Int^+$ with $p, m \ge 2$. 
It is easy to see that ${\sf hash}(w)$ can be seen as a generalization of $\parseInt$ followed by a modulo operation. 
In particular, if $\Sigma = \Sigma_{\sf num}$ and $p = 10$, then ${\sf hash}(w) = \parseInt(w) \bmod m$. Thus, the problem of finding a suitable input $w$ such that the last $k$ digits of ${\sf hash}(w)$ are zeros can be modeled as a string constraint with $\parseInt$. Although the example is seemingly simple, it is already challenging for most state-of-the-art solvers, as shown by our experiment results in Section~\ref{sec-exp}. 
With the optimizations introduced in Section~\ref{sec-opt}, 
our implementation manages to solve several variants of the string-hash examples as well as some randomly generated problem instances better than the state-of-the-art solvers (Section~\ref{sec-exp}). 
 
\paragraph*{Structure}
After the preliminaries in Section~\ref{sec:pre}, we describe how to flatten a string constraint with string-integer conversion to an existential $\paexp$ formula in Section~\ref{sec:string-solving}. We present the quantifier elimination procedure for $\paexp$ in Section~\ref{sec-dec} and its optimizations in Section~\ref{sec-opt}. Finally, we describe the implementation and experiment results in Section~\ref{sec-exp}.


