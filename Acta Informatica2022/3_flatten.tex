%!TEX root = paper.tex

In this section, we first define the class of string constraints with string-integer conversion, denoted by $\strint$. Then we define the extension of Presburger arithmetic with exponential functions, denoted by $\paexp$. Finally, we show how to flatten the string constraints in $\strint$ into the arithmetic constraints in the existential fragment of $\paexp$. 

\subsection{String constraints with string-integer conversion ($\strint$)}

In the sequel, we shall define $\strint$, the class of string constraints with the string-integer conversion function {\parseInt}.

The function {\parseInt} takes a decimal string as input and returns the integer represented by the string
\footnote{The {\parseInt} function in scripting languages, e.g. Javascript, is more general in the sense that the base can be a number between 2 and 36. Although our approach works for arbitrary positive bases, we choose to focus on the base 10 in this paper, for simplicity.}.
For example, ${\parseInt}($`$0123$'$) = {\parseInt}($`$123$'$)=1*10^2+2*10+3 = 123$. 
Note here we use the quotation marks to delimit strings.

Formally, the semantics of the $\parseInt$ function is defined as follows. 
In order to simplify the presentation, we assume all string variables ranging over numerical symbols $\Sigma_{\textit{num}}=\{0,1, \ldots, 9\}$. 
Note that one can easily extend our approach to allow arbitrary finite alphabet. Then ${\parseInt}: \Sigma_{\textit{num}}^+ \mapsto \Nat$ is recursively defined by, for every $w\in \Sigma_{\textit{num}}^+$,
\begin{itemize}
    \item if $w=$`$i$' for $i \in \Sigma_{\textit{num}}$, then ${\parseInt}$$($`$i$'$)=i$;
    \item for $w = w'\cdot$`$i$' for $i \in \Sigma_{\textit{num}}$ with $\len(w') \ge 1$, 
        ${\parseInt}(w) = 10*{\parseInt}(w')+{\parseInt}($`$i$'$)$.
\end{itemize} 
Note that $\parseInt$ is undefined with $\varepsilon$ as the input.

%%%%%%%%%%%%%%%%%%%%%%%%%%%%%%%
\hide{
In $\strint$, there are two types of variables, i.e. the string variables $\svarx,\svary,\ldots \in \svars$ and the integer variables $\ivarx,\ivary,\ldots \in \ivars$.
A $\strint$ formula $\varphi$ is defined by the following rules, where $\len(\sterm)$ denotes the length of a string $\sterm$.
\[
\begin{array}{l c l}
\sterm & \Def & a \mid \svarx \mid \sterm \cdot \sterm, \\
\iterm & \Def & n \mid \ivarx \mid \len(\sterm) \mid \parseInt(\sterm) \mid \iterm + \iterm \mid \iterm - \iterm,\\
\varphi & \Def & \sterm = \sterm \mid x \in \Aut \mid \iterm\ \op\ \iterm \mid \varphi \wedge \varphi \mid \varphi \vee \varphi \mid \neg \varphi,
\end{array}
\]
where $a \in (\Sigma_{\textit{num}})_\varepsilon$, $n \in \Nat$, $\Aut$ is an FA, and $\op \in \{=, < , >, \le, \ge\}$. Let us call $\sterm$ as string terms, $\iterm$ as integer terms, $\sterm = \sterm$ as string equality constraints, $x \in \Aut$ as regular constraints, $\iterm \op\ \iterm$ as arithmetic constraints. 
Let  $\svar(\varphi)$ and $\ivar(\varphi)$ denote the set of string variables and integer variables respectively occurring in $\varphi$.
}
%%%%%%%%%%%%%%%%%%%%%%%%%%%%%%%%%%%


In $\strint$, there are two types of variables, i.e. the string variables $\svarx,\svary,\ldots \in \svars$ and the integer variables $\ivarx,\ivary,\ldots \in \ivars$.
The syntax of $\strint$ is defined as follows:
a string term, denoted by $\sterm$, is of the form 
$$\sterm\  \Def\  a \mid \svarx \mid \sterm \cdot \sterm,$$
an integer term, denoted by $\iterm$, is of the form
$$\iterm\  \Def\  c \mid \ivarx \mid \len(\sterm) \mid \parseInt(\sterm) \mid \iterm + \iterm \mid \iterm - \iterm,$$
and a $\strint$ formula, denoted by $\varphi$ is of the form 
$$\varphi\ \Def\ \sterm = \sterm \mid \sterm \in \Aut \mid \iterm\ \op\ \iterm \mid \varphi \wedge \varphi \mid \varphi \vee \varphi \mid \neg \varphi$$
where $a \in (\Sigma_{\textit{num}})_\varepsilon$, $c \in \mathbb Z$, $\Aut$ is an FA, $\op \in \{=, < , >, \le, \ge\}$ and $\len(\sterm)$ denotes the length of a string $\sterm$. 
Let us call $\sterm = \sterm$ as string equality constraints, $\sterm \in \Aut$ as regular constraints, $\iterm \op \iterm$ as arithmetic constraints. 
Let  $\svar(\varphi)$ and $\ivar(\varphi)$ denote the set of string variables and integer variables occurring in $\varphi$ respectively.

The $\strint$ formulas are interpreted on $I=(I_s, I_i)$ where $I_s$ is a partial function from $\svars$ (the set of string variables) to $\Sigma^*$ and $I_i$ is a partial function from $\ivars$ (the set of integer variables) to $\mathbb Z$. 
% nat -> Z
Moreover, it is required that the domains of $I_s, I_i$ are finite. Given $I = (I_s, I_i)$, the interpretations of string terms, integer terms, as well as the formulas $\varphi$ under $I$ are easy to comprehend, thus omitted to avoid tediousness. 
For $\varphi \in \strint$ and $I = (I_s, I_i)$, $I$ satisfies $\varphi$ if the interpretation of $\varphi$ under $I$ is $\ltrue$.
Let us use $\lVert \varphi \rVert$ to denote the set of $I = (I_s, I_i)$ satisfying $\varphi$.

\begin{example} Given a FA $\Aut$, the constraint
$$\svarx \in \Aut \ \wedge\ 
\parseInt(\svarx) = 109\ivarx\ \wedge\ 
\len(\svarx) < 100$$
is a $\strint$ formula.
\end{example}

The satisfiability problem of $\strint$ is to decide for a given constraint $\varphi \in \strint$,
whether $\lVert  \varphi \rVert \neq \emptyset$.

\subsection{Presburger Arithmetic with Exponential Functions (\paexp)}

{\paexp} extends Presburger arithmetic with two partial functions, the \emph{exponential} function $10^\ivarx$ and the \emph{integer logarithmic} function $\ell_{10}(\ivarx)$ (cf. \cite{Point86}). The function $\ell_{10}(\ivarx)$ is defined as follows: for $n \ge 1$, $\ell_{10}(n) = m$ if $10^m \le n < 10^{m+1}$. Note that $10^\ivarx$ is undefined for $\ivarx<0$ and $\ell_{10}(\ivarx)$ is undefined for $\ivarx\le 0$.

The syntax of {\paexp} is obtained from that of {\pa} by adding $10^\iterm$ and $\ell_{10}(\iterm)$ to the definition of terms. An {\paexp} term, denoted by $\iterm$, is of the form
$$\iterm\ \Def\ c \mid \ivarx \mid \iterm + \iterm \mid \iterm - \iterm \mid 10^{\iterm} \mid \ell_{10}(\iterm) $$
where $\ivarx$ is an integer variable and $c\in \mathbb Z$ is an constant integer. In addition, we require that $10^\iterm$ and $\ell_{10}(\iterm)$ terms are well-defined, which restricts the interpretations of $\iterm$ therein.

A formula of {\paexp}, denoted by $\phi$, is of the form
$$\phi \ \Def \ \iterm \ \op\ \iterm \mid c \vert \iterm \mid c \nmid \iterm \mid \neg \phi \mid \phi \wedge \phi \mid \phi \vee \phi \mid \exists \ivarx.\ \phi \mid \forall \ivarx.\ \phi,$$
where $\op \in \{=, < , >, \le, \ge\}$. 

The semantics of {\paexp} are defined similarly to that of {\pa} with the only difference that partial functions are required to be well-defined. The quantifier-free and existential {\paexp} formulas are defined similarly to {\pa} as well.
We also assume that quantifier-free {\paexp} formulas are free of negations, for the same reason we have explained for {\pa}.

\hide{
The semantics of {\paexp} are defined similarly to {\pa} with the only difference that variables are interpreted over $\Nat$ rather than $\mathbb Z$, that is, an interpretation is a function $I:\free(\phi)\to \mathbb \Nat$. 
This restriction ensures that terms $10^\iterm$ and $\ell_{10}(\iterm)$ are well-defined: In the \emph{normalization} step of the quantifier elimination procedure, for each $\ell_{10}(\iterm)$ term, we will replace all its occurrences by a fresh variable $\ivarz$, and add a conjunct $10^\ivarz \le \iterm < 10\cdot 10^\ivarz$. 
Then $\ivarz\in\Nat$ guarantees that $\iterm \ge 1$ is always satisfied.
Exponential functions are treated similarly
(more details can be found in Section~\ref{qe-norm}).
Note that we can extend the interpretation to $\mathbb Z$ since each integer can be written as the difference of two positive integers.
}

For convenience, when working with exponential and logarithmic functions, we use the notation $\lambda_{10}(\iterm)$ to denote $10^{\ell_{10}(\iterm)}$. It is easy to show that for all $n \ge 1$, $\lambda_{10}(n) \le n < 10\lambda_{10}(n)$ holds.

\begin{example}
$10^{10^\ivarx}+10^{\ivarx+ \ivary} + 3\ivary    = 2 \ivarz +1 $ is an {\paexp} formula.
\end{example} 
 
\subsection{Flattening $\strint$ into $\paexp$}

We first recall the flattening approach for string constraints in \cite{Parosh:20:PLDI}, then show how to extend it to deal with {\parseInt}.

A \emph{flat domain restriction} for a string constraint $\varphi\in \strint$ is a function $\flatfun_\varphi$ that maps each string variable $\svarx \in \svar(\varphi)$ to a flat language $(w_{\svarx,1})^* \cdots (w_{\svarx, k_\svarx})^*$, where $w_{\svarx, i} \in \Sigma_{\textit{num}}^+$ for every $i \in [k_\svarx]$. 
The flattened semantics of $\varphi \in \strint$ is defined as $\llbracket \varphi \rrbracket_{\flatfun_\varphi} = \{(I_s, I_i) \in \llbracket \varphi  \rrbracket \mid \forall \svarx \in \svar(\varphi).\ I_s(x) \in  \flatfun_\varphi(\svarx)\}$.  

%%%%%%%%%%%%%%%%%%%%%%%%%%%%%%%%%%%%%%%%%%
\hide{
The flattening of $\varphi \in \strint$ under a flat domain restriction $\flatfun_\varphi$ is a {\paexp} formula, denoted by $\flatten_{\flatfun_\varphi}(\varphi)$, that encodes its flattened semantics.
More concretely, $\flatten_{\flatfun_\varphi}(\varphi)$ is a formula over the integer variables $\ivar(\varphi)$,  and flattening variables $\pvar_{\flatfun_\varphi}(\varphi) = \bigcup_{\svarx \in \svar(\varphi)} \pvar_{\flatfun_\varphi}(\svarx)$, where $\pvar_{\flatfun_\varphi}(\svarx) = \{l_{\svarx,i} \mid i \in [k_\svarx]\}$, plus some other auxiliary variables, such that 
$$\llbracket \varphi \rrbracket_{\flatfun_\varphi} =\decode_{\flatfun_\varphi}(\llbracket \flatten_{\flatfun_\varphi}(\varphi) \rrbracket \vert_{\ivar(\phi) \cup \pvar_{\flatfun_\varphi}(\varphi)}).$$
}
%%%%%%%%%%%%%%%%%%%%%%%%%%%%%%%%%%%%%%%%%

Under a flat domain restriction $\flatfun_\varphi$, the flattening of $\varphi \in \strint$ is an $\paexp$ formula, denoted by $\flatten_{\flatfun_\varphi}(\varphi)$, that encodes the flattened semantics $\llbracket \varphi \rrbracket_{\flatfun_\varphi}$. 
The idea is that $x$ can be expressed as $(w_{\svarx,1})^{l_{x,1}} \cdots (w_{\svarx, k_\svarx})^{l_{x,k_x}}$ with $l_{x,i}$ denoting the number of iterations of $w_{x,i}$,  
so we can eliminate the occurrences of $x$ in $\varphi$ by introduce a set of integer variables $\pvar_{\flatfun_\varphi}(\svarx) = \{l_{\svarx,i} \mid i \in [k_\svarx]\}$. 
More concretely, $\flatten_{\flatfun_\varphi}(\varphi)$ is a formula over the integer variables $\ivar(\varphi)$,  and flattening variables $\pvar_{\flatfun_\varphi}(\varphi) = \bigcup_{\svarx \in \svar(\varphi)} \pvar_{\flatfun_\varphi}(\svarx)$ plus some other auxiliary variables, such that the interpretations $I_e: \ivar(\varphi) \cup \pvar_{\flatfun_\varphi}(\varphi) \rightarrow \mathbb Z$ of $\llbracket \flatten_{\flatfun_\varphi}(\varphi) \rrbracket$and interpretations $(I_s, I_i)$ of $\llbracket \varphi \rrbracket_{\flatfun_\varphi}$ have the following correspondence 

\begin{itemize}
    \item  
    for every $ \svarx \in \svar(\varphi)$, $I_s(\svarx) = w_{\svarx,1}^{I_e(l_{\svarx,1})} \ldots  w_{\svarx,k_\svarx}^{I_e(l_{\svarx,k_\svarx})}$, where $I_e(l_{x,i})\ge 0$ for $1\le i\le k_x$.
    %
    \item for every $ \ivarx \in \ivar(\varphi)$, $I_i(\ivarx) = I_e(\ivarx)$.
\end{itemize}

%%%%%%%%%%%%%%%%%%%%%%%%%%%%%%
\hide{
$$\llbracket \varphi \rrbracket_{\flatfun_\varphi} =\decode_{\flatfun_\varphi}(\llbracket \flatten_{\flatfun_\varphi}(\varphi) \rrbracket \vert_{\ivar(\phi) \cup \pvar_{\flatfun_\varphi}(\varphi)}).$$

The decoding function above decodes an interpretation of integer and flattening variables $I_e: \ivar(\varphi) \cup \pvar_{\flatfun_\varphi}(\varphi) \rightarrow \mathbb Z$ as a set $\decode_{\flatfun_\varphi}(I_e)$ of interpretations of the $\varphi$'s integer and string variables $(I_s, I_i)$ with $I_s: \svar(\varphi) \rightarrow \Sigma^*$ and $I_i: \ivar(\phi) \rightarrow \Nat$ such that 
}
%%%%%%%%%%%%%%%%%%%%%%%%%%%%%%

The formula $\flatten_{\flatfun_\varphi}(\varphi)$ is constructed inductively on the structure of $\varphi$: $\flatten_{\flatfun_\varphi}(\varphi_1\ \mathfrak{o}\ \varphi_2) = \flatten_{\flatfun_\varphi}(\varphi_1) \ \mathfrak{o}\  \flatten_{\flatfun_\varphi}(\varphi_2)$, where $\mathfrak{o} \in \{\wedge, \vee\}$, and $\flatten_{\flatfun_\varphi}(\neg \varphi_1) = \neg \flatten_{\flatfun_\varphi}(\varphi_1)$. Therefore, it is sufficient to show how to construct $\flatten_{\flatfun_\varphi}(\varphi)$ for atomic constraints $\varphi$. 
In the sequel, we will show how to construct $\flatten_{\flatfun_\varphi}(\iterm_1 \op \iterm_2)$ where $\parseInt(\sterm)$ may occur in $\iterm_1$ or $\iterm_2$. The construction of $\flatten_{\flatfun_\varphi}(\varphi)$ for the other atomic constraints is omitted because it is essentially the same as that in \cite{Parosh:20:PLDI} (we summarize as Theorem~\ref{thm:string}).

\begin{theorem}\label{thm:string}\cite{Parosh:20:PLDI} The satisfiability of Boolean combinations of string equality constraints, regular constraints and arithmetic constraints under flat domain restrictions can be reduced to the satisfiability of existential {\pa}   formulas, thus is decidable.
\end{theorem}

For simplicity, we assume that each occurrence of $\parseInt$ (resp. $\len(t)$) in $\iterm_1 \op \iterm_2$ is of the form $\parseInt(\svarx)$ (resp. $\len(\svarx)$) for a string variable $\svarx$. (Otherwise, we can introduce a fresh variable $\svarx'$ for $\sterm$ in $\parseInt(\sterm)$ or $\len(\sterm)$ and add the constraint $\svarx' = \sterm$.)
Then $\flatten_{\flatfun_\varphi}(\iterm_1 \op \iterm_2)$ is obtained from $\iterm_1 \op \iterm_2$ by replacing $\parseInt(\svarx)$ with $\flatten_{\flatfun_\varphi}(\parseInt(\svarx))$ and $\len(\svarx)$ with $\flatten_{\flatfun_\varphi}(\len(\svarx))$, where 
\begin{itemize}
\item 
$\flatten_{\flatfun_\varphi}(\parseInt(\svarx)) \Def \iterm_{\svarx,1}$  such that $(\iterm_{\svarx, i})_{i \in [k_\svarx]}$ are inductively defined as follows: 
\begin{itemize}
\item for $i = k_\svarx$, 
$$\iterm_{\svarx, i} = \frac{\parseInt(w_{\svarx, k_\svarx}) (10^{\len(w_{\svarx,k_\svarx})l_{\svarx, k_\svarx}} -1 )}{(10^{\len(w_{\svarx, k_\svarx}) } -1 )}$$ 
%
\item for $i \in [k_\svarx-1]$, 
%
$$ \hspace{-1.2cm} \iterm_{\svarx, i} =  \frac{\parseInt(w_{\svarx, i}) (10^{\len(w_{\svarx, i})  l_{\svarx, i} } -1 ) 10^{\big(\sum_{i+1\le j\le k_x} \len(w_{\svarx, j})  l_{\svarx, j}\big)}} {(10^{\len(w_{\svarx, i}) } -1 )} + \iterm_{\svarx, i+1}$$
\end{itemize}
%
\item Notice that here $\len( w_{\svarx, \textunderscore } )$ are constants while $l_{\svarx, \textunderscore}$ are (flattening) variables. So we have 
$$\flatten_{\flatfun_\varphi}(\len(\svarx)) \Def \sum \limits_{i \in [k_{\svarx}]} \len(w_{\svarx, i})  l_{\svarx, i}.$$ 
\end{itemize} 

The following example helps to illustrate the main idea of the flattening technique.

\begin{example}
Suppose $\parseInt(\svarx) = 2\ivarx$ is an atomic constraint and $\flatfun_\varphi(\svarx) = 1^*2^*$. Then 
\[
\small
\begin{array}{l l}
& \flatten_{\flatfun_\varphi}(\parseInt(\svarx)  =  2\ivarx)  \\
\Def & 1 \frac{10^{l_{\svarx,1}}-1}{10-1}10^{l_{\svarx,2}}  + 2 \frac{10^{l_{\svarx,2}}-1}{10-1} = 2\ivarx   \\
\equiv & 10^{l_{\svarx,1}+l_{\svarx,2}} - 10^{l_{\svarx,2}}  + 2*10^{2l_{\svarx,2}}- 2 = 18\ivarx \\
\equiv & 10^{l_{\svarx,1}+l_{\svarx,2}} +  10^{l_{\svarx,2}} = 18\ivarx + 2.
\end{array}
\]
\end{example}

By Theorem~\ref{thm:string}, the above reduction, and the decidability of $\paexp$'s satisfiability (see Theorem~\ref{thm-paexp}), we have 
\begin{theorem} \label{thm:string-parInt}
    The satisfiability of  $\strint$ under flat domain restrictions can be reduced to the satisfiability of existential $\paexp$ formulas, and thus is decidable.
\end{theorem}
