%!TEX root = paper.tex

Semënov first proved that  {\paexp} admits quantifier elimination in \cite{Semenov84}, thus its satisfiability problem is decidable. However, Semënov did not give a concrete quantifier elimination procedure. Remedying this, Point proposed a quantifier elimination procedure for {\paexp} in \cite{Point86}. In this section, we describe how Point's quantifier elimination procedure works. 

\vspace*{-3mm}

\begin{theorem}[\cite{Point86}]
\label{thm-paexp}
{\paexp} admits quantifier elimination. 
\end{theorem}
\vspace*{-3mm}

Compared to \cite{Point86}, the presentation here is more accessible to computer science researchers. Moreover, the procedure presented here slightly improves Point's procedure, in the following two aspects: 1) DNF (disjunctive normal form) was required in Point's procedure, which is not required here, 2) in Point's procedure, the divisibility constraints produced by the elimination of linear occurrences of a variable should be converted to equality constraints (by introducing fresh variables) before the elimination of exponential occurrences of some other variable, which is unnecessary here, since the divisibility constraints are directly dealt with in the elimination of exponential occurrences of variables. 

As $\forall \ivarx. \ \varphi$ is equivalent to $\neg \exists \ivarx. \ \neg \varphi$, to prove Theorem~\ref{thm-paexp}, we only need to show that every existentially quantified $\paexp$ formula $\exists \ivarx.\ \varphi \in \paexp$, where $\varphi$ is quantifier-free, can be transformed into an equivalent quantifier-free formula $\varphi' \in \paexp$. 

An informal sketch of Point's procedure is presented in Algorithm~\ref{qe-algo}.

\hide{Before a formal description of the quantifier elimination procedure, let us use a simple example to illustrate the main idea.}

\begin{algorithm2e}[ht]\label{qe-algo}
    \SetAlgoLined
    \KwIn{{\paexp} formula $\exists \ivarx_1.\varphi$, where $\varphi$ is quantifier-free}
    \KwOut{an equivalent quantifier-free formula}
    \SetKwFor{ForOrder}{For each}{do}{end}
    \SetKwRepeat{Do}{do}{while}
    
    $\varphi \gets \textrm{normalize } \varphi \textrm{ w.r.t }\ivarx_1 $ (Subsection~\ref{qe-norm});

    \tcp{suppose fresh variables $\ivarx_2,\cdots,\ivarx_n$ are introduced}
    $\mathcal S_n\gets$ enumerate linear orders among $\ivarx_1,\cdots,\ivarx_n$ (Subsection~\ref{qe-enum})\;
    \ForOrder{$\sigma\in \mathcal S_n$}
    {
        $\varphi'_\sigma  \gets \exists \vec{\ivarx}.\ \varphi' \wedge \bigwedge \limits_{i \in [n-1]} \ivarx_{\sigma(i)} \le \ivarx_{\sigma(i+1)}$\; 
        $i\gets n$\;
        \For{$i> 0$}{
            eliminate exponential occurrences of $\ivarx_{\sigma(i)}$ in $\varphi'_\sigma$ (Subsection~\ref{qe-elim-exp})\;
            eliminate linear occurrences of $\ivarx_{\sigma(i)}$ in $\varphi'_\sigma$ (Cooper's QE algorithm)\;
            \tcp{variable $\ivarx_{\sigma(i)}$ is removed}
        $i\gets i-1$\;
        }
        \tcp{all quantified variables in $\varphi'$ are removed}
        $\varphi''_\sigma\gets \varphi'_\sigma$\;
    }
    
    \Return $\bigvee_{\sigma\in\mathcal S_n}\varphi''_\sigma$\;
    
    \caption{Quantifer elimination for \paexp}
\end{algorithm2e}


\hide{
\subsection{An overview of the quantifier elimination procedure}

Consider the formula $\varphi \Def \exists \ivarx_2.\ 10^{\ivarx_1 + \ivarx_2} - 10^{\ivarx_2} \le \ivary + 1001$. 

At first, we \emph{normalize} $\varphi$ w.r.t. $\ivarx_2$ by introducing a fresh variable $\ivarx_3$ for $\ivarx_1 + \ivarx_2$ and get the formula 
$$\varphi_1 \Def \exists \ivarx_3 \exists \ivarx_2.\ 10^{\ivarx_3} - 10^{\ivarx_2} \le \ivary + 1001 \wedge \ivarx_3 = \ivarx_1 + \ivarx_2.$$

Then, we enumerate different \emph{orders} of the quantified variables, i.e. $\ivarx_2$ and $\ivarx_3$. Since $\ivarx_3 = \ivarx_1 + \ivarx_2$, there is only one possible order, that is, $\ivarx_3 \ge \ivarx_2$.

Next, we illustrate how to eliminate the quantifier $\exists \ivarx_3$, assuming $\ivarx_3 \ge \ivarx_2$. The elimination of $\exists \ivarx_2$ is similar and simpler, thus omitted.

The elimination of $\exists \ivarx_3$ consists of two steps, namely, eliminating the exponential occurrences of $\ivarx_3$ first, and the linear occurrences next.

The main idea of the elimination of the exponential occurrences of $\ivarx_3$ is to observe that if $\ivarx_3 \ge \ell_{10}(\ivary+1001)+2$ and $\ivarx_3 \ge \ivarx_2 + 3$, then $10^{\ivarx_3} - 10^{\ivarx_2}$ is dominated by $10^{\ivarx_3}$, that is, $10^{\ivarx_3} - 10^{\ivarx_2} \ge 10^{\ivarx_3} - 10^{\ivarx_3 - 3} = (1-10^{-3}) 10^{\ivarx_3} \ge 10^{\ell_{10}(\ivary+1001)+1} = 10\lambda_{10}(\ivary+1001) > \ivary+1001$ (see Lemma~\ref{lem:exp-ineq} for the choice of the constants $2$ and $3$ in $\ivarx_3 \ge \ell_{10}(\ivary+1001)+2$ and $\ivarx_3 \ge \ivarx_2 + 3$.). 
Therefore, a necessary condition for $10^{\ivarx_3} - 10^{\ivarx_2} \le \ivary + 1001$ is that either $\ivarx_3 \le \ell_{10}(\ivary+1001)+1$ or $\ivarx_3 \le \ivarx_2 + 2$ holds.  

%\yfc{why $+2$ and $+3$ in $\ivarx_3 \ge \ell_{10}(\ivary+1001)+2$ and $\ivarx_3 \ge \ivarx_2 + 3$?}
\begin{itemize}
\item If $\ivarx_3 \le \ell_{10}(\ivary+1001)+1$, then we distinguish between whether $\ivarx_3 \le \ell_{10}(\ivary+1001)$ or  $\ivarx_3 = \ell_{10}(\ivary+1001)+1$. 
\begin{itemize}
\item If $\ivarx_3 \le \ell_{10}(\ivary+1001)$, then $10^{\ivarx_3} - 10^{\ivarx_2} \le 10^{\ell_{10}(\ivary+1001)} = \lambda_{10}(\ivary+1001) \le \ivary + 1001$. In this case, $10^{\ivarx_3} - 10^{\ivarx_2} \le \ivary + 1001 \wedge \ivarx_3 = \ivarx_1 + \ivarx_2$ can simplified into $\ltrue$.
%
\item If $\ivarx_3 = \ell_{10}(\ivary+1001)+1$, then $10^{\ivarx_3} - 10^{\ivarx_2} \le \ivary + 1001$ can be turned into $10^{\ell_{10}(\ivary+1001)+1} - 10^{\ivarx_2} \le \ivary + 1001 \equiv 10 \lambda_{10}(\ivary+1001) - 10^{\ivarx_2} \le \ivary + 1001 $.
\end{itemize} 
%
\item If $\ivarx_3 \le \ivarx_2 + 2$, then $\ivarx_3 = \ivarx_2 + j$ for $j \in \{0,1,2\}$. Thus $10^{\ivarx_3} - 10^{\ivarx_2} \le \ivary + 1001$ can be transformed to $\bigvee \limits_{j \in \{0,1,2\}} 10^{\ivarx_2 + j} - 10^{\ivarx_2} \le \ivary + 1001$.
\end{itemize}

To summarize, $\varphi_1$ is transformed into 
\[
\small
\begin{array}{l}
\varphi_2 \Def \exists \ivarx_3 \exists \ivarx_2. \\
\begin{array}{l}
%\vspace{2mm}
\left(
\begin{array}{l}
\ivarx_3 \le \ell_{10}(\ivary+1001)\ \vee \\
\left(
\begin{array}{l}
\ivarx_3 = \ell_{10}(\ivary+1001)+1\ \wedge \\
10 \lambda_{10}(\ivary+1001) - 10^{\ivarx_2} \le \ivary + 1001 
\end{array}
\right) \vee \\
%
 \bigvee \limits_{j \in \{0,1,2\}}  \left(\ivarx_3 = \ivarx_2 +j \wedge 10^{\ivarx_2 + j} - 10^{\ivarx_2} \le \ivary + 1001\right)
\end{array}
\right) 
\wedge \\
 \ivarx_3 = \ivarx_1 + \ivarx_2.
 \end{array}
\end{array}
\]
Note that $\varphi_2$ contains \emph{only linear} occurrences of $\ivarx_3$.

Finally, we can eliminate the linear occurrences of $\ivarx_3$, thus the quantifier $\exists \ivarx_3$, by applying the quantifier elimination algorithm of {\pa}, e.g. Cooper's algorithm in \cite{Cooper72}. The elimination of $\ivarx_3$ in $\varphi_2$ here is simple, with $\ivarx_3$ replaced by $\ivarx_1 + \ivarx_2$. 

In the remainder of this section, we are going to describe the aforementioned steps of the decision procedures: Normalization, the enumeration of the variable orders, 
the elimination of the exponential occurrences of variables. The elimination of the  linear occurrences of variables is essentially the quantifier elimination of the {\pa} and omitted.

Let us assume that $\varphi \Def \exists \ivarx.\ \varphi'(\ivarx, \vec{\ivary})$, where $\varphi'$ is a quantifier-free formula. 

%\vspace*{-3mm}

}
\subsection{Normalization}\label{qe-norm}

The normalization step comprises the following sub-steps.
\begin{enumerate}

\hide{ % do not allow \neg in PAexp
\item \textit{NNF transformation} At first, we transform $\varphi'(\ivarx,\vec{\ivary})$ into the NNF (negation normal form). Moreover, we remove the occurrences of $\neg$ by replacing 
(a) $\neg c \vert  \iterm$ with $ c \nmid \iterm$, 
(b) $\neg (\iterm_1 = \iterm_2)$ with $\iterm_1 < \iterm_2 \vee \iterm_2 < \iterm_1$, 
(c) $\neg (\iterm_1 < \iterm_2)$ with $\iterm_2 \le \iterm_1$, 
(d) $\neg (\iterm_1 \le \iterm_2)$ with $\iterm_2 < \iterm_1$, and so on.
}

\item \textit{replace $\ell_{10}(\iterm)$ terms} Repeat the following procedure, until there are no $\ell_{10}(\iterm)$ with $\ivarx$ occurs in $\iterm$: for each occurrence of $\ell_{10}(\iterm)$ such that $\ivarx$ occurs in $\iterm$, introduce a fresh variable, say $\ivarz$, and replace all occurrences of $\ell_{10}(\iterm)$ by $\ivarz$, moreover, add the constraint $10^\ivarz \le \iterm < 10^{\ivarz + 1}$ as a conjunct. Note that if $\iterm$ contains no variables, then $\ell_{10}(\iterm)$ is a constant. In this case, we can also assume that $\iterm$ contains $\ivarx$ and perform the same replacements, which helps in the analysis of complexity in \ref{sec-cpx}. Let the resulting formula be $\varphi''$.
%

\item \textit{flatten $10^\iterm$ terms} Then repeat the following procedure to $\varphi''$, until for each occurrence of $10^{\iterm}$ with $\ivarx$ occurs in $\iterm$, we have $\iterm = \ivarx$: For each occurrence of the $10^{\iterm}$ in $\varphi''$, such that $\iterm$ contains $\ivarx$ but is not equal to $\ivarx$, introduce a fresh variable, say $\ivarz$, and replace all occurrences of $10^{\iterm}$ by $10^\ivarz$, moreover, add the constraint $\ivarz = \iterm$ as a conjunct. Let $\varphi'''$ denote the resulting formula.  

\item \textit{$\le$ transformation} Do the following replacements to $\varphi'''$, so that all the atomic formulas in $\varphi^\dag$ are of the form $\iterm_1 \le \iterm_2$, $c \vert  \iterm$ or $c \nmid \iterm$: Replace every occurrence of $\iterm_1 \ge \iterm_2$ with $\iterm_2 \le \iterm_1$. Replace every occurrence of $\iterm_1 < \iterm_2$ (resp. $\iterm_1 > \iterm_2$) with $\iterm_1 \le \iterm_2 - 1$ (resp. $\iterm_2 \le \iterm_1-1$). Replace ever occurrence of $\iterm_1 = \iterm_2$ wtih $\iterm_2 \le \iterm_1 \wedge \iterm_1 \le \iterm_2$. Let $\varphi^\dag$ denote the resulting formula. 

\item Let $\vec{\ivarz} = \ivarz_1,\ldots, \ivarz_n$ be an enumeration of the freshly introduced variables. Then the result of the normalization procedure is 
$\exists \vec{\ivarz}\exists \ivarx.\ \varphi^\dag$.
\end{enumerate}

Intuitively, the normalization step first removes all occurrences of $\ell_{10}(\iterm)$ where $\ivarx$ occurs in $\iterm$, by encoding them with the exponential function. Moreover, for each occurrence of $10^\iterm$ such that $\ivarx$ occurs in $\iterm$, it introduces a fresh variable $\ivarz$, replaces $10^\iterm$ with $10^\ivarz$, and adds the equality $\ivarz = \iterm$. All equalities and inequalities will be rewritten into the form $\iterm_1\le\iterm_2$. Finally, add quantifiers for the introduced fresh variables.

After the normalization, the resulting formula is of the following shape: 1) it contains no occurrences of $\ell_{10}(\iterm)$ such that $\ivarx$ occurs in $\iterm$, 2)  it contains no occurrences of $10^\iterm$ such that $\ivarx$ occurs in $\iterm$, but $\iterm \neq \ivarx$, 3) all the atomic formulas are of the form $\iterm_1 \le \iterm_2$, $c \vert  \iterm$ or $c\nmid \iterm$. 

\subsection{Enumeration of the variable orders}\label{qe-enum}

Suppose $n-1$ fresh variables are introduced in the normalization procedure, rename the original variable $\ivarx$ and the $n-1$ introduced variables as $\ivarx_i,1\le i\le n$.
Let the output of the normalization procedure be $\exists \vec{\ivarx}.\ \varphi'$ with $\vec{\ivarx} = (\ivarx_1,\ldots, \ivarx_n)$. 
We then enumerate all the linear orders of $\{\ivarx_1,\ldots, \ivarx_n\}$. Each linear order can be represented by a permutation $\sigma \in \mathcal{S}_n$ (where $\mathcal{S}_n$ is the permutation group on $[n]$), with the intention that $\ivarx_{\sigma(n)} \ge \cdots \ge \ivarx_{\sigma(1)}$.

Assuming a linear order $\sigma \in \mathcal{S}_n$ of $\{\ivarx_1,\ldots, \ivarx_n\}$, we then consider $\varphi'_\sigma  = \exists \vec{\ivarx}.\ \varphi' \wedge \bigwedge \limits_{i \in [n-1]} \ivarx_{\sigma(i)} \le \ivarx_{\sigma(i+1)}$ and eliminate the quantifiers $\exists \ivarx_{\sigma(n)}$, $\ldots$, $\exists \ivarx_{\sigma(1)}$,  one by one and from $\ivarx_{\sigma(n)}$ to $\ivarx_{\sigma(1)}$. To eliminate the $i$-th quantifer $\exists \ivarx_{\sigma(i)}$, we first eliminate all its exponential occurrences, i.e. $10^{\ivarx_{\sigma(i)}}$, using the fact that $\ivarx_{\sigma(i)}$ is the largest among the remaining quantified variables (see Lemma~\ref{lem:exp-ineq}). Then linear terms of $\ivarx_{\sigma(i)}$ are eliminated further by applying Cooper's algorithm so that the quantifier $\exists \ivarx_{\sigma(i)}$ can be removed. Let $\varphi''_\sigma$ denote the resulting formula.


Finally, $\exists \vec{\ivarx}.\ \varphi'$ is transformed into the quantifier-free formula $\bigvee \limits_{\sigma \in \mathcal{S}_n} \varphi''_{\sigma}$. 


\hide{
In the sequel, assuming a linear order $\sigma \in \mathcal{S}_n$, for $i \in [n]$, let $\exists \ivarx_{\sigma(1)} \ldots \exists \ivarx_{\sigma(i)}.\ \varphi''_{\sigma,i}$ be the formula obtained from $\varphi'_\sigma$ by eliminating the quantifiers $\exists \ivarx_{\sigma(n)}$, $\ldots$, $\exists \ivarx_{\sigma(i+1)}$, we show how to eliminate the exponential occurrences of $\ivarx_{\sigma(i)}$ in $\exists \ivarx_{\sigma(1)} \ldots \exists \ivarx_{\sigma(i)}.\ \varphi''_{\sigma,i}$. We would like to remark that the linear occurrences of $\ivarx_{\sigma(i)}$ should be eliminated further so that the quantifier $\exists \ivarx_{\sigma(i)}$ can be eliminated. The elimination of linear occurrences of $\ivarx_{\sigma(i)}$ is essentially the quantifier elimination algorithm of {\pa}.

Note that the order $\ivarx_{\sigma(i)} \ge \ldots \ge \ivarx_{\sigma(1)}$ guarantees the maximality of $\ivarx_{\sigma(i)}$ among $\ivarx_{\sigma(i)}, \ldots, \ivarx_{\sigma(1)}$, which is essential for the elimination of $10^{\ivarx_{\sigma(i)}}$ from $\varphi''_{\sigma,i}$ (see Lemma~\ref{lem:exp-ineq}).}

\subsection{Elimination of exponential occurrences of variables}\label{qe-elim-exp}

Let $i \in [n]$ and $\exists \ivarx_{\sigma(1)} \ldots \exists \ivarx_{\sigma(i)}.\ \varphi''_{\sigma,i}(\ivarx_{\sigma(i)}, \ldots, \ivarx_{\sigma(1)}, \vec{\ivary})$ be the formula obtained from $\varphi'_\sigma$ by eliminating the quantifiers $\exists \ivarx_{\sigma(n)}$, $\ldots$, $\exists \ivarx_{\sigma(i+1)}$. We show how to eliminate the exponential occurrences of $\ivarx_{\sigma(i)}$ in $\varphi''_{\sigma,i}$. The elimination is \emph{local} in the sense that it is applied to the atomic formulas independently. 

Recall that after normalization, the atomic formulas are of the form $\iterm_1 \le \iterm_2$, $c \vert  \iterm$ or $c \nmid \iterm$. Therefore, we can assume that the atomic formulas in $\varphi''_{\sigma,i}$ are  of the form 
%
$a_i 10^{\ivarx_{\sigma(i)}}+\sum_{j=1}^{i-1} a_j 10^{\ivarx_{\sigma(j)}} + \sum_{k=1}^{i}b_k \ivarx_{\sigma(k)} \le \iterm(\vec{\ivary})$
or  
$c \ \big  \vert  \ \big(a_i 10^{\ivarx_{\sigma(i)}}+\sum_{j=1}^{i-1} a_j 10^{\ivarx_{\sigma(j)}} + \sum_{k=1}^{i}b_k \ivarx_{\sigma(k)} + \iterm(\vec{\ivary})\big)$
(or $\nmid$), where $\iterm(\vec{\ivary})$ collects all terms in the formula without $\ivarx_i$.

\subsubsection{Inequality atoms}
In the following, we illustrate how to eliminate the exponential occurrences of $\ivarx_{\sigma(i)}$ for inequality atomic formulas of the form 

\vspace*{-2mm}

\begin{equation}\label{tau}
    \begin{aligned}
        &\tau(\ivarx_{\sigma(i)}, \ldots, \ivarx_{\sigma(1)}, \vec{\ivary}) \Def  \\
        &a_i 10^{\ivarx_{\sigma(i)}}+\sum_{j=1}^{i-1} a_j 10^{\ivarx_{\sigma(j)}} + \sum_{k=1}^{i}b_k \ivarx_{\sigma(k)} \le \iterm(\vec{\ivary})
    \end{aligned}            
\end{equation}
where $a_i\neq 0$, $\iterm(\vec \ivary)$ is the sum of all other terms without $\ivarx_{\sigma(i)}$.

The elimination of the exponential occurrences of $\ivarx_{\sigma(i)}$ in $\tau(\ivarx_{\sigma(i)}, \ldots, \ivarx_{\sigma(1)}, \vec{\ivary})$ relies on the following lemma. 
Intuitively, 
The lemma states that when the left-hand-side of the inequality is dominated by the $a_i 10^{\ivarx_{\sigma(i)}}$ term, if $\ivarx_{\sigma(i)}\le \alpha(\vec{\ivary})-1$ or $\ivarx_{\sigma(i)} \ge \alpha(\vec{\ivary})+2$, it can be determined directly whether the formula is true or false.

\begin{lemma} \label{lem:exp-ineq}
Let $\tau(\ivarx_{\sigma(i)}, \ldots, \ivarx_{\sigma(1)}, \vec{\ivary})$ be of form (\ref{tau}) 
with $a_i \neq 0$. Let $A\Def 1+\sum_{j=1}^{i-1}\vert a_j\vert$, 
$B \Def 1+\sum_{j=1}^{i}\vert b_j\vert$, 
$\delta\Def  \ell_{10}(A)+2$,
and $\gamma \Def 2\ell_{10}(B)+3$. 
When $\ivarx_{\sigma(i)} \ge \gamma$ and  $\ivarx_{\sigma(i)} \ge \ivarx_{\sigma(i-1)} +\delta$ are all satisfied,
\begin{itemize}
    \item for $a_i > 0$, let $\alpha(\vec{\ivary}) \Def \ell_{10}(\iterm(\ivary))- \ell_{10}(a_i)$, then 
    \begin{itemize}
        \item if $\ivarx_{\sigma(i)} \le \alpha(\vec{\ivary})  -1$, then $\tau(\ivarx_{\sigma(i)}, \ldots, \ivarx_{\sigma(1)}, \vec{\ivary})$ holds,
        \item if $\ivarx_{\sigma(i)} \ge \alpha(\vec{\ivary})  +2$, then $\tau(\ivarx_{\sigma(i)}, \ldots, \ivarx_{\sigma(1)}, \vec{\ivary})$ \textbf{does not} hold.
    \end{itemize}
    \item for $a_i < 0$, let $\alpha(\vec{\ivary})  \Def \ell_{10}(-\iterm(\ivary))- \ell_{10}(-a_i)$, then 
    \begin{itemize}
        \item if $\ivarx_{\sigma(i)} \le \alpha(\vec{\ivary})  -1$, then $\tau(\ivarx_{\sigma(i)}, \ldots, \ivarx_{\sigma(1)}, \vec{\ivary})$ \textbf{does not} hold,
        \item if $\ivarx_{\sigma(i)} \ge \alpha(\vec{\ivary})  +2$, then $\tau(\ivarx_{\sigma(i)}, \ldots, \ivarx_{\sigma(1)}, \vec{\ivary})$ holds.
    \end{itemize}
\end{itemize}
\end{lemma}

We need the following two propositions for the proof of Lemma~\ref{lem:exp-ineq}. 
Proposition~\ref{prop:linear} shows that when $\ivarx$ is large enough, the linear term $n \ivarx$ can always be bound by the exponential term $10^\ivarx$. Proposition~\ref{prop:case} can be seen as a special case of Lemma~\ref{lem:exp-ineq} where the left-hand-side of $\tau(\ivarx_{\sigma(i)}, \ldots, \ivarx_{\sigma(1)}, \vec{\ivary})$ has only one term $a_i 10^{\ivarx_{\sigma(i)}}$. 

\vspace*{-3mm}

\begin{proposition} \label{prop:linear}
Given $n\ge 1$, if $\ivarx \ge 2\ell_{10}(n)+1$, then 
$n \ivarx  \le 10^\ivarx$ holds.
\end{proposition}

\vspace*{-3mm}

\begin{proof}
When $1\le n\le 9$, we have $n\ivarx \le 10\ivarx \le 10^\ivarx$ for all $\ivarx \in\mathbb N$.

For $n\ge 10$, that is, $\ell_{10}(n)\ge 1$, then 
$$\begin{aligned}
10^\ivarx&\ge 10^{\ell_{10}(n)+1}10^{\ivarx-\ell_{10}(n)-1}\\
    &\ge 10^{\ell_{10}(n)+1}10(\ivarx-\ell_{10}(n)-1)\\
    &\ge 10^{\ell_{10}(n)+1}\ivarx\\
    &\ge n\ivarx
\end{aligned}$$
\end{proof}


\begin{proposition} \label{prop:case}
    Given $a>0$, 
    if $\ivarx\le \ell_{10}(\ivary)-\ell_{10}(a)-1$, then $(a+\frac{1}{2})\cdot 10^\ivarx\le\ivary$; 
    if $\ivarx\ge \ell_{10}(\ivary)-\ell_{10}(a)+2$, then $(a-\frac{1}{2})\cdot 10^\ivarx\ge\ivary$.
\end{proposition}

\begin{proof}

Suppose that $\ivarx\le \ell_{10}(\ivary)-\ell_{10}(a)-1$,
we have 
$$
(a+\frac{1}{2})\cdot 10^\ivarx \le (a+\frac{1}{2})\cdot \frac{\lambda_{10}(\ivary)}{10\lambda_{10}(a)}=\frac{a+\frac{1}{2}}{10\lambda_{10} (a)}\lambda_{10}(\ivary)\le\lambda_{10}(\ivary) \le \ivary. 
$$

Now, suppose that $\ivarx\ge \ell_{10}(\ivary)-\ell_{10}(a)+2$,
$$
(a-\frac{1}{2})\cdot 10^\ivarx \ge (a-\frac{1}{2})\frac{100\lambda_{10}(\ivary)}{\lambda_{10}(a)}=\frac{10(a-\frac{1}{2})}{\lambda_{10}(a)}10\lambda_{10}(\ivary)\ge 10\lambda_{10}(\ivary) \ge \ivary
$$
\end{proof}



\begin{proof}[Proof of Lemma~\ref{lem:exp-ineq}]
We only prove for the case $a_i > 0$, the $a_i < 0$ case is symmetric. The proof is twofold.

1) First, we prove that if $\ivarx_{\sigma(i)}\ge \gamma$ and $\ivarx_{\sigma(i)} \ge \ivarx_{\sigma(i-1)} + \delta$, then the left-hand-side of $\tau(\ivarx_{\sigma(i)}, \ldots, \ivarx_{\sigma(1)}, \vec{\ivary})$ is dominated by $a_i 10^{\ivarx_{\sigma(i)}}$, i.e. ,
\begin{equation} \label{eq:bound}
(a_i-\frac{1}{2})10^{\ivarx_{\sigma(i)}}
< a_i 10^{\ivarx_{\sigma(i)}}+\sum_{j=1}^{i-1} a_j 10^{\ivarx_{\sigma(j)}} + \sum_{k=1}^{i}b_k \ivarx_{\sigma(k)}
< (a_i+\frac{1}{2})10^{\ivarx_{\sigma(i)}}
\end{equation}

To obtain the formula above, we need to bound the absolute values $\vert\sum_{j=1}^{i-1}a_j10^{\ivarx_{\sigma(j)}}\vert$ and  $\vert\sum_{k=1}^i b_k\ivarx_{\sigma(k)}\vert$ by some fraction of $10^{\ivarx_{\sigma(i)}}$ respectively. Here we choose the bound to be $\frac{1}{10}10^{\ivarx_{\sigma(i)}}$.

Using $\ivarx_{\sigma(i)}\ge \ivarx_{\sigma(i-1)}+\delta$ and the linear order $\ivarx_{\sigma(i)}\ge \ivarx_{\sigma(i-1)} \ge \cdots \ge \ivarx_{\sigma(1)}$,
we can prove that 
\begin{equation}\label{eq:appr-exp}
    \begin{aligned}
        \vert \sum_{j=1}^{i-1}a_j10^{\ivarx_{\sigma(j)}} \vert &\le  \sum_{j=1}^{i-1} \vert a_j\vert 10^{\ivarx_{\sigma(j)}}\\
        &< A\cdot 10^{\ivarx_{\sigma(i)}-\delta}\\
        &\le \frac{A}{100\lambda_{10}(A)} 10^{\ivarx_{\sigma(i)}}\\
        &< \frac{1}{10} 10^{\ivarx_{\sigma(i)}} && \text{(by $\ivarx < 10 \lambda_{10}(\ivarx)$)}.
    \end{aligned}
\end{equation}

Similarly, using $\ivarx_{\sigma(i)}\ge \gamma$, we have
\begin{equation}\label{eq:appr-linear}
    \begin{aligned}
        \vert \sum_{k=1}^i b_k\ivarx_{\sigma(k)} \vert &\le \sum_{k=1}^i \vert b_k \vert \ivarx_{\sigma(k)} \\
        &< B \cdot \ivarx_{\sigma(i)}\\
        &=\frac{1}{10}\cdot 10B \cdot \ivarx_{\sigma(i)}\\
        &\le \frac{1}{10} 10^{\ivarx_{\sigma(i)}} &&\text{(by Proposition~\ref{prop:linear}, set $n=10B$)}.     
    \end{aligned}
\end{equation}

Combining formula (\ref{eq:appr-exp}) and (\ref{eq:appr-linear}), we have 
$\vert \sum_{j=1}^{i-1}a_j10^{\ivarx_{\sigma(j)}}
+ \sum_{k=1}^i b_k\ivarx_{\sigma(k)}\vert 
< \frac{3}{10} 10^{\ivarx_{\sigma(i)}}< \frac{1}{2} 10^{\ivarx_{\sigma(i)}}$, then formula (\ref{eq:bound}) can be easily deduced.  

2) Then, we give the sufficient conditions for $\tau(\ivarx_{\sigma(i)}, \ldots, \ivarx_{\sigma(1)}, \vec{\ivary})$ to hold or not by comparing $\iterm(\vec\ivary)$ with $(a_i \pm\frac{1}{2}) 10^{\ivarx_{\sigma(i)}}$. 
Note that when $\iterm(\vec\ivary)\le 0$, $\tau(\ivarx_{\sigma(i)}, \ldots, \ivarx_{\sigma(1)}, \vec{\ivary})$ does not hold since $0< (a_i-\frac{1}{2})10^{\ivarx_{\sigma(i)}}$, so we only need to consider the $\iterm(\vec\ivary)> 0$ case.

Utilizing Proposition~\ref{prop:case}, we obtain that 

\begin{equation}
    \begin{aligned}
        \ivarx_{\sigma(i)} \le \alpha(\vec\ivary)-1
        &\implies (a_i+\frac{1}{2}) \cdot 10^{\ivarx_{\sigma(i)}}\le \iterm(\vec\ivary)\\
       &\implies \tau(\ivarx_{\sigma(i)}, \ldots, \ivarx_{\sigma(1)}, \vec{\ivary}) \textrm{ is satisfied }
    \end{aligned}
\end{equation}
and
\begin{equation}
    \begin{aligned}
        \ivarx_{\sigma(i)} \ge \alpha(\vec\ivary)+2
        &\implies (a_i-\frac{1}{2})\cdot 10^{\ivarx_{\sigma(i)}}\ge\iterm(\vec\ivary)\\ 
        &\implies \tau(\ivarx_{\sigma(i)}, \ldots, \ivarx_{\sigma(1)}, \vec{\ivary}) \textrm{ is unsatisfied }
    \end{aligned}
\end{equation}

Thus the lemma is proved.
\end{proof}



%%%%%%%%%%%%%%%%%%%%%%%%%%%%%%%%%%%%%%%%%%
\hide{
Let $A\Def \sum_{j=1}^{i-1}\vert a_j\vert $, $\gamma \Def 2(\ell_{10}(\sum_{j=1}^{i}\vert b_j\vert )+3)$, and $\delta\Def  \ell_{10}(A)+3$. 

Suppose $\ivarx_{\sigma(i)} \ge \gamma$ and $\ivarx_{\sigma(i)} \ge \ivarx_{\sigma(i-1)} + \delta$. Moreover, let $\alpha(\vec{\ivary}) \Def  \ell_{10}(\iterm(\ivary))- \ell_{10}(a_i)$.

Note that
 \begin{equation} 
   A10^{-\delta} =A 10^{-\ell_{10}(A)-3} = \frac{A}{1000\lambda_{10}(A)} \le \frac{1}{100}.   
 \end{equation}
 
From $\ivarx_{\sigma(i)} \ge \ivarx_{\sigma(i-1)} + \delta$ and $\ivarx_{\sigma(i-1)} \ge \ldots \ge \ivarx_{\sigma(1)}$, 
we know   
$$-A 10^{\ivarx_{\sigma(i)} - \delta} \le  \sum_{j=1}^{i-1} a_j 10^{\ivarx_{\sigma(j)}} \le A 10^{\ivarx_{\sigma(i)} - \delta}$$
and
$$-(\sum_{j=1}^{i}\vert b_j\vert ) \ivarx_{\sigma(i)} \le \sum_{k=1}^{i} b_k \ivarx_{\sigma(k)} \le (\sum_{j=1}^{i}\vert b_j\vert ) \ivarx_{\sigma(i)}.$$ 

Moreover, let $n = 100\sum_{j=1}^{i}\vert b_j\vert $, $m = 1$, and $p = \ivarx_{\sigma(i)}$, then 
$n \ge m \ge 1$. From $2(\ell_{10}(n)- \ell_{10}(m)+1) = 2 (\ell_{10}(100 \sum_{j=1}^{i}\vert b_j\vert ) + 1) = 2 (\ell_{10}(\sum_{j=1}^{i}\vert b_j\vert ) + 3) = \gamma$, we deduce that $p = \ivarx_{\sigma(i)} \ge \gamma = 2(\ell_{10}(n)- \ell_{10}(m)+1)$.
Then according to Proposition~\ref{prop:linear}, 
$$100(\sum_{j=1}^{i}\vert b_j\vert )\ivarx_{\sigma(i)} = np \le m 10^p = 10^{\ivarx_{\sigma(i)}}.$$
Thus 
$(\sum_{j=1}^{i}\vert b_j\vert )\ivarx_{\sigma(i)}  \le \frac{1}{100} 10^{\ivarx_{\sigma(i)}}.$

If $\ivarx_{\sigma(i)} \ge   \alpha(\vec{\ivary}) + 2$, then 

$
\begin{array}{l}
a_i 10^{\ivarx_{\sigma(i)}}+\sum_{j=1}^{i-1} a_j 10^{\ivarx_{\sigma(j)}} + \sum_{k=1}^{i}b_k \ivarx_{\sigma(k)} \ge \\
a_i 10^{\ivarx_{\sigma(i)}} - A 10^{\ivarx_{\sigma(i)} - \delta} - (\sum_{j=1}^{i}\vert b_j\vert ) \ivarx_{\sigma(i)}  \ge \\
a_i 10^{\ivarx_{\sigma(i)}} - \frac{1}{100}10^{\ivarx_{\sigma(i)}} - \frac{1}{100} 10^{\ivarx_{\sigma(i)}} = \\
(a_i - \frac{1}{50}) 10^{\ivarx_{\sigma(i)}} \ge (a_i - \frac{1}{50}) 10^{ \alpha(\vec{\ivary}) + 2} = \\
(a_i - \frac{1}{50}) 10^{  \ell_{10}(\iterm(\ivary))- \ell_{10}(a_i) + 2} = \\
\frac{10(a_i - \frac{1}{50})}{10^{\ell_{10}(a_i)} } 10^{ \ell_{10}(\iterm(\ivary))+1} \ge \frac{10(a_i - \frac{1}{50})} {a_i} 10^{ \ell_{10}(\iterm(\ivary))+1} \ge \\
(10 - \frac{1}{5a_i}) 10^{ \ell_{10}(\iterm(\ivary))+1} > 10^{ \ell_{10}(\iterm(\ivary))+1} \ge \iterm(\ivary).
\end{array}
$


Therefore, in this case, $\tau(\ivarx_{\sigma(i)}, \ldots, \ivarx_{\sigma(1)}, \vec{\ivary})$ \emph{does not} hold.

If $\ivarx_{\sigma(i)} \le   \alpha(\vec{\ivary}) - 1$, then 

$
\begin{array}{l}
a_i 10^{\ivarx_{\sigma(i)}}+\sum_{j=1}^{i-1} a_j 10^{\ivarx_{\sigma(j)}} + \sum_{k=1}^{i}b_k \ivarx_{\sigma(k)} \le \\
a_i 10^{\ivarx_{\sigma(i)}} + A 10^{\ivarx_{\sigma(i)} - \delta} + (\sum_{j=1}^{i}\vert b_j\vert ) \ivarx_{\sigma(i)} \le \\
a_i 10^{\ivarx_{\sigma(i)}} + \frac{A} {10^\delta}10^{\ivarx_{\sigma(i)}} + \frac{1}{100} 10^{\ivarx_{\sigma(i)}} \le \\
(a_i + \frac{1}{100} + \frac{1}{100})10^{\ivarx_{\sigma(i)}} = (a_i + \frac{1}{50}) 10^{\ivarx_{\sigma(i)}} \le \\
(a_i + \frac{1}{50}) 10^{\alpha(\vec{\ivary}) - 1} = (a_i + \frac{1}{50}) 10^{\ell_{10}(\iterm(\ivary))- \ell_{10}(a_i) - 1} = \\
\frac{a_i + \frac{1}{50}} {10^{\ell_{10}(a_i) + 1}} 10^{\ell_{10}(\iterm(\ivary))} = \frac{a_i + \frac{1}{50}} {10 \lambda_{10}(a_i)} 10^{\ell_{10}(\iterm(\ivary))} \le \\
\frac{a_i + \frac{1}{50}} {a_i+1} 10^{\ell_{10}(\iterm(\ivary))} \le 10^{\ell_{10}(\iterm(\ivary))} \le \iterm(\ivary).
\end{array}
$
Therefore,  in this case, $\tau(\ivarx_{\sigma(i)}, \ldots, \ivarx_{\sigma(1)}, \vec{\ivary})$ holds.
}
%%%%%%%%%%%%%%%%%%%%%%%%%%%%%%%%%%%%%%%%

For $a_i>0$, by lemma~\ref{lem:exp-ineq},
$\tau(\ivarx_{\sigma(i)}, \ldots, \ivarx_{\sigma(1)}, \vec{\ivary})$ (for readability, denoted by $\tau$ below) is equivalent to following formula without exponential occurrences of $\ivarx_{\sigma(i)}$:
%&&(\ivarx_{\sigma(i)}<\gamma \text{ or } \ivarx_{\sigma(i)}\le \ivarx_{\sigma(i-1)}+\delta)
\begin{equation}\label{lemma-example}\begin{aligned}
    &\bigvee_{p=0}^{\gamma-1} (\ivarx_{\sigma(i)}=p \wedge \tau[p/\ivarx_{\sigma(i)}])\\
    \vee &\bigvee_{p=0}^{\delta-1} (\ivarx_{\sigma(i)}=\ivarx_{\sigma(i-1)}+p\wedge \tau[\ivarx_{\sigma(i-1)}+p/\ivarx_{\sigma(i)}])\\
    \vee & \bigg( \ivarx_{\sigma(i)}\ge \gamma \wedge \ivarx_{\sigma(i)} \ge \ivarx_{\sigma(i-1)}+\delta \wedge \ivarx_{\sigma(i)}\le \alpha(\vec\ivary)-1\bigg)\\
    \vee & \bigvee_{p=0,1} \bigg( \ivarx_{\sigma(i)}\ge \gamma \wedge \ivarx_{\sigma(i)} \ge \ivarx_{\sigma(i-1)}+\delta \wedge \ivarx_{\sigma(i)} = \alpha(\vec\ivary)+p \wedge \tau[\alpha(\vec\ivary)+p/\ivarx_{\sigma(i)}]\bigg).
\end{aligned}\end{equation}
The elimination of the exponential occurrences of $\ivarx_{\sigma(i)}$ for the case $a_i < 0$ is similar. 

We would like to make a few remarks about Lemma~\ref{lem:exp-ineq}:
(1) When $i=1$, we can just ignore the condition $\ivarx_{\sigma(i)}\ge \ivarx_{\sigma(i-1)}+\delta$.
(2) The constant $1$ in the definition of $A,B$ is to ensure that $\ell_{10}(A),\ell_{10}(B)$ are well-defined. 
(3) The lemma still holds when the base of exponential function is changed to any natural number $n\ge 2$ (constants in $\delta,\gamma$ should be changed accordingly).


%%%%%%%%%%%%%%%%%%%%%%%%%%%%%%%%%%%%%%%%%
\hide{
\[
\small
\begin{array}{l}
\bigvee \limits_{p=0}^{\gamma-1} a_i 10^{p}+\sum_{j=1}^{i-1} a_j 10^{\ivarx_{\sigma(j)}} + b_i p + \sum_{k=1}^{i-1}b_k \ivarx_{\sigma(k)} \le \iterm(\vec{\ivary}) \\
\bigvee \big(\ivarx_{\sigma(i)} \ge \gamma \wedge \ivarx_{\sigma(i)} \le \alpha(\vec{\ivary})  -1  \wedge \ivarx_{\sigma(i)} \ge \ivarx_{\sigma(i-1)} +\delta \big)\\
%
\bigvee \bigvee \limits_{p=0}^{\delta-1} 
\left(
\begin{array}{l}
\ivarx_{\sigma(i)} \ge \gamma \wedge \ivarx_{\sigma(i)} \le \alpha(\vec{\ivary})  -1 \ \wedge \\
 \ivarx_{\sigma(i)} = \ivarx_{\sigma(i-1)} +p\ \wedge\\
 \tau(\ivarx_{\sigma(i)}, \ldots, \ivarx_{\sigma(1)}, \vec{\ivary})[\ivarx_{\sigma(i-1)} + p /\ivarx_{\sigma(i)}] 
\end{array}
\right)\\
\bigvee 
\left(
\begin{array}{l}
\ivarx_{\sigma(i)} \ge \gamma \wedge \ivarx_{\sigma(i)} = \alpha(\vec{\ivary})\ \wedge \\
\tau(\ivarx_{\sigma(i)}, \ldots, \ivarx_{\sigma(1)}, \vec{\ivary})[\alpha(\vec{\ivary})/\ivarx_{\sigma(i)}]
\end{array}
\right)  \\
\bigvee 
\left(
\begin{array}{l}
\ivarx_{\sigma(i)} \ge \gamma \wedge \ivarx_{\sigma(i)} = \alpha(\vec{\ivary})+1\ \wedge \\
\tau(\ivarx_{\sigma(i)}, \ldots, \ivarx_{\sigma(1)}, \vec{\ivary})[\alpha(\vec{\ivary})+1/\ivarx_{\sigma(i)}]
\end{array}
\right)  \\
\bigvee \bigvee \limits_{p=0}^{\delta-1} 
\left(
\begin{array}{l}
\ivarx_{\sigma(i)} \ge \gamma \wedge \ivarx_{\sigma(i)} \ge \alpha(\vec{\ivary})+2\ \wedge\\
 \ivarx_{\sigma(i)} = \ivarx_{\sigma(i-1)} +p\ \wedge \\
 %\vspace{-1mm}
 \tau(\ivarx_{\sigma(i)}, \ldots, \ivarx_{\sigma(1)}, \vec{\ivary})[\ivarx_{\sigma(i-1)}+ p /\ivarx_{\sigma(i)}]
\end{array}
\right),
\end{array}
\]
}
%%%%%%%%%%%%%%%%%%%%%%%%%%%%%%%%%

\subsubsection{Divisibility atoms}

For divisibility atoms, we utilize its periodic property to enumerate $\ivarx_{\sigma(i)}$ within a finite range.
Consider a divisiblilty atomic formula 
\begin{equation}
    \begin{aligned}
        &\tau(\ivarx_{\sigma(i)}, \ldots, \ivarx_{\sigma(1)}, \vec{\ivary}) \Def  \\
        &d\ \big \vert \ \big(a_i 10^{\ivarx_{\sigma(i)}} + \sum_{j=1}^{i-1} a_j 10^{\ivarx_{\sigma(j)}} + \sum_{k=1}^{i} b_k \ivarx_{\sigma(k)} 
        + \iterm(\vec{\ivary}) \big)               
    \end{aligned}
\end{equation}
with $a_i \neq 0, d\ge 1$. The indivisibility case can be treated analogously.

Let $d = 2^{r_1} 5^{r_2}  d_0$ such that $d_0$ is divisible by neither $2$ nor $5$. Moreover, let $r = \max(r_1, r_2)$. Then $d \vert  (10^rd_0)$. 

Since $10$ and $d_0$ are relatively prime, according to Euler's theorem (cf. \cite{HW80}), $10^{\phi(d_0)} \equiv 1 \bmod d_0$, where $\phi$ is the Euler function. Suppose $10^{\phi(d_0)} = kd_0 + 1$ for some $k \in \Nat$. 
Then for every $n \in \Nat$ with $n \ge r$, 
$$\begin{aligned}
    10^{n + \phi(d_0)} 
    &\equiv 10^{n-r} 10^r (kd_0 + 1) \bmod d\\
    &\equiv 10^{n-r} (k 10^rd_0 + 10^r) \bmod d\\
    &\equiv 10^{n-r} (0+10^r) \bmod d\\
    &\equiv 10^n \bmod d.
\end{aligned}$$

\hide{
$$
\begin{array}{l}
10^{n + \phi(d_0)} \bmod d =10^{n-r} 10^r (kd_0 + 1) \bmod d = \\
10^{n-r} (k 10^rd_0 + 10^r) \bmod d = \\
10^{n-r} (0+10^r) \bmod d = 10^n \bmod d.
\end{array}
$$
}

Then $\tau(\ivarx_{\sigma(i)}, \ldots, \ivarx_{\sigma(1)}, \vec{\ivary})$ is equivalent to the following formula

\begin{equation}\label{lem-div}
\begin{array}{l}
\bigvee \limits_{p=0}^{r-1} \tau(\ivarx_{\sigma(i)}, \ldots, \ivarx_{\sigma(1)}, \vec{\ivary})[p/\ivarx_{\sigma(i)}]\\
\vee
\left(
\begin{array}{l}
\ivarx_{\sigma(i)} \ge r\ \wedge \\
\bigvee \limits_{p = 0}^{\phi(d_0)-1} 
\left(
\begin{array}{l}
\phi(d_0)\ \big \vert \ (\ivarx_{\sigma(i)} - r - p)\ \wedge \\
d\ \big \vert  
\left(
\begin{array}{l}
a_i 10^{r+p} + \sum_{j=1}^{i-1} a_j 10^{\ivarx_{\sigma(j)}} + \\
\sum_{k=1}^{i} b_k \ivarx_{\sigma(k)} + \iterm(\vec{\ivary})
\end{array}
\right) 
\end{array}
\right)
\end{array}
\right),
\end{array}
\end{equation}
where the exponential occurrences of $\ivarx_{\sigma(i)}$ disappear. When $\ivarx_{\sigma(i)}>r$, we only need to enumerate $\ivarx_{\sigma(i)}$ in one period, i.e. from $r$ to $r+\phi(d_0)-1$.


For a special case $d_0 = 1$, since $\phi(1)=1$, (\ref{lem-div}) can be simplified to 
$$\begin{aligned}
    \bigvee_{p=0}^{r-1} \tau [p/\ivarx_{\sigma(i)}] \vee  \bigg(\ivarx_{\sigma(i)}\ge r \wedge d\ \big \vert \big(\sum_{j=1}^{i-1} a_j 10^{\ivarx_{\sigma(j)}} + \sum_{k=1}^{i} b_k \ivarx_{\sigma(k)} 
    + \iterm(\vec{\ivary}) \big)\bigg)
\end{aligned}.$$


%%%%%%%%%%%%%%%%%%%%%%%%%%%%%%%%%%%%%%%%%%%%%%

\hide{

then $10^r$ is divisible by $d = 2^{r_1}5^{r_2}$. Thus for every $n \ge r$, $d \ \vert \ 10^n$.  Therefore, in this case, $\tau(\ivarx_{\sigma(i)}, \ldots, \ivarx_{\sigma(1)}, \vec{\ivary})$ is equivalent to 
\[
\small
\begin{array}{l}
\bigvee \limits_{p = 0}^{r - 1} d\ \big \vert  \big(a_i 10^{p} + \sum_{j=1}^{i-1} a_j 10^{\ivarx_{\sigma(j)}} + b_kp + \sum_{k=1}^{i-1} b_k \ivarx_{\sigma(k)} 
+ \iterm(\vec{\ivary}) \big)\\
%
\vee \big(\ivarx_{\sigma(i)} \ge r \wedge d\ \big \vert  \big(\sum_{j=1}^{i-1} a_j 10^{\ivarx_{\sigma(j)}} + \sum_{k=1}^{i} b_k \ivarx_{\sigma(k)} 
+ \iterm(\vec{\ivary}) \big)\big),
\end{array}
\]
where the exponential occurrences of $\ivarx_{\sigma(i)}$ disappear.

Next, let us assume $d_0 > 1$. 
}

\hide{
\[
\begin{array}{l}
\bigvee \limits_{p=0}^{r-1} \tau(\ivarx_{\sigma(i)}, \ldots, \ivarx_{\sigma(1)}, \vec{\ivary})[p/\ivarx_{\sigma(i)}]\ \vee \\
\left(
\begin{array}{l}
\ivarx_{\sigma(i)} \ge r\ \wedge \\
\bigvee \limits_{q = 0}^{\phi(d_0)-1} 
\left(
\begin{array}{l}
\phi(d_0)\ \big \vert \ (\ivarx_{\sigma(i)} - r - q)\ \wedge \\
d\ \big \vert  
\left(
\begin{array}{l}
a_i 10^{r+q} + \sum_{j=1}^{i-1} a_j 10^{\ivarx_{\sigma(j)}} + \\
\sum_{k=1}^{i} b_k \ivarx_{\sigma(k)} + \iterm(\vec{\ivary})
\end{array}
\right) 
\end{array}
\right)
\end{array}
\right),
\end{array}
\]
where the exponential occurrences of $\ivarx_{\sigma(i)}$ disappear.
}

\hide{
\begin{equation}\label{lem-div}
    \begin{aligned}
        &\bigvee_{p=0}^{r-1} \tau[p/\ivarx_{\sigma(i)}]\\ 
        \vee &\left( x_{\sigma(i)}\ge r \wedge 
            \bigvee_{p=0}^{\phi(d_0)-1} \left(\tau[r+p/\ivarx_{\sigma(i)}]\wedge \phi(d_0)\ \big \vert\ (\ivarx_{\sigma(i)}-r-p) \right)
            \right).
    \end{aligned}
\end{equation}
}
%%%%%%%%%%%%%%%%%%%%%%%%%%%%%%%%%%%%%%%%%%%%%%%%%%%%%%