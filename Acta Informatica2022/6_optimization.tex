%!TEX root = paper.tex

In the last section we showed the complexity of Point's procedure over existential {\paexp} formulas to be 3-EXPSPACE, which  is quite expensive and a faithful implementation would not scale\footnote{We did implement Point's algorithm and discovered that the implementation could only solve formulas of very small size.}.
Note that this high complexity holds even for quantifier-free $\paexp$ formulas: For a quantifier-free formula $\varphi$, we solve its satisfiability problem by adding the existential quantifiers for all the variables occurring in $\varphi$, then eliminate the quantifiers one by one, resulting into $\ltrue$ or $\lfalse$ in the end. The original formula $\varphi$ is satisfiable if $\ltrue$ is obtained in the end.

\hide{
In Section~\ref{sec-dec}, we illustrated the main idea of the decision procedure for  {\paexp}. In last section, we show that the decision procedure has a high complexity in the sense that the elimination of the exponential occurrences of each variable incurs an exponential blow-up, similarly for linear occurrences. Note that this high complexity holds even for quantifier-free formulas containing exponential terms: For a quantifier-free formula $\varphi$ containing exponential terms, we solve its satisfiability problem by adding the existential quantifiers for all the variables occurring in $\varphi$, then eliminate the quantifiers one by one, resulting into $\ltrue$ or $\lfalse$ in the end. The original formula $\varphi$ is satisfiable if $\ltrue$ is obtained in the end.
}

In this section, we focus on quantifier-free {\paexp} formulas (or existential {\paexp} formulas since they are satisfiability-equivalent), and present various optimizations of the quantifier elimination procedure for {\paexp}, aiming at an efficient implementation. The focus on quantifier-free {\paexp} formulas is motivated by the following two facts: 1) the flattening of {$\strint$} constraints results into such formulas, 2) these formulas are already challenging for state-of-the-art SMT solvers (with exponential functions defined as recursive functions). 

Let $\varphi$ be a quantifier-free {\paexp} formula in the remainder of this section. Moreover, we assume that $\varphi$ is normalized since the optimizations presented in the sequel are for normalized formulas. Furthermore, for technical convenience, we assume that all the inequality atomic formulas are of the form 
%
$\sum_{j=1}^{n} a_j 10^{\ivarx_{j}} + \sum_{k=1}^{n}b_k \ivarx_{k} \le c$, 
%
where $c$ is an integer constant. (Implicitly, we assume that there are no free variables and all the variables are existentially quantified.)

In the sequel, we will explain two major optimizations in \ref{para: opt reduce} and \ref{para: opt under appro}. Additional optimizations are listed in \ref{para: opts}.

\subsection{Reduce the number of enumerated variable orders by over approximation}\label{para: opt reduce}

Recall that in Point's procedure for {\paexp}, after the normalization, the variable orders are enumerated and for each order, the exponential and linear occurrences of variables are eliminated. Since the quantifier elimination is expensive and applied to each possible order of variables, if we could reduce the candidate variable orders in the very beginning, it would facilitate considerable speed-up for the decision procedure. 

Our main idea is to consider an over approximation of $\varphi$, which is a {\pa} formula $\varphi'$, and use $\varphi'$ to remove the infeasible candidate variable orders.

Note that all the exponential terms in $\varphi$ is of the form $10^{\ivarz}$ for some integer variable $\ivarz$. 
%
The over approximation is based on the observation that $10^n \ge 9 n + 1$ for every $n \in \Nat$. Then we obtain the over approximation $\varphi'$ from $\varphi$ by replacing each exponential term $10^{\ivarz}$ with a fresh variable $\ivarz'$ and add $\ivarz' \ge 9 \ivarz +1$ as a conjunct.

Then during the enumeration of the linear orders for the variables $\ivarx_1, \ldots, \ivarx_n$, we can quickly remove those infeasible candidates $\sigma$ such that $\varphi' \wedge \bigwedge \limits_{i \in [n-1]}\ivarx_{\sigma(i)} \le \ivarx_{\sigma(i+1)}$ is unsatisfiable. A special case is that if $\varphi'$ is unsatisfiable, then we can directly conclude that original formula $\varphi$ is unsatisfiable.

%Given any formula in $\paexp$, we first invoke \textbf{Normalization} to translate it into a simpler form. Then we  try to detect if there are obvious conflicts, for example, if the linear constraints combined are unsatisfiable, the whole formula is of course unsatisfiable.

%Suppose the normalized formula $\theta$ contains variables $\{x_i:i\in[n]\}$, so the exponential terms could be $e^{x_i}$. For each $e^{x_i}$, we subtitute $e^{x_i}$ by a fresh variable $y_i$ in $\theta$ and add the constraints $(e-1)x_i+1\le y_i$ to get $\theta'$. Since $(e-1)x_i+1\le e^{x_i}$ holds for all $e,x\in \Nat$, $\theta'$ is an over-approximation of $\theta$ and if $\theta'$ is unsatisfiable over positive reals, then $\theta$ is unsatisfiable over $\Nat$.

%\vspace*{-3mm}
\subsection{Avoid the elimination of linear occurrences of variables by under approximation}\label{para: opt under appro}

The decision procedure of {\paexp} in Section~\ref{sec-dec} requires the elimination of both exponential and linear occurrences of variables. Considering the fact that {\pa} formulas can be solved efficiently by the state-of-the-art solvers, e.g. CVC4 and Z3, one natural idea is to try to only eliminate the exponential occurrences, but not the linear occurrences, of variables, and obtain the {\pa} formulas in the end, which can then be solved by the state-of-the-art solvers.

Recall that Lemma~\ref{lem:exp-ineq} enables us to eliminate the exponential occurrences of $\ivarx_{\sigma(i)}$ from an atomic formula $\tau$ of the form
\hide{$$
\begin{array}{l}
\tau(\ivarx_{\sigma(i)}, \ldots, \ivarx_{\sigma(1)}) \Def  \\
a_i 10^{\ivarx_{\sigma(i)}}+\sum_{j=1}^{i-1} a_j 10^{\ivarx_{\sigma(j)}} + \sum_{k=1}^{i}b_k \ivarx_{\sigma(k)} \le c.
\end{array}
$$}
\begin{equation}
    \begin{aligned}
        &\tau(\ivarx_{\sigma(i)}, \ldots, \ivarx_{\sigma(1)}) \Def  \\
        &a_i 10^{\ivarx_{\sigma(i)}}+\sum_{j=1}^{i-1} a_j 10^{\ivarx_{\sigma(j)}} + \sum_{k=1}^{i}b_k \ivarx_{\sigma(k)} \le c.
    \end{aligned}            
\end{equation}


Actually, Lemma~\ref{lem:exp-ineq} does more in the sense that all occurrences of $\ivarx_{\sigma(i)}$, including the linear ones, are eliminated from the atomic formulas resulted from $\tau$, e.g. $\tau[\ivarx_{\sigma(i-1)} + p /\ivarx_{\sigma(i)}]$. Then we can continue eliminating the exponential occurrences of $\ivarx_{\sigma(i-1)}$ from $\tau[\ivarx_{\sigma(i-1)} + p /\ivarx_{\sigma(i)}]$, provided that the coefficient of $\ivarx_{\sigma(i-1)}$ therein is nonzero. Iterating this process would produce a {\pa} formula eventually.

Nevertheless, the condition of Lemma~\ref{lem:exp-ineq}, namely $a_i \neq 0$, undermines the aforementioned natural idea. If $a_i = 0$, but $b_i \neq 0$, 
%namely, 
%$$
%\begin{array}{l}
%\tau(\ivarx_{\sigma(i)}, \ldots, \ivarx_{\sigma(1)}) \Def  \\
%\hspace{4mm} 
%\sum_{j=1}^{i-1} a_j 10^{\ivarx_{\sigma(j)}} + \sum_{k=1}^{i}b_k \ivarx_{\sigma(k)} \le c,
%\end{array}
%$$
then we are unable to utilize Lemma~\ref{lem:exp-ineq} to eliminate the linear occurrences of $\ivarx_{\sigma(i)}$ from $\tau$. In this case, the quantifier elimination algorithm of {\pa} has to be applied to eliminate $\ivarx_{\sigma(i)}$, so that later on, we can eliminate the exponential occurrences of $\ivarx_{\sigma(i-1)}$, which requires that $\ivarx_{\sigma(i-1)}$ is the maximum variable in the left-hand side of the inequality.

To avoid applying the quantifier elimination algorithm of {\pa}, we consider the following under approximation of $\tau$, namely, we additionally assume that $\ivarx_{\sigma(i)} \le 10^{u}$ for some constant bound $u \in \Nat$. Then $\tau$ can be rewritten as 
%\begin{equation}\label{eqn-org}
%\vspace{-2mm}
\[\tau'(\ivarx_{\sigma(i)}, \ldots, \ivarx_{\sigma(1)}) \Def \sum_{j=1}^{i-1} a_j 10^{\ivarx_{\sigma(j)}} + \sum_{k=1}^{i-1}b_k \ivarx_{\sigma(k)} \le c - b_i  \ivarx_{\sigma(i)}.\]
%\vspace{-1mm}
%\end{equation}
Let us assume $a_{i-1} > 0$.
Define $c_1, c_2$ as follows: If $b_i > 0$, then $c_1 = c- b_i 10^u$ and $c_2 = c$, otherwise, $c_1 = c$ and $c_2 = c - b_i 10^u$.
It is easy to observe that $c_1 \le c - b_i \ivarx_{\sigma(i)} \le c_2$.
Then we can apply Lemma~\ref{lem:exp-ineq} to the following two inequalities to eliminate $10^{\ivarx_{\sigma(i-1)}}$,
%\vspace{-2mm}
\begin{equation}\label{eqn-lb}
\sum_{j=1}^{i-1} a_j 10^{\ivarx_{\sigma(j)}} + \sum_{k=1}^{i-1}b_k \ivarx_{\sigma(k)} \le c_1
\end{equation}
and
%\vspace{-2mm}
\begin{equation}\label{eqn-ub}
\sum_{j=1}^{i-1} a_j 10^{\ivarx_{\sigma(j)}} + \sum_{k=1}^{i-1}b_k \ivarx_{\sigma(k)} \le c_2.
\end{equation}

Let $\alpha_1 = \ell_{10}(c_1) -  \ell_{10}(a_{i-1})$ and $\alpha_2 = \ell_{10}(c_2) -  \ell_{10}(a_{i-1})$. Then from Lemma~\ref{lem:exp-ineq}, let $A\Def 1+\sum_{j=1}^{i-2}\vert a_j\vert$, 
$B \Def 1+\sum_{j=1}^{i-1}\vert b_j\vert$, 
$\delta\Def  \ell_{10}(A)+2$,
and $\gamma \Def 2\ell_{10}(B)+3$. 
\begin{itemize}
\item if $\ivarx_{\sigma(i-1)} \ge \gamma$, $\ivarx_{\sigma(i-1)} \le \alpha_1 - 1$, and $\ivarx_{\sigma(i-1)} \ge \ivarx_{\sigma(i-2)} + \delta$, then inequality (\ref{eqn-lb}), thus also $\tau'$, is evaluated to $\ltrue$,
\item if $\ivarx_{\sigma(i-1)} \ge \gamma$, $\ivarx_{\sigma(i-1)} \ge \alpha_2 + 2$, and $\ivarx_{\sigma(i-1)} \ge \ivarx_{\sigma(i-2)} + \delta$, then inequality (\ref{eqn-ub}), thus also $\tau'$, is evaluated to $\lfalse$.
\end{itemize}

Therefore, $\tau'$ and $\tau$ are equivalent to 
\begin{equation}\label{lemma-example}\begin{aligned}
    &\bigvee_{p=0}^{\gamma-1} (\ivarx_{\sigma(i-1)}=p \wedge \tau[p/\ivarx_{\sigma(i-1)}])\\
    \vee &\bigvee_{p=0}^{\delta-1} \bigg(\ivarx_{\sigma(i-1)}=\ivarx_{\sigma(i-2)}+p\wedge \tau[\ivarx_{\sigma(i-2)}+p/\ivarx_{\sigma(i-1)}]\wedge \ivarx_{\sigma(i-1)}\le \alpha_1-1\bigg)\\
    \vee &\bigvee_{p=0}^{\delta-1} \bigg(\ivarx_{\sigma(i-1)}=\ivarx_{\sigma(i-2)}+p\wedge \tau[\ivarx_{\sigma(i-2)}+p/\ivarx_{\sigma(i-1)}]\wedge \ivarx_{\sigma(i-1)}\ge \alpha_2+2\bigg)\\
    \vee & \bigg( \ivarx_{\sigma(i-1)}\ge \gamma \wedge \ivarx_{\sigma(i-1)} \ge \ivarx_{\sigma(i-2)}+\delta \wedge \ivarx_{\sigma(i-1)}\le \alpha(\vec\ivary)-1\bigg)\\
    \vee & \bigvee_{p=\alpha_1}^{\alpha_2+1} \bigg( \ivarx_{\sigma(i-1)}\ge \gamma \wedge \ivarx_{\sigma(i-1)} \ge \ivarx_{\sigma(i-2)}+\delta \wedge \ivarx_{\sigma(i-1)} = p \wedge \tau[p/\ivarx_{\sigma(i-1)}]\bigg),
\end{aligned}\end{equation}
where all exponential occurrences of $\ivarx_{\sigma(i-1)}$ are eliminated.

\hide{
\[
\small
\begin{array}{l}
\bigvee \limits_{p=0}^{B-1} ( \ivarx_{\sigma(i-1)} = p \wedge \tau'(\ivarx_{\sigma(i)}, \ldots, \ivarx_{\sigma(1)}) [p/\ivarx_{\sigma(i-1)}])   \\
\bigvee \big(\ivarx_{\sigma(i-1)} \ge B \wedge \ivarx_{\sigma(i-1)} \le \alpha_1  - 1  \wedge \ivarx_{\sigma(i-1)} \ge \ivarx_{\sigma(i-2)} +\delta \big)\\
%
\bigvee \bigvee \limits_{p=0}^{\delta-1} 
\left(
\begin{array}{l}
\ivarx_{\sigma(i-1)} \ge B \wedge \ivarx_{\sigma(i-1)} \le \alpha_1  -1 \ \wedge \\
 \ivarx_{\sigma(i-1)} = \ivarx_{\sigma(i-2)} +p\ \wedge\\
 \tau'(\ivarx_{\sigma(i)}, \ldots, \ivarx_{\sigma(1)})[\ivarx_{\sigma(i-2)} + p /\ivarx_{\sigma(i-1)}] 
\end{array}
\right)\\
\bigvee \bigvee \limits_{p=0}^{\delta-1} 
\left(
\begin{array}{l}
\ivarx_{\sigma(i-1)} \ge B \wedge \ivarx_{\sigma(i-1)} \ge \alpha_2 + 2 \ \wedge \\
 \ivarx_{\sigma(i-1)} = \ivarx_{\sigma(i-2)} +p\ \wedge\\
 \tau'(\ivarx_{\sigma(i)}, \ldots, \ivarx_{\sigma(1)})[\ivarx_{\sigma(i-2)} + p /\ivarx_{\sigma(i-1)}] 
\end{array}
\right)\\
\bigvee \bigvee \limits_{p = \alpha_1}^{\alpha_2 + 1}
\left(
\begin{array}{l}
\ivarx_{\sigma(i-1)} \ge B \wedge \ivarx_{\sigma(i-1)} = p\ \wedge \\
 \tau'(\ivarx_{\sigma(i)}, \ldots, \ivarx_{\sigma(1)})[p /\ivarx_{\sigma(i-1)}] 
\end{array}
\right),
\end{array}
\]
}

Similarly, we can eliminate the exponential occurrences of $\ivarx_{\sigma(i-2)}$ from $\tau'[\ivarx_{\sigma(i-2)} + p /\ivarx_{\sigma(i-1)}]$ as well as  $\tau'[p /\ivarx_{\sigma(i-1)}]$, and so on. Eventually, we obtain a {\pa} formula.

%We also propose some additional optimizations, which are omitted due to the page limit and can be found in Appendix~\ref{app-opt}.

\subsection{Additional optimization techniques}\label{para: opts}
\paragraph{Eliminate exponential occurrences in atomic formulas simutaneously}

Although Lemma~\ref{lem:exp-ineq} is stated for a single atomic formula, the elimination of the exponential occurrences of the same variable in different atomic formulas can actually be conducted simutaneously. That is,  let $\alpha^\tau_{1}, \alpha^\tau_{2}, \gamma^\tau, \delta^\tau$ be the constants as stated in the aforementioned under-approximation of an inequality $\tau$, define $\alpha^{\min}_1, \alpha^{\max}_2, B^{\max}, \delta^{\max}$ as the minimum of $\alpha^\tau_1$, the maximum of $\alpha^\tau_2$, the maximum of $B^\tau$, and the maximum of $\delta^\tau$ respectively with $\tau$ ranging over the inequalities of $\varphi$. Then we can use the same constants $\alpha^{\min}_1, \alpha^{\max}_2, B^{\max}, \delta^{\max}$ for different inequalities when eliminating the exponential occurrences of the same variable. 

\paragraph{Avoid the formula-size blow-up by depth-first search}

The {\pa} formula resulting from the elimination of exponential occurrences is essentially a disjunction of large number of disjuncts of small size. If we store this large formula naively, then its size quickly blows up and exhausts the memory. Alternatively, we choose to do a depth-first search (DFS) and consider the disjuncts, which are of small sizes, one by one, and solve the satisfiability problem for these disjuncts. When a satisfiable disjunct is found, then the search terminates and ``SAT'' is reported.

\paragraph{Preprocess with small upper bound}

We believe that if a quantifier-free {\paexp} formula is satisfiable, then more likely it may be satisfied with an assignment in which all variables are uniformly bounded. Consequently, as a preprocessing step, we put a small upper bound, e.g. the biggest constant occurring in the formula, on the values of variables, and perform a depth-first search, so that a model can be quickly found if there is any. If this preprocessing is unsuccessful, then we continue the search with the greater upper bound $10^u$ for some proper predefined $u \in \mathbb N$.
