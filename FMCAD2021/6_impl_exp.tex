%!TEX root = paper.tex

\subsection{Implementation}
We implemented the decision procedure in Wolfram Mathematica, and obtain a solver, called the {\paexp}-solver, which is able to solve the satisfiability of quantifier-free {\paexp} formulas.

The {\paexp}-solver takes a quantifier-free {\paexp} formula as the input. Moreover, it allows specifying a constant upper bound $10^u$ for the values of variables. If a constant upper bound $10^u$ is specified, then the problem is to decide whether there is an assignment of values no more than $10^u$ to variables satisfying the given {\paexp} formula.

The outputs of the {\paexp}-solver are either ``SAT'', ``UNSAT'', ``B-UNSAT'', or ``TIMEOUT'', corresponding to the facts that the given formula is satisfiable, unsatisfiable, unsatisfiable up to $10^u$, or the search goes beyond the predetermined time limit. If the output is ``SAT'', then a model (namely, an assignment of values to variables) is also returned.

%a prototype of our algorithm in Wolfram Mathematica, called  $\paexp$-Solver. Any postive Integer is allowed to be the base of exponential functions in our algorithm, but $\parseInt$ in Z3 and CVC4 only takes a numeric string as input. For this reason, we fix the exponential function to be $10^x$ in our experiments.

%%%%%%%%%%%%%%%%%%%%%%%%%%%%%%%%%%%%%%%%%%%%%%%%%%
%%%%%%%%%%%%%%%%%%%%%%%%%%%%%%%%%%%%%%%%%%%%%%%%%%
\hide{
Our solver takes an existential quantified {$\paexp$} formula together with an upper bound for variables. As we have mentioned in \textbf{modified Elim-exp}, if there is no variables only occurs linearly in exponential constraints, the upper bound can be omitted, otherwise it needs to be specified manually. So for arithmetic constraints, we have generated benchmarks of two forms, named \textbf{BOUND} and \textbf{UNBOUND}, depends on whether the bound is needed.

The outputs of our solver can be SAT, UNSAT, B-UNSAT (short for bounded unsat) or TIMEOUT. For satisfiable cases, if a satisfying assignment is found, our solver returns ``SAT" together with the model. For unsatisfiable cases, our solver will decide whether the manually specified bound for variables is needed. The output ``UNSAT" means the formula is unsatisfiable over $\Nat$, while ``B-UNSAT" means no solution within the range. ``TIMEOUT" will be returned if the computation excesses the time limit. 
}
%%%%%%%%%%%%%%%%%%%%%%%%%%%%%%%%%%%%%%%%%%%%%%%%%%
%%%%%%%%%%%%%%%%%%%%%%%%%%%%%%%%%%%%%%%%%%%%%%%%%%

\subsection{Benchmarks}

To evaluate the performance of the {\paexp}-solver, we created two benchmark suites, ARITHMETIC, and STRINGHASH.
%\begin{itemize}
%\item 

\paragraph{The ARITHMETIC benchmark suite} 
This suite comprises three groups of randomly generated {\paexp} formulas. Each group is characterized by four parameters ($EV$, $LV$, $EI$, $LI$), where $EV, LV$ represent the number of variables with exponential occurrences and with only linear occurrences respectively, and $EI, LI$ represent the number of inequalities with exponential terms and with only linear terms respectively. 
We consider three parameter classes, $(2, 3, 3, 4)$, $(3, 4, 4, 5)$, and $(4, 5, 5, 6)$. 
Each group of the ARITHMETIC benchmark suite consists of 200\zhilin{check later} randomly generated problem instances. The coefficients of exponential terms are randomly selected from the interval $[-10, 10]$ and the other coefficients/constants are randomly selected from the interval $[-10^5, 10^5]$. The two intervals are chosen with the intention that the coefficients of exponential terms are smaller so that they do not always dominate the left-hand side of the inequalities. Moreover, we randomly replace some occurrences of $\le$ by $=$ in order to include some equalities in the problem instances. We also create SMTLib2 files for each problem instance, in order to facilitate the comparison with the state-of-the-art of SMT solvers CVC4 and Z3. Because neither CVC4 nor Z3 supports the exponential functions natively,  in the SMTLib2 files, we encode $10^x$ as a recursive function $f(x)$ defined by: $f(0) = 10$ and $f(n+1) = 10*f(n)$.

%
%\item 

\paragraph{The STRINGHASH benchmark suite} 
This suite comprises six groups of string constraints generated from the string hash functions ${\sf hash}(w)$ such that for each $w= a_1 \ldots a_n$, 
${\sf hash}(w) = \sum \limits_{i = 1}^n a_i p^{n-i} \bmod m$, where $p, m \in \Int^+$ with $p, m \ge 2$. It is easy to see that ${\sf hash}(w)$ can be seen as a generalization of $\parseInt$ followed by a modulo operation. In particular, if $\Sigma = \Sigma_{\sf num}$ and $p = 10$, then ${\sf hash}(w) = \parseInt(w) \bmod m$. 
Aiming at a comparison with the state-of-the-art string constraint solvers that support $\parseInt$, e.g. CVC4, Z3, and Trau, we assume $\Sigma =\Sigma_{\sf num}$ and $p = 10$, so that the string hash function can be encoded by $\parseInt$. 

The string constraints in the STRINGHASH benchmark suite are of the form 
$$\svarx \in \Aut \wedge \left( \parseInt(\svarx) \bmod m \right) \bmod m^\prime = 0   \notag \wedge \len(\svarx) < 100,$$ 
where $\Aut$ is an FA, $m, m^\prime \ge 2.$ 
The six groups of string constraints are characterized by different classes of regular constraints, namely, $12345w^+_1 w^+_2$,  $12345 w^+_1  w^+_2 6789$, $w^+_1 w^+_2 6789$, $12345\Sigma_{\sf num}^*$, $12345\Sigma_{\sf num}^* 6789$, and $\Sigma_{\sf num}^* 6789$, where $w_1,w_2 \in \Sigma^+_{\sf num}$. The first three are flat languages, while the latter three are not. We generate the SMTLib2 files for these string constraints, as inputs to the SMT solvers. For the {\paexp}-solver, we do the following.   
\begin{itemize}
\item For flat regular constraints, we generate the {\paexp} formulas corresponding to the string constraints, as the inputs to the {\paexp}-solver.
%
\item For non-flat regular constraints, we use flat languages $(a)^* (b_{1} \ldots b_{k})$ to under approximate 
$\Sigma_{\sf num}^*$,  where $a, b_1,\ldots, b_k \in \Sigma_{\sf num}$. 
%
%the first part is a single digit and the second part is a string length $l$, which can be seen as a FL with 2 cycles and force the second cycle do not repeat. 
%
We iterate the following procedure to find a model or reach the time limit: first set $k=1$ and iterate by assigning $0, \ldots, 9$ to $a$. For each assignment, we encode the resulting string constraints into the {\paexp} formulas with only one exponential variables. If the resulting {\paexp} formula is unsatisfiable, then we increase $k$ by 1 and repeat this process. 
\end{itemize}
We would like to remark that the flattening strategy for non-flat regular constraints is a strict generalization of that in \cite{Parosh:20:PLDI}: Patterns of the form $0^*(b_1...b_k)$ were considered therein and {\pa} formulas are sufficient to encode such patterns. On the other hand, we consider patterns of the form e.g. $12345 (a)^* (b_1 \ldots b_k) 6789$ (where $a \in \Sigma_{\sf num}$ can be nonzero), which requires {\paexp} formulas to encode.

%\end{itemize}

%%%%%%%%%%%%%%%%%%%%%%%%%%%%%%%%%%%%%%%%%%%%%%%%%%%%
%%%%%%%%%%%%%%%%%%%%%%%%%%%%%%%%%%%%%%%%%%%%%%%%%%%%
\hide{
We are not aware of any SMT solver or constraints solver that directly support arithmetic formulas with exponential functions over Integer domain. In Z3 and CVC4, we need to define exponential functions as recursive functions. Whereas string constraints with $\parseInt$ constraints are supported in Z3, CVC4 and other string constraints solvers like Trau, HAMPI etc. Therefore, we designed two benchmarks to evaluate our $\paexp$-Solver. The first benchmark is pure arithmetic formulas with exponential functions generated randomly, we compared our solver with Z3 and CVC4. The second benchmark is string constraints about hash functions, we tranlated them to arithmetic constraints for our solver and compared the result with Z3, CVC4 and Trau, who treated this benchmark as just string constraints. The details of the two benchmarks will be illustrate following:

\paragraph{arithmetic}

We have 8 groups of experiments for arithmetic constraints in all, including 4 groups for \textbf{BOUND} case and 4 for \textbf{UNBOUND} case.

Each group consists of 100 trails paratrized by 4 natuaral number parameters, which in turn represent the number of exponential variables, linear variables, exponential inequalities and linear inequalities.For \textbf{BOUND} case, assume $m$ exponential variables and $(n-m)$ linear variables, an exponential inequalities is of the form
$$\sum_{i=1}^m a_i 10^{x_i} + \sum_{j=1}^n b_j x_j \le c$$

For \textbf{UNBOUND} case, we further restrict in the exponential inequalities, $a_i\neq 0$ for $i:1\le i\le m$ and $b_j=0$ for $j:m+1\le j \le n$.

For both cases, we always require the number of constraints is more than the number variables. It is reasonable because the coefficients are randomly generated and some constraints may become inactive. We also have 2 groups include equalities by changing the first exponential inequality into equality.  

Coefficients $a_i$, $b_j$ and $c$ are generated randomly for each constarint within the same range. We set $a_i\in [-10,10]$, and $b_j,c\in[-10^5,10^5]$. $a_i$ is set smaller so that exponential terms will not always dominate the left hand side of an inequality.

%\paragraph{string hash function}

In the sequel, we consider string hash functions of the form  
\begin{equation}
    \text{hash}(x)= \parseInt(x) \mod m_1  \notag
\end{equation}
where $m_1$ is usually a large (prime) number. We would like to find if there exists strings that match desired patterns and are hashed into certain integer values. In our benchmark, we focus string constraints of the form
\begin{equation}
    \left( \parseInt(x) \mod m_1\right) \mod m_2 = 0   \notag
\end{equation}
where $x$ either has a prefix ``12345'' or a suffix ``6789'' or both, for example $x=``12345"y``6789"$. We choose 2 patterns for $y$ to match: 

The first pattern is a FL with 2 cycles, $y \in (a)^+(b)^+$. Here we assume each cycle repeat at least once so that it will not degrade into FL with 1 cycle. The constraints is encoded to a $\paexp$ formula where two exponential variables denote the number of cycle $a$ and $b$ respectively.\wuhao{Need example?}

The Second pattern is arbitrary string, that is $y\in \Sigma_{num}^*$. We use languages of the form $(a^1_1)^* (a^2_1...a^2_l)$ to under-approximate arbitrary strings, the first part is a single digit and the second part is a string length $l$, which can be seen as a FL with 2 cycles and force the second cycle do not repeat. We first set $l=1$ and enumerate $a^1_1$ from $1$ to $9$, for each case we obtain an arithmetic formula with $1$ exponential variables. If no satisfying model is found, we increase $l$ by 1 and repeat this procedure until it finds a model or timeout. This strategy is similar to \cite{Abdulla2020} with the difference that they force $a^1_1=0$ so that no exponential terms will occur.
}
%%%%%%%%%%%%%%%%%%%%%%%%%%%%%%%%%%%%%%%%%%%%%%%%%%%%
%%%%%%%%%%%%%%%%%%%%%%%%%%%%%%%%%%%%%%%%%%%%%%%%%%%%

\subsection{Experiment Results}

We compare the {\paexp}-solver against the state-of-the-art SMT solvers on the generated benchmarks. Specifically, 
\begin{itemize}
\item over the ARITHMETIC benchmark suite, we compare the {\paexp}-solver against CVC4 and Z3,
\item over the STRINGHASH benchmark suite, we compare the {\paexp}-solver against CVC4, Z3, and Trau. 
\end{itemize}
All the experiments are run on a lap-top with the Intel i5 1.4GHz CPU and 8GB memory. We set the timeout threshold as 60 seconds per problem instance. 

The experiment results are summarized in Table ~\ref{xxx}. 

\zhilin{stopped here}



\wuhao{where to place tables?}

From Table \ref{table:arithmetic}, we can see that for all cases except bounded 4-5-5-6 case, $\paexp$-Solver answers more \textit{sat} and \textit{unsat} than Z3 and CVC4. Our solver significantly outperforms when equalities are included. The average time needed to assert \textit{sat} or \textit{unsat} increase when we add more exponential variables.\wuhao{results of unsat cases are not correct!}


For string hash function benchmark, we did compared our solver with Z3, CVC4 and Trau, but the experiments show that Trau would incorrectly answer \textit{unsat} in most cases even though a model can be found. So in \ref{table:string}, we still show the results of the other 3 solvers. As we can see, for flat patterns $(a)^*(b)^*$, Z3 and CVC4 would run out of time on most cases while $\paexp$-solver can give an answer for almost all trails. For arbitrary strings $(\Sigma_{num})^*$, CVC4 and Z3 seem to be biased against trails with surfix or prefix. By applying the under-approximation technique, our tool is able to find a match on all trails. \wuhao{more details needed}


% Please add the following required packages to your document preamble:
% \usepackage{multirow}
\begin{table*}[]
    \label{table:arithmetic}
    \caption{Results of Z3, CVC4 and $\paexp$-Solver on arithmetic benchmark}
    \begin{tabular}{cc|cc|cc|cc}
    \multicolumn{1}{l}{} &  & \multicolumn{2}{c|}{Z3} & \multicolumn{2}{c|}{CVC4} & \multicolumn{2}{c}{MySolver} \\
    \multicolumn{1}{l}{} &  & number & avg & number & avg & number & avg \\ \hline
    \multirow{3}{*}{\begin{tabular}[c]{@{}c@{}}unbound\\ 2-3-3-4\end{tabular}} & SAT & 31 & \textless{}0.1 & 26 & 1.2 & 35 & \textless{}0.1 \\
     & UNSAT & 37 & \textless{}0.1 & 38 & \textless{}0.1 & 65 & 0.3 \\
     & TIMEOUT & 32 &  & 36 &  & 0 &  \\ \hline
    \multirow{3}{*}{\begin{tabular}[c]{@{}c@{}}unbound\\ 3-4-4-5\end{tabular}} & SAT & 27 & \textless{}0.1 & 16 & \textless{}0.1 & 34 & 0.4 \\
     & UNSAT & 22 & \textless{}0.1 & 24 & \textless{}0.1 & 66 & 2.8 \\
     & TIMEOUT & 51 &  & 60 &  & 0 &  \\ \hline
    \multirow{3}{*}{\begin{tabular}[c]{@{}c@{}}unbound\\ 3-4-4-5 \\ with equality\end{tabular}} & SAT & 4 & 19 & 0 & \textless{}0.1 & 16 & 3.7 \\
     & UNSAT & 30 & \textless{}0.1 & 31 & 0.2 & 44 & \textless{}0.1 \\
     & TIMEOUT & 66 &  & 69 &  & 28 & 19.6 \\ \hline
    \multirow{3}{*}{\begin{tabular}[c]{@{}c@{}}unbound\\ 4-5-5-6\end{tabular}} & SAT & 20 & 0.2 & 15 & 8.2 & 33 & 5.2 \\
     & UNSAT & 22 & \textless{}0.1 & 23 & 0.2 & 50 & 4.8 \\
     & TIMEOUT & 58 &  & 62 &  & 17 &  \\ \hline\hline
    \multirow{4}{*}{\begin{tabular}[c]{@{}c@{}}bound\\ 2-3-3-4\end{tabular}} & SAT & 31 & \textless{}0.1 & 14 & \textless{}0.1 & 37 & 0,6 \\
     & UNSAT & 31 & \textless{}0.1 & 30 & \textless{}0.1 & 46 & \textless{}0.1 \\
     & B-UNSAT &  &  &  &  & 17 & 9.3 \\
     & TIMEOUT & 38 &  & 56 &  & 0 &  \\ \hline
    \multirow{4}{*}{\begin{tabular}[c]{@{}c@{}}bound\\ 3-4-4-5\end{tabular}} & SAT & 31 & 0.8 & 13 & \textless{}0.1 & 36 & 3.2 \\
     & UNSAT & 20 & \textless{}0.1 & 20 & \textless{}0.1 & 42 & \textless{}0.1 \\
     & B-UNSAT &  &  &  &  & 3 & 54 \\
     & TIMEOUT & 49 &  & 67 &  & 19 &  \\ \hline
    \multirow{4}{*}{\begin{tabular}[c]{@{}c@{}}bound\\ 3-4-4-5 \\ with equality\end{tabular}} & SAT & 3 & 23 & 1 & 39 & 14 & 27.5 \\
     & UNSAT & 25 & \textless{}0.1 & 25 & \textless{}0.1 & 47 & \textless{}0.1 \\
     & B-UNSAT &  &  &  &  & 2 & 57.9 \\
     & TIMEOUT & 72 &  & 74 &  & 37 &  \\ \hline
    \multirow{4}{*}{\begin{tabular}[c]{@{}c@{}}bound\\ 4-5-5-6\end{tabular}} & SAT & 29 & 2 & 11 & \textless{}0.1 & 28 & 10 \\
     & UNSAT & 20 & \textless{}0.1 & 20 & \textless{}0.1 & 38 & \textless{}0.1 \\
     & B-UNSAT &  &  &  &  & 0 &  \\
     & TIMEOUT & 51 &  & 69 &  & 34 & 
    \end{tabular}
\end{table*}

% Please add the following required packages to your document preamble:
% \usepackage{multirow}
\begin{table*}[]
    \label{table:string}
    \caption{Results of Z3, CVC4 and $\paexp$-Solver on string hash function benchmark}
    \begin{tabular}{c|c|ccc|ccc|ccc}
                                                       &     & \multicolumn{3}{c|}{Z3} & \multicolumn{3}{c|}{CVC4} & \multicolumn{3}{c}{MySolver}       \\
    PATTERN                                            & $\ell_{10}(m_1)$   & SAT  & UNSAT  & TIMEOUT & SAT   & UNSAT  & TIMEOUT  & SAT & UNSAT & TIMEOUT              \\ \hline
    \multirow{5}{*}{12345(a)*(b)*}                     & 2   & 5    &        & 20      & 23    &        & 2        & 24  & 1     & 0                    \\
                                                       & 3   & 0    &        & 25      & 4     &        & 21       & 10  & 14    & 1                    \\
                                                       & 4   & 0    &        & 25      & 1     &        & 24       & 2   & 22    & 1                    \\
                                                       & 5   & 0    &        & 25      & 1     &        & 24       & 1   & 23    & 1                    \\
                                                       & total & 5    &        & 95      & 29    &        & 71       & 37  & 60    & 3                    \\ \hline
    \multirow{5}{*}{12345(a)*(b)*6789}                 & 2   & 10   &        & 15      & 18    &        & 7        & 24  & 1     & 0                    \\
                                                       & 3   & 1    &        & 24      & 8     &        & 17       & 10  & 15    & 0                    \\
                                                       & 4   & 0    &        & 25      & 3     &        & 22       & 3   & 22    & 0                    \\
                                                       & 5   & 0    &        & 25      & 0     &        & 25       & 0   & 25    & 0                    \\
                                                       & total & 11   &        & 89      & 29    &        & 71       & 37  & 63    & 0                    \\ \hline
    \multirow{5}{*}{(a)*(b)*6789}                      & 2   & 14   &        & 11      & 23    &        & 2        & 22  & 3     & 0                    \\
                                                       & 3   & 4    &        & 21      & 6     & 1      & 18       & 15  & 10    & 0                    \\
                                                       & 4   & 0    &        & 25      & 1     &        & 24       & 3   & 22    & 0                    \\
                                                       & 5   & 0    &        & 25      & 0     &        & 25       & 1   & 24    & 0                    \\
                                                       & total & 18   &        & 82      & 30    & 1      & 69       & 41  & 59    & 0                    \\ \hline\hline
    \multirow{5}{*}{12345$(\Sigma_{num})^*$}     & 2   & 25   &        & 0       & 25    &        & 0        & 25  &       &                      \\
                                                       & 3   & 25   &        & 0       & 25    &        & 0        & 25  &       &                      \\
                                                       & 4   & 23   &        & 2       & 25    &        & 0        & 25  &       &                      \\
                                                       & 5   & 9    &        & 16      & 25    &        & 0        & 25  &       &                      \\
                                                       & total & 82   &        & 18      & 100   &        & 0        & 100 &       &                      \\ \hline
    \multirow{5}{*}{12345$(\Sigma_{num})^*$6789} & 2   & 25   &        & 0       & 15    &        & 10       & 25  &       & \multicolumn{1}{l}{} \\
                                                       & 3   & 23   &        & 2       & 1     &        & 24       & 25  &       & \multicolumn{1}{l}{} \\
                                                       & 4   & 9    &        & 16      & 1     &        & 24       & 25  &       & \multicolumn{1}{l}{} \\
                                                       & 5   & 3    &        & 22      & 0     &        & 25       & 25  &       &                      \\
                                                       & total & 60   &        & 40      & 17    &        & 83       & 100 &       &                      \\ \hline
    \multirow{5}{*}{$(\Sigma_{num})^*$6789}      & 2   & 25   &        & 0       & 19    &        & 6        & 25  &       & \multicolumn{1}{l}{} \\
                                                       & 3   & 25   &        & 0       & 4     &        & 21       & 25  &       & \multicolumn{1}{l}{} \\
                                                       & 4   & 16   &        & 9       & 4     &        & 21       & 25  &       & \multicolumn{1}{l}{} \\
                                                       & 5   & 2    &        & 23      & 0     &        & 25       & 25  &       & \multicolumn{1}{l}{} \\
                                                       & total & 68   &        & 32      & 27    &        & 73       & 100 &       &                     
    \end{tabular}
\end{table*}