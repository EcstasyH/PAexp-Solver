%!TEX root = paper.tex

\subsection{Complexity analysis}\label{app:cpx}

In this part, we analyse the complexity of the quantifier elimination procedure applied on a formula with one existential quantifer, that is, a formula of the form $\exists \ivarx. \varphi(\ivarx, \vec{\ivary})$. We will prove that the complexity of eliminating one quantifer has a 3-EXPSPACE upper bound. Since the quantifier elimination procedure for $\pa$ works as a subprocedure (in eliminating linear occurrences of variables) and also has a 3-EXPSPACE complexity, the bound may not be easily improved.

We first analyse the changes of formula length in the normalization step. For the rest of the decision procedure, we adopt the style in the complexity analysis of \pa \ (cf. \cite{Oppen73}). The idea is that the upper bound of the formula length can be expressed using the product of the number of atoms, the number of coefficients and the length of the maximum constant. So we track the changes of such quantities throughout the procedure.

Given a formula $\exists \ivarx. \varphi(\ivarx, \vec{\ivary})$ with length $n$. After the normalization step, suppose we obtain a formula with $m$ quantified variables $\exists \ivarx_1 \dots\exists \ivarx_m. \varphi'(\ivarx_1,\dots, \ivarx_m, \vec{\ivary})$. We show that the length of the new formula is at most $10n$ and the number of quantified variables $m\le n$. In each sub-step of normalization, we analyse the worst situation: 
(1) \textit{NNF transformation} suppose all atoms are of the form $\iterm_1=\iterm_2$, then taking negation will double the number of atoms, the length increases to $2n$ at most. 
(2) \textit{replace $\ell_{10}(\iterm)$ terms} the original formula has at most $n$ terms of the form $\ell_{10}(\iterm)$, for each of them, we introduce a fresh variable and two conjuncts. The formula length increases to $4n$.
(3) \textit{flatten $10^\iterm$ terms} similar to (2), the formula has at most $n$ such terms and for each term, a fresh variable and a conjunct are added. The formula length increases to $5n$. 
(4) \textit{$\le$ transformation} similar to (1), here we assume all atoms are of the form $\iterm_1  = \iterm_2$, so the length of the formula with increase to $10n$ at most. 

The analysis for the normalization step is coarse, for example, the worst case in (1)\&(4) or (2)\&(3) can not happen at the same time. However, what we need is that the increased length of the formula is bounded by a constant factor,i.e. 10. In addition, we can also conclude that the number of fresh variables, $m$, is less than $n$ since there are at most $n$ different forms of terms. 

After the normalization step, denote the obtained formula by $\exists \vec{\ivarx}. \varphi(\vec{\ivarx},\vec{\ivary})$ with quantified variables $\vec{\ivarx} = (\ivarx_1,\dots,\ivarx_m)$ and free variables $\vec{\ivary}$. $\varphi$ contains no quantifiers. 
Let $n'$ denote the length of the formula.

According to our quantifier elimination procedure,we eliminate the quantified variables one by one: each time select the largest variable among $\ivarx_i,1\le i\le m$, first eliminate the exponential occurrences, then linear occurrences. The linear order of quantified variables are given by a permutation. For $m$ quantified variables, there are $m!$ possible linear orders.

Suppose the specified linear order is $\ivarx_m\ge \dots \ge \ivarx_1$. Let $\exists \ivarx_1\dots \exists\ivarx_{m-k}.\varphi_k$ denote the formula obtained from $\exists \vec{\ivarx}. \varphi(\vec{\ivarx},\vec{\ivary})$ by eliminating quantifiers $\exists \ivarx_m,\dots,\exists\ivarx_{m-k+1}$. Let $\exists \ivarx_1 \dots \exists \ivarx_{m-k+1}.\varphi'_k$ denote the formula obtained from $\exists \ivarx_1\dots \exists\ivarx_{m-k+1}.\varphi_{k-1}$ by eliminating linear occurrences of $\ivarx_{m-k+1}$. Let $\varphi_0$ denote $\varphi$.

Let $c_k$ be the number of distinct $d$ in atoms of the form $d \ \big | \ \iterm$ plus the number of distinct coefficients of \emph{linear occurrences} of quantified variables in $\varphi_k$. Let $s_k$ be the largest constant (including coefficients) and $a_k$ be the number of atomic formulas in $\varphi_k$. Similarly we define $c'_k$, $s'_k$ and $a'_k$ for $\varphi'_k$. Let $c,s,a$ be $c_0,s_0,a_0$ respectively. 


First we analyse the sub-procedure to eliminate exponential occurrences of $\ivarx_m$. We prove the following lemma.

\begin{lemma}\label{lem:cpx exp}
    \begin{align}
        c'_1&\le c^2 \notag \\
        s'_1&\le ms^2 \notag\\
        a'_1&\le sa\notag 
    \end{align} 
\end{lemma}

\begin{proof}
   
The analysis is separated into 2 cases where all atoms are of the same form (inequalities atoms or divisiblilty atoms). 

If all atoms are inequalities atomic formulas of the form in Lemma~\ref{lem:exp-ineq}. We know that each atomic formula $\tau$ with exponential occurrence of $\ivarx_m$ is replaced by a new formula. Only the coefficients of linear occurrences of $\ivarx_m$ and $\ivarx_{m-1}$ will be changed: constant coefficient $1$ is introduced, and if we substitute $\ivarx_m$ by $\ivarx_{m-1}+p$ for some $p$, coefficient of linear occurrence of $\ivarx_{m-1}$ will become $b_m+b_{m-1}$ ($b_m,b_{m-1}$ are coefficients for $\ivarx_{m}$ and $\ivarx_{m-1}$ in $\tau$, see Lemma~\ref{lem:exp-ineq} ). Since the new coefficient is obtained by adding two linear coefficient together, we have $c'_1\le c^2$. Note that $\delta$ and $B$ are at most $\ell_{10}(ms)$, when we substitute $\ivarx_m$ by $\ivarx_{m-1}+\delta-1$ or by $B-1$, the largest constant in the formula is at most $s\cdot  10^{\ell_{10}(ms)}\le ms^2$. And an inequality is replaced by at most $4\ell_{10}(ms)$ atomic formulas, so $a'_1\le 4\ell_{10}(ms)a$.

If all atoms are divisiblilty atomic formulas of the form $d \ \big | \ \iterm$ or $d \ \nmid \ \iterm$. We have $c'_1<2c$ since a divisibility atomic formula  will produce at most two forms of atomic formulas $d \ \big | \ \iterm$ and $\phi(d) \ \big | \ \iterm$. Note that any constant in $\iterm$ , say $l$, can be replaced by ($l \bmod d)$, so $s'_1\le s$. When $d$ is a large prime number, $\phi(d)=d-1$, a divisibility atomic formula is replaced by roughly $d$ atomic formulas, so $a'_1\le sa$. 

Choose larger upper bound for $c'_1$, $s'_1$ and $a'_1$ respectively, then the lemma is proved.

\end{proof}

Since linear occurrences of $\ivarx_m$ are eliminated using Cooper's algorithm, we combine Oppen's analysis for this sub-procedure:

\begin{lemma}\label{lem:cpx oppen}
    \begin{align}
        c_1&\le c'^4 \notag \\
        s_1&\le s'^{4c'} \notag\\
        a_1&\le {a'}^4 s'^{2c'}\notag 
    \end{align} 
\end{lemma}

From Lemma~\ref{lem:cpx exp} and Lemma~\ref{lem:cpx oppen}, we have
\begin{align}
    c_1&\le c^8 \notag \\
    s_1&\le (ms^2)^{4c^2} \notag\\
    a_1&\le (sa)^4(ms^2)^{2c^2}\notag 
\end{align} 

Assuming $m\le n$, by induction on $k$ we get

\begin{lemma}
    \begin{align}
        c_k&\le c^{8^k} \notag \\
        s_k&\le n^{(4c)^{ 8^k}} s^{(8c)^{ 8^k}} \notag\\
        a_k&\le a^{4^k}n^{(4c)^{ 8^k}} s^{(8c)^{ 8^k}} \notag
    \end{align} 
\end{lemma}

If we assume $c\le n$, $a\le n$ and $s\le n$, the space required to store the quantifier free formula, $\varphi_k$, is bounded by the product of the number of linear orders $m!$, the number of atoms $a_k$, the maximum number of constants $2m+2$ per atom, the maximum amount of space $s_k$ to store each constant and some constant $q$:
 
$${\sf space}\le q \cdot m! \cdot a_k \cdot (2m+2) \cdot s_k \le 2^{2^{2^{pn \log n}}}$$
for some large constant $p$. 

REMARK: When the original formula has no alternating quantifiers, for example, the formula of the form $\exists \ivarx_m \exists \ivarx_{m-1} \dots\exists \ivarx_1.\varphi(\ivarx_1,\dots,\ivarx_{m-1},\ivarx_m,\vec{\ivary})$. Since we can carry out the normalization step for all quantified variable $\ivarx_i(1\le i\le m)$ at the same time and enumerate the linear order for all introduced variables, the complexity of eliminating all quantifers is still 3-EXPSPACE. However, if there are  alternating quantifers at the very beginning, say the formula is of the form $\forall \ivarx_2\exists \ivarx_1.\varphi(\ivarx_1,\ivarx_2,\vec{\ivary})$. We have to eliminate the occurrences of $\ivarx_1$ first, treating $\ivarx_2$ as a free variable like $\vec{\ivary}$. When eliminating exponential occurrences of variables in inequalities (see Lemma~\ref{lem:exp-ineq}), $\ivarx_2$ will be collected in term $\iterm(\ivary)$. Since $\iterm(\ivary)$ may change due to the  substitutions of $\ivarx_1$, when we carry out normalization step for $\ivarx_2$, the number of introduced fresh variables is not bounded by the length of original formula. Hence, we conjecture that the overall complexity for eliminating all quantifiers in a formula will be non-elementary. 
