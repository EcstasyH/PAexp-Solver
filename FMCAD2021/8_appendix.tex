%!TEX root = paper.tex


\subsection{Details of the normalization step}\label{app-norm}
%
The normalization step comprises the following sub-steps.
\begin{enumerate}
\item At first, we transform $\varphi'(\ivarx,\vec{\ivary})$ into the NNF (negation normal form). Moreover, we remove the occurrences of $\neg$ by (a) replacing $\neg c | \iterm$ with $\bigvee \limits_{j \in [c-1]} c | (\iterm+j)$, (b)  $\neg (\iterm_1 = \iterm_2)$ with $\iterm_1 < \iterm_2 \vee \iterm_2 < \iterm_1$, (c) $\neg (\iterm_1 < \iterm_2)$ with $\iterm_2 \le \iterm_1$, (d) $\neg (\iterm_1 \le \iterm_2)$ with $\iterm_2 < \iterm_1$, and so on.
%
\item Repeat the following procedure, until there are no $\lambda_{10}(\iterm)$ or $\ell_{10}(\iterm)$ with $\ivarx$ occurs in $\iterm$: Replace each occurrence of $\lambda_{10}$ in $\varphi'$, say $\lambda_{10}(\iterm)$, such that $\ivarx$ occurs in $\iterm$, by $10^{\ell_{10}(\iterm)}$. Then replace each occurrence of $\ell_{10}(\iterm)$ such that $\ivarx$ occurs in $\iterm$, introduce a fresh variable, say $\ivarz$, and replace all occurrences of $\ell_{10}(\iterm)$ by $\ivarz$, moreover, add the constraint $10^\ivarz \le \iterm < 10^{\ivarz + 1}$ as a conjunct. Let the resulting formula be $\varphi''$.
%

\item Then repeat the following procedure to $\varphi''$, until for each occurrence of $10^{\iterm}$ with $\ivarx$ occurs in $\iterm$, we have $\iterm = \ivarx$: For each occurrence of the $10^{\iterm}$ in $\varphi''$, such that $\iterm$ contains $\ivarx$ but is not $\ivarx$, introduce a fresh variable, say $\ivarz$, and replace all occurrences of $10^{\iterm}$ by $10^\ivarz$, moreover, add the constraint $\ivarz = \iterm$ as a conjunct. Let $\varphi'''$ denote the resulting formula.  

\item Do the following replacements to $\varphi'''$, so that all the atomic formulas in $\varphi^\dag$ are of the form $\iterm_1 \le \iterm_2$ or $c | \iterm$: Replace every occurrence of $\iterm_1 \ge \iterm_2$ with $\iterm_2 \le \iterm_1$. Replace every occurrence of $\iterm_1 < \iterm_2$ (resp. $\iterm_1 > \iterm_2$) with $\iterm_1 \le \iterm_2 - 1$ (resp. $\iterm_2 \le \iterm_1-1$). Replace ever occurrence of $\iterm_1 = \iterm_2$ wtih $\iterm_2 \le \iterm_1 \wedge \iterm_1 \le \iterm_2$. Let $\varphi^\dag$ the resulting formula. 
%
\item Let $\vec{\ivarz} = \ivarz_1,\ldots, \ivarz_n$ be an enumeration of the freshly introduced variables. Then the result of the normalization procedure is 
$\exists \vec{\ivarz}\exists \ivarx.\ \varphi^\dag$.
\end{enumerate}


\subsection{Proof of Lemma~\ref{lem:exp-ineq}}

Note that $\ell_{10}(n)$ is undefined for $n\le 0$ in $\paexp$. For convenience, we define for all $n\le 0$ that $\ell_{10}(n)=\ell_{10}(\max\{1,n\})=0$ in the following proof. 

\noindent {\bf Lemma~\ref{lem:exp-ineq}}.
\emph{Let  
%
$$
\begin{array}{l}
\tau(\ivarx_{\sigma(i)}, \ldots, \ivarx_{\sigma(1)}, \vec{\ivary}) \Def  \\
\hspace{4mm} 
a_i 10^{\ivarx_{\sigma(i)}}+\sum_{j=1}^{i-1} a_j 10^{\ivarx_{\sigma(j)}} + \sum_{k=1}^{i}b_k \ivarx_{\sigma(k)} \le \iterm(\vec{\ivary}).
\end{array}
$$
%
with $a_i \neq 0$, $A\Def \sum_{j=1}^{i-1}|a_j|$, 
%$B\Def  \sum_{j=1}^{i}|b_j|$, 
$B \Def 2(\ell_{10}(\sum_{j=1}^{i}|b_j|)+3)$,
and $\delta\Def  \ell_{10}(A)+3$. 
\begin{itemize}
    \item If $a_i > 0$, let $\alpha(\vec{\ivary}) \Def \ell_{10}(\iterm(\ivary))- \ell_{10}(a_i)$, then 
    \begin{itemize}
        \item if $\ivarx_{\sigma(i)} \le \alpha(\vec{\ivary})  -1$, $\ivarx_{\sigma(i)} \ge B$ and $\ivarx_{\sigma(i)} \ge \ivarx_{\sigma(i-1)} +\delta $, then $\tau(\ivarx_{\sigma(i)}, \ldots, \ivarx_{\sigma(1)}, \vec{\ivary})$ holds,
        \item if $\ivarx_{\sigma(i)} \ge \alpha(\vec{\ivary})  +2$, $\ivarx_{\sigma(i)} \ge B$ and $\ivarx_{\sigma(i)}  \ge \ivarx_{\sigma(i-1)} +\delta$, then $\tau(\ivarx_{\sigma(i)}, \ldots, \ivarx_{\sigma(1)}, \vec{\ivary})$ \textbf{does not} hold.
    \end{itemize}
    \item If $a_i < 0$, let $\alpha(\vec{\ivary})  \Def \ell_{10}(-\iterm(\ivary))- \ell_{10}(-a_i)$, then 
    \begin{itemize}
        \item if $\ivarx_{\sigma(i)} \le \alpha(\vec{\ivary})  -1$, $\ivarx_{\sigma(i)} \ge B$ and $\ivarx_{\sigma(i)} \ge \ivarx_{\sigma(i-1)} +\delta $, then $\tau(\ivarx_{\sigma(i)}, \ldots, \ivarx_{\sigma(1)}, \vec{\ivary})$ \textbf{does not} hold,
        \item if $\ivarx_{\sigma(i)} \ge \alpha(\vec{\ivary})  +2$, $\ivarx_{\sigma(i)} \ge B$ and $\ivarx_{\sigma(i)} \ge \ivarx_{\sigma(i-1)} +\delta $, then $\tau(\ivarx_{\sigma(i)}, \ldots, \ivarx_{\sigma(1)}, \vec{\ivary})$ holds.
    \end{itemize}
\end{itemize}
}

We need the following proposition for the proof of Lemma~\ref{lem:exp-ineq}.

\begin{proposition} \label{prop:1}
If $n\ge m\ge 1$ and $p \ge 2(\ell_{10}(n)- \ell_{10}(m)+1)$, then 
$n p \le m10^p$ holds.
\end{proposition}

\begin{proof}

First we show that for any $n' \in \Nat$, if $p \ge 2n'$, then $10^p \ge 10^{n'}10^{(p-n')}\ge 10^{n'}10(p-n')\ge 10^{n'}(5p+5(p-2n'))\ge  10^{n'}p$.

If $ p \ge 2 (\ell_{10}(n) - \ell_{10}(m)+1)$, then  $n p \le 10\lambda_{10}(n) p = 10 * 10^{\ell_{10}(n)} p = 10^{\ell_{10}(n)-\ell_{10}(m) + 1}  10^{\ell_{10}(m)} p$.

Because $p \ge 2(\ell_{10}(n)-\ell_{10}(m) + 1)$, we deduce that $10^{\ell_{10}(n)-\ell_{10}(m) + 1}  p \le 10^p$.  Therefore, $10^{\ell_{10}(n)-\ell_{10}(m) + 1}  10^{\ell_{10}(m)} p \le 10^{\ell_{10}(m)} 10^p \le m 10^p$.
We conclude that $np \le m 10^p$.
\end{proof}


%Then we give the proof for Theorem~\ref{thm:exp-ineq}.
\begin{proof}[Proof of Lemma~\ref{lem:exp-ineq}]
We only prove for the case $a_i > 0$, the other case is symmetric. 


Let $A\Def \sum_{j=1}^{i-1}|a_j|$, $B \Def 2(\ell_{10}(\sum_{j=1}^{i}|b_j|)+3)$, and $\delta\Def  \ell_{10}(A)+3$. 

Suppose $\ivarx_{\sigma(i)} \ge B$ and $\ivarx_{\sigma(i)} \ge \ivarx_{\sigma(i-1)} + \delta$. Moreover, let $\alpha(\vec{\ivary}) \Def  \ell_{10}(\iterm(\ivary))- \ell_{10}(a_i)$.

Note that
 \begin{equation} 
   A10^{-\delta} =A 10^{-\ell_{10}(A)-3} = \frac{A}{1000\lambda_{10}(A)} \le \frac{1}{100}.   \label{eq:thm-ineq-1}
 \end{equation}
 
 From $\ivarx_{\sigma(i)} \ge \ivarx_{\sigma(i-1)} + \delta$ and $\ivarx_{\sigma(i-1)} \ge \ldots \ge \ivarx_{\sigma(1)}$, 
we know   
$$-A 10^{\ivarx_{\sigma(i)} - \delta} \le  \sum_{j=1}^{i-1} a_j 10^{\ivarx_{\sigma(j)}} \le A 10^{\ivarx_{\sigma(i)} - \delta}$$
and
$$-(\sum_{j=1}^{i}|b_j|) \ivarx_{\sigma(i)} \le \sum_{k=1}^{i} b_k \ivarx_{\sigma(k)} \le (\sum_{j=1}^{i}|b_j|) \ivarx_{\sigma(i)}.$$ 

Moreover, let $n = 100\sum_{j=1}^{i}|b_j|$, $m = 1$, and $p = \ivarx_{\sigma(i)}$, then 
$n \ge m \ge 1$. From $2(\ell_{10}(n)- \ell_{10}(m)+1) = 2 (\ell_{10}(100 \sum_{j=1}^{i}|b_j|) + 1) = 2 (\ell_{10}(\sum_{j=1}^{i}|b_j|) + 3) = B$, we deduce that $p = \ivarx_{\sigma(i)} \ge B = 2(\ell_{10}(n)- \ell_{10}(m)+1)$.
Then according to Proposition~\ref{prop:1}, 
$$100(\sum_{j=1}^{i}|b_j|)\ivarx_{\sigma(i)} = np \le m 10^p = 10^{\ivarx_{\sigma(i)}}.$$
Thus 
$(\sum_{j=1}^{i}|b_j|)\ivarx_{\sigma(i)}  \le \frac{1}{100} 10^{\ivarx_{\sigma(i)}}.$

If $\ivarx_{\sigma(i)} \ge   \alpha(\vec{\ivary}) + 2$, then 

$
\begin{array}{l}
a_i 10^{\ivarx_{\sigma(i)}}+\sum_{j=1}^{i-1} a_j 10^{\ivarx_{\sigma(j)}} + \sum_{k=1}^{i}b_k \ivarx_{\sigma(k)} \ge \\
a_i 10^{\ivarx_{\sigma(i)}} - A 10^{\ivarx_{\sigma(i)} - \delta} - (\sum_{j=1}^{i}|b_j|) \ivarx_{\sigma(i)}  \ge \\
a_i 10^{\ivarx_{\sigma(i)}} - \frac{1}{100}10^{\ivarx_{\sigma(i)}} - \frac{1}{100} 10^{\ivarx_{\sigma(i)}} = \\
(a_i - \frac{1}{50}) 10^{\ivarx_{\sigma(i)}} \ge (a_i - \frac{1}{50}) 10^{ \alpha(\vec{\ivary}) + 2} = \\
(a_i - \frac{1}{50}) 10^{  \ell_{10}(\iterm(\ivary))- \ell_{10}(a_i) + 2} = \\
\frac{10(a_i - \frac{1}{50})}{10^{\ell_{10}(a_i)} } 10^{ \ell_{10}(\iterm(\ivary))+1} \ge \frac{10(a_i - \frac{1}{50})} {a_i} 10^{ \ell_{10}(\iterm(\ivary))+1} \ge \\
(10 - \frac{1}{5a_i}) 10^{ \ell_{10}(\iterm(\ivary))+1} > 10^{ \ell_{10}(\iterm(\ivary))+1} \ge \iterm(\ivary).
\end{array}
$

Therefore, in this case, $\tau(\ivarx_{\sigma(i)}, \ldots, \ivarx_{\sigma(1)}, \vec{\ivary})$ \emph{does not} hold.

If $\ivarx_{\sigma(i)} \le   \alpha(\vec{\ivary}) - 1$, then 

$
\begin{array}{l}
a_i 10^{\ivarx_{\sigma(i)}}+\sum_{j=1}^{i-1} a_j 10^{\ivarx_{\sigma(j)}} + \sum_{k=1}^{i}b_k \ivarx_{\sigma(k)} \le \\
a_i 10^{\ivarx_{\sigma(i)}} + A 10^{\ivarx_{\sigma(i)} - \delta} + (\sum_{j=1}^{i}|b_j|) \ivarx_{\sigma(i)} \le \\
a_i 10^{\ivarx_{\sigma(i)}} + \frac{A} {10^\delta}10^{\ivarx_{\sigma(i)}} + \frac{1}{100} 10^{\ivarx_{\sigma(i)}} \le \\
(a_i + \frac{1}{100} + \frac{1}{100})10^{\ivarx_{\sigma(i)}} = (a_i + \frac{1}{50}) 10^{\ivarx_{\sigma(i)}} \le \\
(a_i + \frac{1}{50}) 10^{\alpha(\vec{\ivary}) - 1} = (a_i + \frac{1}{50}) 10^{\ell_{10}(\iterm(\ivary))- \ell_{10}(a_i) - 1} = \\
\frac{a_i + \frac{1}{50}} {10^{\ell_{10}(a_i) + 1}} 10^{\ell_{10}(\iterm(\ivary))} = \frac{a_i + \frac{1}{50}} {10 \lambda_{10}(a_i)} 10^{\ell_{10}(\iterm(\ivary))} \le \\
\frac{a_i + \frac{1}{50}} {a_i+1} 10^{\ell_{10}(\iterm(\ivary))} \le 10^{\ell_{10}(\iterm(\ivary))} \le \iterm(\ivary).
\end{array}
$

Therefore,  in this case, $\tau(\ivarx_{\sigma(i)}, \ldots, \ivarx_{\sigma(1)}, \vec{\ivary})$ holds.


\end{proof}

We would like to remark that Lemma~\ref{lem:exp-ineq} still holds when the base of exponential function is changed to any natural number $n\ge 2$.

%\subsection{Complexity analysis of Point's algorithm}

\subsection{Elimination of exponential occurrences of variables for divisibility atomic formulas}\label{app-div}

Consider
%
$$
\begin{array}{l}
\tau(\ivarx_{\sigma(i)}, \ldots, \ivarx_{\sigma(1)}, \vec{\ivary}) \Def  \\
d\ \big |\ \big(a_i 10^{\ivarx_{\sigma(i)}} + \sum_{j=1}^{i-1} a_j 10^{\ivarx_{\sigma(j)}} + \sum_{k=1}^{i} b_k \ivarx_{\sigma(k)} 
+ \iterm(\vec{\ivary}) \big)
\end{array}
$$
with $a_i \neq 0$.
%

Let $d = 2^{r_1} 5^{r_2}  d_0$ such that $d_0$ is divisible by neither $2$ nor $5$. Moreover, let $r = \max(r_1, r_2)$. Then $d | (10^rd_0)$. 

If $d_0 = 1$, then $10^r$ is divisible by $d = 2^{r_1}5^{r_2}$. Thus for every $n \ge r$, $d \ |\ 10^n$.  Therefore, in this case, $\tau(\ivarx_{\sigma(i)}, \ldots, \ivarx_{\sigma(1)}, \vec{\ivary})$ is equivalent to 
\[
\small
\begin{array}{l}
\bigvee \limits_{p = 0}^{r - 1} d\ \big | \big(a_i 10^{p} + \sum_{j=1}^{i-1} a_j 10^{\ivarx_{\sigma(j)}} + b_kp + \sum_{k=1}^{i-1} b_k \ivarx_{\sigma(k)} 
+ \iterm(\vec{\ivary}) \big)\\
%
\vee \big(\ivarx_{\sigma(i)} \ge r \wedge d\ \big | \big(\sum_{j=1}^{i-1} a_j 10^{\ivarx_{\sigma(j)}} + \sum_{k=1}^{i} b_k \ivarx_{\sigma(k)} 
+ \iterm(\vec{\ivary}) \big)\big),
\end{array}
\]
where the exponential occurrences of $\ivarx_{\sigma(i)}$ disappear.

Next, let us assume $d_0 > 1$. Since $10$ and $d_0$ are relatively prime, according to Euler's theorem (cf. \cite{HW80}), $10^{\phi(d_0)} \equiv 1 \bmod d_0$, where $\phi$ is the Euler function. Suppose $10^{\phi(d_0)} = kd_0 + 1$ for some $k \in \Nat$. 
Then for every $n \in \Nat$ with $n \ge r$, 
$$
\begin{array}{l}
10^{n + \phi(d_0)} \bmod d =10^{n-r} 10^r (kd_0 + 1) \bmod d = \\
10^{n-r} (k 10^rd_0 + 10^r) \bmod d = \\
10^{n-r} (0+10^r) \bmod d = 10^n \bmod d.
\end{array}
$$

Then $\tau(\ivarx_{\sigma(i)}, \ldots, \ivarx_{\sigma(1)}, \vec{\ivary})$ is equivalent to 
\[
\begin{array}{l}
\bigvee \limits_{p=0}^{r-1} \tau(\ivarx_{\sigma(i)}, \ldots, \ivarx_{\sigma(1)}, \vec{\ivary})[p/\ivarx_{\sigma(i)}]\ \vee \\
\left(
\begin{array}{l}
\ivarx_{\sigma(i)} \ge r\ \wedge \\
\bigvee \limits_{q = 0}^{\phi(d_0)-1} 
\left(
\begin{array}{l}
\phi(d_0)\ \big |\ (\ivarx_{\sigma(i)} - r - q)\ \wedge \\
d\ \big | 
\left(
\begin{array}{l}
a_i 10^{r+q} + \sum_{j=1}^{i-1} a_j 10^{\ivarx_{\sigma(j)}} + \\
\sum_{k=1}^{i} b_k \ivarx_{\sigma(k)} + \iterm(\vec{\ivary})
\end{array}
\right) 
\end{array}
\right)
\end{array}
\right),
\end{array}
\]
where the exponential occurrences of $\ivarx_{\sigma(i)}$ disappear.

\subsection{Complexity analysis}\label{app:cpx}

In this part, we will analysis the complexity of the decision procedure described in Section~\ref{sec-dec}. For convenience, we will slightly change the algorithm: the normalization procedure is invoked for all quantified variables in the beginning, which means step 2 and step 3 in the normalization procedure are performed for all quantified variables instead of a specified $\ivarx$. 

Let the formula after the normalization step be $\varphi = \exists \vec{\ivarx}. \theta$ with $\vec{\ivarx} = (\ivarx_1,\dots,\ivarx_m)$. Let $n$ denote the length of formula $\varphi$.

Then we eliminate the exponential occurrences and linear occurrences of variables one by one according to a specified linear order of $\vec{\ivarx}$. Suppose the linear order is $\ivarx_1\ge \dots \ge \ivarx_m$.

We adopt the method in the analysis of the complexity of \pa \ (cf. \cite{Oppen73}).  Let $\varphi'_k = \exists \ivarx_m \dots  \exists \ivarx_{k}\theta'_k$ denote the formula after $k-1$ variables and exponential occurrences of $\ivarx_k$ are eliminated from $\varphi$. Let $\varphi_k$ denote the formula obtained by eliminating linear occurrences of $\ivarx_k$ from $\varphi'_k$ and $\varphi_0=\varphi$.


Let $c_k$ be the number of distinct $d$ in atoms of the form $d \ \big | \ \iterm$ plus the number of distinct coefficients of \emph{linear occurrences} of variables in $\varphi_k$. Let $s_k$ be the largest constant (including coefficients) and $a_k$ be the number of atomic formulas in $\varphi_k$. Similarly we define $c'_k$, $s'_k$ and $a'_k$ for $\varphi'_k$. And let $c,s,a$ be $c_0,s_0,a_0$ respectively. We will prove the following lemma.
\begin{lemma}\label{lem:cpx exp}
    \begin{align}
        c'_1&\le c^2 \notag \\
        s'_1&\le ms^2 \notag\\
        a'_1&\le sa\notag 
    \end{align} 
\end{lemma}

\begin{proof}
We analyze the complexity in the elimination of exponential occurrences of variable $\ivarx_1$. It will be discussed into 2 cases.

If all atoms are inequalities atomic formulas of the form in Lemma~\ref{lem:exp-ineq}. We know that each atomic formula $\tau$ with exponential occurrence of $\ivarx_1$ is replaced by a new formula. Only the coefficients of linear occurrences of $\ivarx_1$ and $\ivarx_2$ will be changed: constant coefficient $1$ is introduced, and if we substitute $x_1$ by $x_2+p$ for some $p$, coefficient of linear occurrence of $x_2$ will become $b_1+b_2$ ($b_1,b_2$ are coefficients for $x_1$ and $x_2$ in $\tau$, see Lemma~\ref{lem:exp-ineq} ). Since the new coefficient is obtained by adding two linear coefficient together, we have $c'_1\le c^2$. Note that $\delta$ and $B$ are at most $\ell_{10}(ms)$, when we substitute $\ivarx_1$ by $\ivarx_2+\delta-1$ or by $B-1$, the largest constant in the formula is at most $s\cdot  10^{\ell_{10}(ms)}\le ms^2$. And an inequality is replaced by at most $4\ell_{10}(ms)$ atomic formulas, so $a'_1\le 4\ell_{10}(ms)a$.

If all atoms are divisiblilty atomic formulas of the form $d \ \big | \ \iterm$. It is easy to see $c'_1<2c$ because a divisibility atomic formula  will produce at most two form of atomic formulas $d \ \big | \ \iterm$ and $\phi(d) \ \big | \ \iterm$. Note that any constant in $\iterm$ , say $l$, can be replaced by ($l \bmod d)$, so $s'_1\le s$. When $d$ is a large prime number, $\phi(d)=d-1$, a divisibility atomic formula is replaced by roughly $d$ atomic formulas, so $a'_1\le sa$ 

Choose larger upper bound for $c'_1$, $s'_1$ and $a'_1$ respectively then the lemma is proved.

\end{proof}

Since linear occurrences of $\ivarx_1$ are eliminated using Cooper's algorithm, we combine Oppen's analysis and obtain:
\begin{align}
    c_1&\le c^8 \notag \\
    s_1&\le (ms^2)^{4c^2} \notag\\
    a_1&\le (sa)^4(ms^2)^{2c^2}\notag 
\end{align} 

Assume $m\le n$, by induction on $k$ we can prove the following lemma.

\begin{lemma}
    \begin{align}
        c_k&\le c^{8^k} \notag \\
        s_k&\le n^{(4c)^{ 8^k}} s^{(8c)^{ 8^k}} \notag\\
        a_k&\le a^{4^k}n^{(4c)^{ 8^k}} s^{(8c)^{ 8^k}} \notag
    \end{align} 
\end{lemma}

Similarly to \cite{Oppen73}, we can assume $c\le n$, $a\le n$ and $s\le n$.So the space required to store $\varphi_k$ is the product of number of atoms $a_k$, the maximum number of constants $2m+2$ per atom, the maximum amount of space $s_k$ to store each constant and some constant $q$:
 
$${\sf space}\le q \cdot a_k \cdot (2m+2) \cdot s_k \le 2^{2^{2^{pn \log n}}}$$
for some large constant $p$. It is also a bound for deterministic time.
Since for arbitrary $\paexp$ formula length $n'$, the normalization for all variables results into a formula at most $n'^3$, so the bound is still 3-EXPTIME for $n'$.

\subsection{Some additional optimizations of the decision procedure for {\paexp}}\label{app-opt}

\paragraph{Synchronize the elimination of exponential occurrences of the same variable in different atomic formulas}

Although Lemma~\ref{lem:exp-ineq} is stated for a single atomic formula, the elimination of the exponential occurrences of the same variable in different atomic formulas can actually be synchronized. That is,  let $\alpha^\tau_{1}, \alpha^\tau_{2}, B^\tau, \delta^\tau$ be the constants as stated in the aforementioned under-approximation of an inequality $\tau$, define $\alpha^{\min}_1, \alpha^{\max}_2, B^{\max}, \delta^{\max}$ as the minimum of $\alpha^\tau_1$, the maximum of $\alpha^\tau_2$, the maximum of $B^\tau$, and the maximum of $\delta^\tau$ respectively with $\tau$ ranging over the inequalities of $\varphi$. Then we can use the same constants $\alpha^{\min}_1, \alpha^{\max}_2, B^{\max}, \delta^{\max}$ for different inequalities when eliminating the exponential occurrences of the same variable. 

\paragraph{Avoid the formula-size blow-up by depth-first search}

The {\pa} formula resulting from the elimination of exponential occurrences is essentially a big disjunction of the formulas of small size. If we store this big disjunction naively, then the formula size quickly blows up and exhausts the memory. Instead, we choose to do a depth-first search (DFS) and consider the disjuncts, which are of small sizes, one by one, and solve the satisfiability problem for these disjuncts. If during the search, a satisfiable disjunct is found, then the search terminates and ``SAT'' is reported.

\paragraph{Preprocess with small upper bound}

We believe that if a quantifier-free {\paexp} formula is satisfiable, then most probably it is satisfiable with small values assigned to variables. Consequently, as a preprocessing step, we put a small upper bound, e.g. the biggest constant occurring in the formula, on the values of variables, and perform a depth-first search, so that a model can be quickly found, if there is any. If this preprocessing is unsuccessful, then we continue the search with the greater upper bound $10^u$ for some proper $u \ge 1$.


\subsection{Detailed experiment results on the STRINGHASH benchmark suite}\label{app-exp}

% Please add the following required packages to your document preamble:
% \usepackage{multirow}
\begin{table}[ht]
    \caption{Experimental results on STRINGHASH,  O: Output, S:SAT, U: UNSAT, B: Bounded UNSAT, F: Fail, 
    $\#$: number of problems, $T$: average time in seconds}
    \centering

    \renewcommand{\arraystretch}{1.1}
    \begin{tabular}{|c|c|c|c|c|c|c|c|c|c|}
    \hline
        \multirow{2}{*}{Group }  & \multirow{2}{*}{O} & \multicolumn{2}{c|}{Z3} & \multicolumn{2}{c|}{CVC4} &  \multicolumn{2}{c|}{Trau} & \multicolumn{2}{c|}{$\paexp$} \\
        \cline{3-10}
       &  & $\#$ & $T$ &  $\#$ & $T$ & $\#$ & $T$ &  $\#$ & $T$ \\ 
       \hline
       \cline{1-10}
       \multirow{3}{*}{\scriptsize{$12345(w_1)^+(w_2)^+$}} & S & 5 & 14.0 & 29 & 8.5 & 3 & {\bf  0.1} & {\bf 37} & 9.9 \\
         \cline{2-10}
        & U & 0 & - & 0 & - & {\bf 60} & {\bf 1.3} & {\bf 60} & 47.2 \\
         \cline{2-10}
        & F & 95 & - & 71 & - & 37 & - & {\bf 3} & - \\ \hline
        \cline{1-10}
       \multirow{3}{*}{\scriptsize{$\begin{array}{l}12345(w_1)^+ \\
       \hspace{2mm}(w_2)^+6789\end{array}$}} & S & 11 & 13.0 & 29 & 12.0 & 0 & - & {\bf 37} & {\bf 10.6} \\
        \cline{2-10}
        & U & 0 & - & 0 & - & {\bf 63} & {\bf 1.2} & {\bf 63} & 50.0 \\
         \cline{2-10}
        & F & 89 & - & 71 & - & 37 & - & {\bf 0} & - \\ \hline
        \cline{1-10}
       \multirow{3}{*}{\scriptsize{$(w_1)^+(w_2)^+6789$}} & S & 18 & 24.0 & 30 & 9.3 & 2 & {\bf  0.1} & {\bf 41} & 16.1 \\
      	 \cline{2-10}
        & U & 0 & - & 1 & 4.0 & {\bf 59} & {\bf 2.5} & {\bf 59} & 45.8 \\
         \cline{2-10}
        & F & 82 &  & 69 &  & 39 &  & {\bf 0} &  \\ \hline
        \cline{1-10}
       \multirow{3}{*}{\scriptsize{$12345\Sigma^*_{\sf num}$}} & S & 82 & 8.7 & {\bf 100} & {\bf 2.2} & 28 & 5.9 & {\bf 100} & 18.5 \\
        \cline{2-10}
        & U & 0 & - & 0 & - & 0 & - & 0 & - \\
         \cline{2-10}
        & F & 18 & - & {\bf 0} & - & 72 & - & {\bf 0} & - \\ \hline
        \cline{1-10}
       \multirow{3}{*}{\scriptsize{$12345\Sigma^*_{\sf num}6789$}} & S & 60 & 9.3 & 17 & 7.8 & 3 & {\bf 0.3} & {\bf 100} & 16.0 \\
        \cline{2-10}
        & U & 0 & - & 0 & - & 0 & - & 0 & - \\
         \cline{2-10}
        & F & 40 & - & 83 & - & 97 & - & {\bf 0} & - \\ \hline
        \cline{1-10}
       \multirow{3}{*}{\scriptsize{$\Sigma^*_{\sf num}6789$}} & S & 68 & 5.5 & 27 & 13.0 & 24 & {\bf 9.0} & {\bf 100} & 15.7 \\
        \cline{2-10}
        & U & 0 & - & 0 & - & 0 & - & 0 & - \\
         \cline{2-10}
        & F & 32 & - & 73 & - & 76 & - & {\bf 0} &- \\
        \hline
       \end{tabular}
           \label{table:string}
\end{table}
